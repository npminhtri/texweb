\documentclass[11pt,letterpaper]{article}
\usepackage[lmargin=0.75in,rmargin=0.75in,tmargin=0.75in,bmargin=0.5in]{geometry}

% -------------------
% Packages
% -------------------
\usepackage{
	amsmath,			% Math Environments
	amssymb,			% Extended Symbols
	enumerate,		    % Enumerate Environments
	graphicx,			% Include Images
	lastpage,			% Reference Lastpage
	multicol,			% Use Multi-columns
	multirow,			% Use Multi-rows
  bbm,           % Mathbb for numbers
  amsthm
}
\usepackage[framemethod=TikZ]{mdframed}
\usepackage{tikz, tabularx}
\usepackage[table,x11names]{xcolor}
\usepackage{graphics}
\usepackage{hyperref}
\newcolumntype{W}{>{\centering\arraybackslash}X}%Para agilizar las columnas.
% -------------------
% Font
% -------------------
\usepackage[T1]{fontenc}
\usepackage{charter}


% -------------------
% Commands
% -------------------
\newcommand{\homework}[2]{\noindent\textbf{Name: Kyra Xian
}{} \hfill \textbf{} \\  \textbf{Due date: #2} \hfill \textbf{}\\}

\newcommand{\prob}{\noindent\textbf{Problem. }}
\newcounter{problem}
\newcommand{\problem}{
	\stepcounter{problem}%
	\noindent \textbf{Problem \theproblem. }%
}
\newcommand{\pointproblem}[1]{
	\stepcounter{problem}%
	\noindent \textbf{Problem \theproblem.} (#1 points)\,%
}
\newcommand{\pspace}{\par\vspace{\baselineskip}}
\newcommand{\ds}{\displaystyle}


% -------------------
% Theorem Environment
% -------------------
\mdfdefinestyle{theoremstyle}{%
	frametitlerule=true,
	roundcorner=5pt,
	linecolor=black,
	outerlinewidth=0.5pt,
	middlelinewidth=0.5pt
}
\mdtheorem[style=theoremstyle]{exercise}{\textbf{Problem}}


% -------------------
% Header & Footer
% -------------------
\usepackage{fancyhdr}

\fancypagestyle{pages}{
	%Headers
	\fancyhead[L]{}
	\fancyhead[C]{}
	\fancyhead[R]{}
\renewcommand{\headrulewidth}{0pt}
	%Footers
	\fancyfoot[L]{}
	\fancyfoot[C]{}
	\fancyfoot[R]{page \thepage \, of \pageref{LastPage}}
\renewcommand{\footrulewidth}{0.0pt}
}
\headheight=0pt
\footskip=14pt

\pagestyle{pages}

\DeclareMathOperator{\1}{\mathbbm{1}}
\DeclareMathOperator{\Log}{Log}

% -------------------
% Content
% -------------------
\begin{document}



% Question 1
\begin{exercise}\label{Galois theory }
  Let $F$ be a field of prime characteristic $p$. Suppose $E = F(\alpha)$ such that
  $\alpha \notin F$ but $\alpha^p -\alpha \in F$.
  \begin{enumerate}
    \item Find $[E:F]$.
    \item Prove that $E/F$ is a Galois extension.
    \item Find the Galois group $\text{Gal}(E/F)$.

          \textbf{Hint:} Note that $(x+1)^p -(x+1)=x^p -x$.
  \end{enumerate}
\end{exercise}
\begin{proof}

\end{proof}
\begin{exercise}
  Let $\zeta := e^{2\pi i/7}$ be a primitive 7th root of unity. Let $K = \mathbb{Q}(\zeta)$.
  \begin{enumerate}
    \item Prove that there exists an element $\alpha \in K$ such that  $[\mathbb{Q}(\alpha):\mathbb{Q}] = 2$.
    \item Express $\alpha$ in terms of $\zeta$.
  \end{enumerate}
\end{exercise}
\begin{proof}
  We will prove two items at once. Consider the element given by
  \[\alpha = \zeta + \zeta^2+\zeta^4\]
  Then it can be seen that the map $\sigma$ such that $\sigma$ such that $\sigma(\zeta) = \zeta^3$ generates the Galois group of the cyclotomic field. This implies the desired field extension will correspond
  to the fixed field of the subgroup generated by $\sigma^2$. We can see that the $\alpha$ defined as above
  is fixed by $\theta=\sigma^2$. Indeed
  \[\theta(\alpha) = \sigma^2(\zeta + \zeta^2+\zeta^4) = \zeta^9 + \zeta^4+\zeta^{36} = \zeta^2 + \zeta^4+\zeta = \alpha\]
  Clearly $\alpha \notin \mathbb{Q}$, since $\zeta$ has degree $6$ over $\mathbb{Q}$. Moreover, $\alpha$ can't have degree $3$ over $\mathbb{Q}$,
  otherwise it is inside the intersection of two intermediate fields of degree $2$ and $3$, thus is rational.
  Hence we can conclude that this is the desired element.
\end{proof}
\textbf{Remark:} A sage code for this problem is given below.
\begin{verbatim}
  k = CyclotomicField(7); k 
  zeta=k.gen(); a = zeta+zeta^2+zeta^4
  a.minpoly()
\end{verbatim}
\end{document}