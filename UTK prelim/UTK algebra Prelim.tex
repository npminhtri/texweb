\documentclass[11pt,letterpaper]{article}
\usepackage[lmargin=0.75in,rmargin=0.75in,tmargin=0.75in,bmargin=0.5in]{geometry}

% -------------------
% Packages
% -------------------
\usepackage{
	amsmath,			% Math Environments
	amssymb,			% Extended Symbols
	enumerate,		    % Enumerate Environments
	graphicx,			% Include Images
	lastpage,			% Reference Lastpage
	multicol,			% Use Multi-columns
	multirow,			% Use Multi-rows
  bbm,           % Mathbb for numbers
  amsthm
}
\usepackage[framemethod=TikZ]{mdframed}
\usepackage{tikz, tabularx}
\usepackage[table,x11names]{xcolor}
\usepackage{graphics}
\usepackage{hyperref}
\newcolumntype{W}{>{\centering\arraybackslash}X}%Para agilizar las columnas.
% -------------------
% Font
% -------------------
\usepackage[T1]{fontenc}
\usepackage{charter}


% -------------------
% Commands
% -------------------


\newcommand{\prob}{\noindent\textbf{Problem. }}
\newcounter{problem}
\newcommand{\problem}{
	\stepcounter{problem}%
	\noindent \textbf{Problem \theproblem. }%
}
\newcommand{\pointproblem}[1]{
	\stepcounter{problem}%
	\noindent \textbf{Problem \theproblem.} (#1 points)\,%
}
\newcommand{\pspace}{\par\vspace{\baselineskip}}
\newcommand{\ds}{\displaystyle}


% -------------------
% Theorem Environment
% -------------------
\mdfdefinestyle{theoremstyle}{%
	frametitlerule=true,
	roundcorner=5pt,
	linecolor=black,
	outerlinewidth=0.5pt,
	middlelinewidth=0.5pt
}
\mdtheorem[style=theoremstyle]{exercise}{\textbf{Problem}}


% -------------------
% Header & Footer
% -------------------
\usepackage{fancyhdr}

\fancypagestyle{pages}{

	\fancyhead[L]{}
	\fancyhead[C]{}
	\fancyhead[R]{}
\renewcommand{\headrulewidth}{0pt}
	%Footers
	\fancyfoot[L]{}
	\fancyfoot[C]{}
	\fancyfoot[R]{page \thepage \, of \pageref{LastPage}}
\renewcommand{\footrulewidth}{0.0pt}
}
\headheight=0pt
\footskip=14pt

\pagestyle{pages}

\DeclareMathOperator{\1}{\mathbbm{1}}
\DeclareMathOperator{\Log}{Log}

% -------------------
% Content
% -------------------
\begin{document}


\section{Galois Theory}
% Question 1
\begin{exercise}\label{Galois theory }
  Let $F$ be a field of prime characteristic $p$. Suppose $E = F(\alpha)$ such that
  $\alpha \notin F$ but $\alpha^p -\alpha \in F$.
  \begin{enumerate}
    \item Find $[E:F]$.
    \item Prove that $E/F$ is a Galois extension.
    \item Find the Galois group $\text{Gal}(E/F)$.

          \textbf{Hint:} Note that $(x+1)^p -(x+1)=x^p -x$. \footnote{This is the Artin-Schreier polynomial.}

  \end{enumerate}
\end{exercise}
\begin{proof}
  \hfill \\
  \begin{enumerate}
    \item This is the hardest part: Let's denote $b=\alpha^p - \alpha \in F$. Consider the
          polynomial $$f(x) = x^p - x - b.$$
          Clearly from the hint, we can see that $\alpha+k, k = 0,1,\ldots,p-1$ are roots of $f(x)$. They are all distinct.
          Thus
          \[f(x) = \prod_{k=0}^{p-1}(x-\alpha-k)\]
          If this polynomial is reducible over $F$, then there exists $n < p$ such that
          \[g(x) = \prod_{i=0}^{n-1}(x-\alpha-k_i) \in F[x]\]
          But this implies that the coefficient of $x^{n-1}$ in $g(x)$ is
          \[n\alpha + k_0+k_1+\ldots+k_{n-1} \in F\]
          which implies $\alpha \in F$, a contradiction. Thus $f(x)$ is irreducible over $F$.
    \item This follows immediate from part 1 that $f$ is irreducible and has $p$ distinct roots. Thus the splitting field of $f$ is $E$ and $E=F(\alpha)$
          is separable as $\alpha$ is separable over $F$. Thus $E/F$ is a Galois extension.
    \item The Galois group is a group of order $p$, thus it is isomorphism to $\mathbb{Z}/p\mathbb{Z}$.
  \end{enumerate}
\end{proof}
\begin{exercise}
  Let $\zeta := e^{2\pi i/7}$ be a primitive 7th root of unity. Let $K = \mathbb{Q}(\zeta)$.
  \begin{enumerate}
    \item Prove that there exists an element $\alpha \in K$ such that  $[\mathbb{Q}(\alpha):\mathbb{Q}] = 2$.
    \item Express $\alpha$ in terms of $\zeta$.
  \end{enumerate}
\end{exercise}
\begin{proof}
  We will prove two items at once. Consider the element given by
  \[\alpha = \zeta + \zeta^2+\zeta^4\]
  Then it can be seen that the map $\sigma$ such that $\sigma$ such that $\sigma(\zeta) = \zeta^3$ generates the Galois group of the cyclotomic field. This implies the desired field extension will correspond
  to the fixed field of the subgroup generated by $\sigma^2$. We can see that the $\alpha$ defined as above
  is fixed by $\theta=\sigma^2$. Indeed
  \[\theta(\alpha) = \sigma^2(\zeta + \zeta^2+\zeta^4) = \zeta^9 + \zeta^4+\zeta^{36} = \zeta^2 + \zeta^4+\zeta = \alpha\]
  Clearly $\alpha \notin \mathbb{Q}$, since $\zeta$ has degree $6$ over $\mathbb{Q}$. Moreover, $\alpha$ can't have degree $3$ over $\mathbb{Q}$,
  otherwise it is inside the intersection of two intermediate fields of degree $2$ and $3$, thus is rational.
  Hence we can conclude that this is the desired element.

  Another way  to do this problem is as follows. We have
  \[\alpha^2 = \zeta^2+\zeta^4+\zeta + 2(\zeta^3+\zeta^6+\zeta^5)=\zeta^2+\zeta^4+\zeta-2(1+\zeta^2+\zeta^4+\zeta)=-2-\alpha\]
  Thus we have the polynomial $x^2+x+2$ which is irreducible over $\mathbb{Q}$, since it has no rational roots. Thus we can conclude that $[\mathbb{Q}(\alpha):\mathbb{Q}] = 2$.
  \[\alpha^2 + \alpha + 2 = 0\]
  This yields the desired element $\alpha$.
\end{proof}
\textbf{Remark:} A Sage code for this problem is given below.
\begin{verbatim}
  k = CyclotomicField(7); k 
  zeta=k.gen(); a = zeta+zeta^2+zeta^4
  a.minpoly()
\end{verbatim}
\begin{exercise}
  Let \( K \) be a finite field of characteristic \( p \) with \( p^k \) elements. Suppose that \( F, L \) are subfields of \( K \) with \( |F| = p^n \) and \( |L| = p^m \). Also, suppose that \( |F \cap L| = p \). Prove that \( K = FL \) if and only if \( nm = k \).
\end{exercise}
\begin{proof}
  Since every finite extension of a finite field is Galois, and we have that
  \[\text{Gal}(FL/F) \cong \text{Gal}(L/L\cap F) = m\]
  In particular, we have $[FL:F] = m$. Thus
  \[[FL:L \cap F] = [FL:F][F:L\cap F ] = mn\]
  Hence $K = FL$ if and only if $mn = k$.  \footnote{  We can also prove this by using the uniqueness of finite extension of finite field of given order.}
\end{proof}
\begin{exercise}
  Let $E$ be a Galois extension of $\mathbb{Q}$ of order $2022$. Show that there exists 
  a cubic polynomial $f \in \mathbb{Q}[x]$ such that $f$ is irreducible and and has 3 distinct roots in $E$.
\end{exercise}
\begin{proof}
Note that we have 
\[2022 = 337 \times 2 \times 3\]
We will show that  the group of order $337$ is a normal subgroup. Indeed 
\end{proof}
\end{document}