\documentclass[11pt,letterpaper]{article}
\usepackage[lmargin=0.75in,rmargin=0.75in,tmargin=0.75in,bmargin=0.5in]{geometry}

% -------------------
% Packages
% -------------------
\usepackage{
	amsmath,			% Math Environments
	amssymb,			% Extended Symbols
	enumerate,		    % Enumerate Environments
	graphicx,			% Include Images
	lastpage,			% Reference Lastpage
	multicol,			% Use Multi-columns
	multirow,			% Use Multi-rows
  bbm,           % Mathbb for numbers
  amsthm
}
\usepackage[framemethod=TikZ]{mdframed}
\usepackage{tikz, tabularx}
\usepackage[table,x11names]{xcolor}
\usepackage{graphics}
\usepackage{hyperref}
\newcolumntype{W}{>{\centering\arraybackslash}X}%Para agilizar las columnas.
% -------------------
% Font
% -------------------
\usepackage[T1]{fontenc}
\usepackage{charter}


% -------------------
% Commands
% -------------------


\newcommand{\prob}{\noindent\textbf{Problem. }}
\newcounter{problem}
\newcommand{\problem}{
	\stepcounter{problem}%
	\noindent \textbf{Problem \theproblem. }%
}
\newcommand{\pointproblem}[1]{
	\stepcounter{problem}%
	\noindent \textbf{Problem \theproblem.} (#1 points)\,%
}
\newcommand{\pspace}{\par\vspace{\baselineskip}}
\newcommand{\ds}{\displaystyle}


% -------------------
% Theorem Environment
% -------------------
\mdfdefinestyle{theoremstyle}{%
	frametitlerule=true,
	roundcorner=5pt,
	linecolor=black,
	outerlinewidth=0.5pt,
	middlelinewidth=0.5pt
}
\mdtheorem[style=theoremstyle]{exercise}{\textbf{Problem}}


% -------------------
% Header & Footer
% -------------------
\usepackage{fancyhdr}

\fancypagestyle{pages}{

	\fancyhead[L]{}
	\fancyhead[C]{}
	\fancyhead[R]{}
\renewcommand{\headrulewidth}{0pt}
	%Footers
	\fancyfoot[L]{}
	\fancyfoot[C]{}
	\fancyfoot[R]{page \thepage \, of \pageref{LastPage}}
\renewcommand{\footrulewidth}{0.0pt}
}
\headheight=0pt
\footskip=14pt

\pagestyle{pages}

\DeclareMathOperator{\1}{\mathbbm{1}}
\DeclareMathOperator{\Log}{Log}

% -------------------
% Content
% -------------------
\begin{document}


\section{Galois Theory}
% Question 1
\begin{exercise}\label{Galois theory }
  Let $F$ be a field of prime characteristic $p$. Suppose $E = F(\alpha)$ such that
  $\alpha \notin F$ but $\alpha^p -\alpha \in F$.
  \begin{enumerate}
    \item Find $[E:F]$.
    \item Prove that $E/F$ is a Galois extension.
    \item Find the Galois group $\text{Gal}(E/F)$.

          \textbf{Hint:} Note that $(x+1)^p -(x+1)=x^p -x$. \footnote{This is the Artin-Schreier polynomial.}

  \end{enumerate}
\end{exercise}
\begin{proof}
  \hfill \\
  \begin{enumerate}
    \item This is the hardest part: Let's denote $b=\alpha^p - \alpha \in F$. Consider the
          polynomial $$f(x) = x^p - x - b.$$
          Clearly from the hint, we can see that $\alpha+k, k = 0,1,\ldots,p-1$ are roots of $f(x)$. They are all distinct.
          Thus
          \[f(x) = \prod_{k=0}^{p-1}(x-\alpha-k)\]
          If this polynomial is reducible over $F$, then there exists $n < p$ such that
          \[g(x) = \prod_{i=0}^{n-1}(x-\alpha-k_i) \in F[x]\]
          But this implies that the coefficient of $x^{n-1}$ in $g(x)$ is
          \[n\alpha + k_0+k_1+\ldots+k_{n-1} \in F\]
          which implies $\alpha \in F$, a contradiction. Thus $f(x)$ is irreducible over $F$.
    \item This follows immediate from part 1 that $f$ is irreducible and has $p$ distinct roots. Thus the splitting field of $f$ is $E$ and $E=F(\alpha)$
          is separable as $\alpha$ is separable over $F$. Thus $E/F$ is a Galois extension.
    \item The Galois group is a group of order $p$, thus it is isomorphism to $\mathbb{Z}/p\mathbb{Z}$.
  \end{enumerate} Hence we are done.
\end{proof}
\begin{exercise}
  Let $\zeta := e^{2\pi i/7}$ be a primitive 7th root of unity. Let $K = \mathbb{Q}(\zeta)$.
  \begin{enumerate}
    \item Prove that there exists an element $\alpha \in K$ such that  $[\mathbb{Q}(\alpha):\mathbb{Q}] = 2$.
    \item Express $\alpha$ in terms of $\zeta$.
  \end{enumerate}
\end{exercise}
\begin{proof}
  We will prove two items at once. Consider the element given by
  \[\alpha = \zeta + \zeta^2+\zeta^4\]
  Then it can be seen that the map $\sigma$ such that $\sigma$ such that $\sigma(\zeta) = \zeta^3$ generates the Galois group of the cyclotomic field. This implies the desired field extension will correspond
  to the fixed field of the subgroup generated by $\sigma^2$. We can see that the $\alpha$ defined as above
  is fixed by $\theta=\sigma^2$. Indeed
  \[\theta(\alpha) = \sigma^2(\zeta + \zeta^2+\zeta^4) = \zeta^9 + \zeta^4+\zeta^{36} = \zeta^2 + \zeta^4+\zeta = \alpha\]
  Clearly $\alpha \notin \mathbb{Q}$, since $\zeta$ has degree $6$ over $\mathbb{Q}$. Moreover, $\alpha$ can't have degree $3$ over $\mathbb{Q}$,
  otherwise it is inside the intersection of two intermediate fields of degree $2$ and $3$, thus is rational.
  Hence we can conclude that this is the desired element.

  Another way  to do this problem is as follows. We have
  \[\alpha^2 = \zeta^2+\zeta^4+\zeta + 2(\zeta^3+\zeta^6+\zeta^5)=\zeta^2+\zeta^4+\zeta-2(1+\zeta^2+\zeta^4+\zeta)=-2-\alpha\]
  Thus we have the polynomial $x^2+x+2$ which is irreducible over $\mathbb{Q}$, since it has no rational roots. Thus we can conclude that $[\mathbb{Q}(\alpha):\mathbb{Q}] = 2$.
  \[\alpha^2 + \alpha + 2 = 0\]
  This yields the desired element $\alpha$.
\end{proof}
\textbf{Remark:} A Sage code for this problem is given below.
\begin{verbatim}
  k = CyclotomicField(7); k 
  zeta=k.gen(); a = zeta+zeta^2+zeta^4
  a.minpoly()
\end{verbatim}
\begin{exercise}
  Let \( K \) be a finite field of characteristic \( p \) with \( p^k \) elements. Suppose that \( F, L \) are subfields of \( K \) with \( |F| = p^n \) and \( |L| = p^m \). Also, suppose that \( |F \cap L| = p \). Prove that \( K = FL \) if and only if \( nm = k \).
\end{exercise}
\begin{proof}
  Since every finite extension of a finite field is Galois, and we have that
  \[\text{Gal}(FL/F) \cong \text{Gal}(L/L\cap F) = m\]
  In particular, we have $[FL:F] = m$. Thus
  \[[FL:L \cap F] = [FL:F][F:L\cap F ] = mn\]
  Hence $K = FL$ if and only if $mn = k$.  \footnote{  We can also prove this by using the uniqueness of finite extension of finite field of given order.}
\end{proof}
\begin{exercise}
  Let $E$ be a Galois extension of $\mathbb{Q}$ of order $2022$. Show that there exists 
  a cubic polynomial $f \in \mathbb{Q}[x]$ such that $f$ is irreducible and and has 3 distinct roots in $E$.
\end{exercise}
\begin{proof}
Note that we have 
\[2022 = 337 \times 2 \times 3\]
\textbf{to be added}
\end{proof}
\section{Group theory}
\begin{exercise}
  Let $G$ be a finite group and $H$ be a proper subgroup. Suppose that $\gcd(|H|,[G:H])>1$. Show that there exists some $g \in G \setminus H$ such that $gHg^{-1} \cap H \ne \left\lbrace e\right\rbrace$. \end{exercise}
\begin{proof}
  \footnote{this solution is originally asked by me here: https://math.stackexchange.com/a/5063992/1231540}Anyway, one way to think about this problem is to consider the action of $H$ on $G/H$ by left multiplication.

The stabilizer of an element $gH \in G/H$ is equal to
\begin{align*}
\{h \in H : hgH = gH \} &= \{h \in H : g^{-1}hgH = H\} \\
&= \{h \in H : g^{-1}hg \in H \} \\
&= \{h \in H : h \in gHg^{-1}\} \\
&= gHg^{-1} \cap H,
\end{align*}
so we simply wish to show that some element of $G/H$, besides the trivial coset $eH$, has nontrivial stabilizer.

Well, suppose for contradiction that this is not the case, i.e. the stabilizer of $gH$ is $\{e\}$ whenever $g \notin H$. Then by the orbit-stabilizer theorem we have at most two types of orbits in the $H$-set $G/H$:
\begin{itemize}
  \item A single orbit of cardinality $1$, namely $\{eH\}$.
  \item Some number (say, $n$) of orbits of cardinality $\lvert H \rvert$
\end{itemize}

Since every $H$-set is the disjoint union of its orbits, we conclude that
$$[G : H] = \lvert G / H \rvert = 1 + n \lvert H \rvert$$
for some integer $n \geq 0$. This implies that $\gcd(\lvert H \rvert, [G:H]) = 1$, which is a contradiction.
\end{proof}
\newpage
\begin{exercise}
  Given $z=-1+i$, show that 
  \[2^n+1-z^n-\overline{z}^n\]
  is an integral multiple of 5.
\end{exercise}
\begin{proof}
First we make an elementary observation
\[z^8 = \overline{z}^8= 16 \equiv 1 \pmod 5\]
So, to check for the remainder when divided by 5 of $z^n+\overline{z}^n$, it is sufficient to look for the first seven value modulo 5. Let's look at the table
\begin{center}
\begin{tabular}{ c |c c c c c c c c  }
 n & 0 & 1 & 2 & 3 & 4 &5 & 6 & 7 \\ 
 \hline
 $z^n+\overline{z}^n \pmod 5$ & 2 & 3 & 0 & 4 & 2 & 3  &0 &4     
\end{tabular}
\end{center}
From the table, we can see a finer description - it is enough to consider $n$ modulo 4 to find the remainder when divided by 5. From Fermat's little theorem, for $n=4k+l$ with $0 \le l \le 3$
\[2^n+1 \equiv 2^l+1 \pmod 5\]
It is easy to check that 
\begin{center}
\begin{tabular}{ c |c c c c  }
 l & 0 & 1 & 2 & 3 \\ 
 \hline
 $2^l+1 \pmod 5$ & 2 & 3 & 0 & 4   
\end{tabular}
\end{center}
The two tables have the same values for $n$ modulo 4, thus the given expression must be divisible by 5.
\end{proof}
\end{document}
\subsection{$\rho$-definition of semi-stability}
There is another, more algebraic way to determine whether a given lattice is semi-stable.
This definition ultilize the notion of parabolic subgroups.







%Using definition \ref{lattice2}, we have $x = g\mathbb{Z}^n$ for some $g \in \text{GL}_n(\mathbb{R})$.
%Clearly for any sublattice of $x$, called $y$ corresponds to a sublattice of $\mathbb{Z}^n$ via

The precise definition
of a lattice is as follows:
\begin{definition}[\label = Euclidean $\mathbb{Z}$-lattices]
    Let $L$ be a finitely generated $\mathbb{Z}$-module. In particular, it is a free $\mathbb{Z}$-module
    of finite rank. Suppose that $P$ is endowed with a real-valued symmetric positive definite\footnote{The non-degenerate implicity state that rank L is the same as $dim L_\mathbb{R}$} bilinear form, called $Q$.
    Then the space $L_\mathbb{R}= L \otimes_\mathbb{Z} \mathbb{R}$ equipped with the bilinear form $Q$ forms a real
    inner product space. We will call  the pair $(L,Q)$ a \textbf{Euclidean $\mathbb{Z}$-lattice}.
\end{definition}
\todo{add proof showing $L$ is a lattice in the second definition}

If there is no further confusion, we can just denote a Euclidean lattice by $L$, without specifying the bilinear form
$Q$. The lattice $l$ determines a full-rank lattice inside $L_\mathbb{R}$, namely, the rank
of the lattice $L$ is equal to the dimension of $L_\mathbb{R}$. We first recall the definition of discrete subgroup
\begin{definition}
    Let $V$ be a finite-dimensional vector space over $\mathbb{R}$, endowed with the natural topology. A subgroup $L$ of the additive group underlying the vector space $V$ is said to be \textit{discrete} if each point $y$ in $L$ has a neighbourhood in $V$ whose intersection with $L$ is $\{y\}$ or, equivalently, if, given a bounded set $C$ in $V$, the set $C \cap L$ is finite.
\end{definition}
Thus, using the following Proposition, $L$ has a structure of a discrete subgroup $V = L_\mathbb{R}$.
\begin{prop}
    Given a finite-dimensional vector space $V$ over $\mathbb{R}$, let $L$ be a subgroup of the additive group $V$, and let $m$ be the dimension of the $\mathbb{R}$-span of $L$ in $V$. Then $L$ is a discrete subgroup if and only if $L$ is a free abelian group of rank $m$.
\end{prop}
A proof can be found in \cite{}.
We now can define the notion of covolume of a lattice:
\begin{definition}[\label = Volume]\label{volume}
    Let's assume that $L$ is a full-rank lattice and has a basis
    \[L = \mathbb{Z}l_1 \oplus \ldots \oplus\mathbb{Z}l_n\]
    Then the volume of this lattice is defined to be the volume of the fundamental
    parallepiped. In particular, let $\left\lbrace e_i\right\rbrace$ be any orthonormal
    basis of the vector space $V = L_\mathbb{R}$. Then
    \[\vol(L) := \left|\det Q(l_i,e_j)\right|\]
\end{definition}


Now, the basic problem we want to deal with is to classify "isomorphic" classes of lattice. Here
we say two lattices $L_1$ and $L_2$ are isomorphic if and only if there is a map $\gamma \in \glnz$ such that
\[\gamma \cdot g_1 = g_2,\]
From the first point of view, we identify $L_i$ with the $\mathbb{Z}-$ module
$\mathbb{Z}^n$ associated to the form $g_i^tg_i$. If we define $X_n$ the space of all
symmetric positive definite bilinear forms, then we are looking at the space $\glnz \backslash X_n$. We can also
regard $L_i \otimes \mathbb{R} \cong \mathbb{R}^n$. From this point of view,
the problem of classification isomorphic classes of lattices is the same as looking for discrete
subgroups of $\mathbb{R}^n$ of rank $n$, modulo rotation. We will interchange these equivalent points of view
depend on the situation.

As Bill Casselman note in his expository, even if we normalize the lattice to get a unimodular
lattice, we will still have to work with arbitrary lattices in the smaller rank. This means we are embedding
several copies of $\text{GL}_m(\mathbb{Z})$ along the diagonal of $\slnz$.
\section{Semi-stable lattices: two definitions}
\subsection{Grayson's definition of semi-stable lattice}
In this section, we introduce the idea of Grayson in defining \textit{semi-stable} lattices.
In particular, he associates every lattices a plot and its convex hull - called \textit{ profiles}. To understand
what this means, we must first introduce the notation of \textit{sublattice}.
\todo{add proof}
An easy observation is that, if $M \subset L$ is a sublattice, then the space $M_\mathbb{R} = M \otimes \mathbb{R}$
is a subspace of $L_\mathbb{R}$, equipped with the restriction of the positive definite symmetric form $Q$ of $L$,hence $M$
is also a lattice of rank not exceeding rank of $L$.

As stated in definition \ref{volume}, we can computed a volume of a lattice by base changing and
choose an orthonormal basis. However, if we view the lattice $L$ as $\mathbb{Z}^n$ under an action
of $g \in \text{GL}_n(\mathbb{R})$ as in definition \ref{lattice2}, it is more convenient to define volume use wedge  product.
Suppose $L$ has rank $n$, then $L$ has a basis $b_1,b_2,\ldots,b_n$ such that
\[b_i = g \cdot e_i, \quad g \in \text{GL}_n(\mathbb{R}),\]
where $e_1,e_2,\ldots,e_n$ is the standard basis of $\mathbb{R}^n$. let $\bigwedge^* \mathbb{R}^n$ denote the corresponding exterior algebra. If $\bigwedge^p \mathbb{R}^n$ denotes the $p$th exterior power of $\mathbb{R}^n$, the products $f_v := e_{v_1} \wedge \cdots \wedge e_{v_p}$, where $v$ ranges over the ordered $p$-tuples $(v_1, \ldots, v_p)$ subject to the condition $1 \leq v_1 < \cdots < v_p \leq n$, then form a basis $\{f_v\}$ of $\bigwedge^p \mathbb{R}^n$. In a natural way, $\bigwedge^p \mathbb{R}^n$ permits the Euclidean norm, $\lVert \cdot \rVert$ defined by $\lVert f_v \rVert = 1$ and $\lVert \sum_v \lambda_v f_v \rVert = \left( \sum_v \lambda_v^2 \right)^{1/2}$.

The group $\mathrm{GL}_n(\mathbb{R})$ operates on the exterior power $\bigwedge^p \mathbb{R}^n$, $p = 1, \ldots, n$, via \[g(e_{v_1} \wedge \cdots \wedge e_{v_p}) = g(e_{v_1}) \wedge \cdots \wedge g(e_{v_p})\] and linear extension.
So if $M$ is a sublattice of $L$ with basis $l_1,\ldots,l_m$ then the volume
of $M$ is the length of the vector $l_1 \wedge \ldots \wedge l_m$. In other words, it is the square root
of the sum of the squares of the determinants of the $m \times m$ minor matrices in the $n \times m$ matrix whose columns are the coordinate of the $l_i$ with respect to any orthonormal basis of $L_\mathbb{R}$.
\todo{add a numerical example.}
Now we are ready to define the \textbf{canonical plot.}

For example, if $M$ is of rank 1, then the volume is just the length of the generator.


The above definition in some sense "measures" the volume of the sublattices of the lattice $x$.
We will illustrate with using Iwasawa's decomposition for $n=3$. Recall that for any $g \in \text{SL}_n(\mathbb{R})$ we have
\[ g = k_ga_gn_g \in K \times A \times N,\]
where $K$ is the special orthogonal subgroup, $A$ is the set of all diagonal matrices with all positive entries along the diagonal
and $N$ is the subgroup of unipotent matrices. If we let $ x\gamma = g = k_ga_gn_g$ and
\[a_g = \begin{bmatrix}
        a_1 & 0   & 0   \\
        0   & a_2 & 0   \\
        0   & 0   & a_3
    \end{bmatrix}\]
Then
\begin{align*}
    \min_{\gamma \in \text{SL}_n(\mathbb{Q})/Q_1(\mathbb{Q})}\left\langle \rho_{Q_1}, H_{Q_1}(x\gamma) \right\rangle = \log(a_1) \\ \min_{\gamma \in \text{SL}_n(\mathbb{Q})/Q_2(\mathbb{Q})}\left\langle \rho_{Q_2}, H_{Q_2}(x\gamma) \right\rangle = \log(a_1a_2)
\end{align*}
So a lattices $x$ in $K \backslash\text{SL}_3(\mathbb{R})$  is semi-stable if and only if the $\log(a_1)$ and $\log(a_1a_2)$ defined as above are nonnegative.
So far we have two distinct definitions of semi-stability. The following theorem asserts that they are equivalent:
\begin{prop}
    Let $x \in X_n = K \backslash \text{SL}_n(\mathbb{R})$ - the space of unit lattice. Then $x$ is semi-stable if one of the following equaivalent
    conditions holds
    \begin{enumerate}
        \item The bottom of the profile of $x$ is a line connect solely two points: the origin and $(n,0)$.
        \item The degree of instability of $x$ is nonnegative, namely, $\deg_{inst}(x) \ge 0$.
    \end{enumerate}
\end{prop}
\begin{remark}
    Let denote $v_i = e_1 \wedge \ldots \wedge e_i$. Clearly we have
    \[||k_g(v)|| = ||v|| \quad \text{ and } \quad n_g(v_i) =v_i\]
    for all $v \in \bigwedge^i \mathbb{R}^n, i = 1,\ldots n$. Thus, in term of volume, the only
    relevant factor is the $A$-component. In particular, we have
    \[||g(v_i)|| =||a_g(v_i)|| = ||a_ge_i\wedge \ldots a_ge_n|| = a_{11}\ldots a_{ii} \]
\end{remark}
The above remark suggests that for each maximal standard parabolic subgroups, the degree of instability detects
for the sublattices with the smallest volume. So if we can prove there is a correspondence between
$\gamma \in \text{SL}_n(\mathbb{Q})/Q_i(\mathbb{Q})$ and a sublattice of rank $i$ of $x$, then we are done.
\begin{proof}
    \hfill \\
    We first need a slight reduction - we identified the quotient $\text{GL}_n(\mathbb{Q})/Q_i(\mathbb{Q})$ with
    the quotient $\text{GL}_n(\mathbb{Z})/(Q_i(\mathbb{Q}) \cap \text{GL}_n(\mathbb{Z})) $. Now let $x$ be an arbitrary lattice of rank $n$.

    We will show the following correspondence
    \[ \text{GL}_n(\mathbb{Z})/(Q_i(\mathbb{Q}) \cap \text{GL}_n(\mathbb{Z})) \longleftrightarrow \left\lbrace \text{ sublattice of rank $i$ of $\mathbb{Z}^n$}\right\rbrace\]
    Indeed, let $M \subset \mathbb{Z}^n$ be a sublattice with a basis $v_1,v_2,\ldots,v_i$. Since $M$ is a sublattice, we can extend the basis to $
        \left\lbrace v_1,\ldots,v_i,v_{i+1},\ldots,v_n\right\rbrace$ to get a basis of $\mathbb{Z}^n$. Let $\left\lbrace e_1,\ldots,e_n\right\rbrace$ be the standard basis
    of $\mathbb{Z}^n$, there clearly there exsists a $\gamma \in \text{SL}_n(\mathbb{Z})$ such that
    \[\gamma \cdot e_i = v_i \quad \forall i =1,\ldots,n\]
    If we identify $M$ with the wedge product $v_1\wedge \ldots \wedge v_n$, then $ \gamma \in \text{GL}_n(\mathbb{Z})$ acts on the exterior product via
    \[\gamma (e_1\wedge \ldots \wedge e_i) = v_1\wedge \ldots \wedge v_i,\]

    So the set of rank $i$ sublattice of $\mathbb{Z}^n$ is the orbit of $e_1\wedge \ldots \wedge e_n$
    under the action of $ \text{GL}_n(\mathbb{Z})$, with the stabilizer as $(Q_i(\mathbb{Q}) \cap \text{GL}_n(\mathbb{Z}))$.
    The correspondence follows from the Orbit - Stabilizer theorem.

    An immediate consequence of the above correspondence is that
    \[ x\text{GL}_n(\mathbb{Z})/(Q_i(\mathbb{Q}) \cap \text{GL}_n(\mathbb{Z})) \longleftrightarrow \left\lbrace \text{ sublattice of rank $i$ of $x\mathbb{Z}^n$}\right\rbrace\]
    So we constructed a bijection between the maximal parabolic subgroups $Q_i$'s and  sublattices of rank $i$ in
    any lattice.

    Now let $L = x\mathbb{Z}^n$  be a lattice. Then $L$ satisfies condition (1) if and only if
    for any sublattice $M$, the slope $\mu(M)$ is non-negative. But this is equivalent to $\log(\vol(M)) \ge 0$.
    Let assume $M$ is of rank $i$. Then the above correspondence show that there must be a $\gamma_M \in \text{GL}_n(\mathbb{Z})$ such that
    \[x\gamma_M(e_1\wedge \ldots\wedge e_i) = l_1\wedge \ldots\wedge l_i,\]
    where $\left\lbrace l_1,\ldots,l_i\right\rbrace$ forms a basis of $M$.
    Using the Iwasawa decomposition, we get
    \[\min\left\langle \rho_{Q_i},H_{Q_i}(x\gamma)\right\rangle = \min\log(\vol(M))\]
    In particular, when taking the minimum all over the parabolic subgroups, we are looking at the volume
    all the sublattices of $L$ of smaller rank, so
    \[\deg_{inst}(x) \ge 0 \Leftrightarrow \min\left\langle \rho_{Q_i},H_{Q_i}(x\gamma)\right\rangle \ge 0  \quad \forall i \Leftrightarrow \mu(M) \ge 0 \quad \text{for all sublattices $M$}\]
    Thus we proved two definitions are equivalent.
\end{proof}


%tikzpicture for lattice
\begin{tikzpicture}
    % Define the lattice range
    \def\xmin{-3}
    \def\xmax{3}
    \def\ymin{-2}
    \def\ymax{2}

    % Draw the lattice points
    \foreach \x in {\xmin,...,\xmax} {
            \foreach \y in {\ymin,...,\ymax} {
                    \fill (\x,\y) circle (2pt);
                }
        }

    % Draw the coordinate axes
    \draw[->, thick] (\xmin-0.5,0) -- (\xmax+0.5,0) node[right] {$x$};
    \draw[->, thick] (0,\ymin-0.5) -- (0,\ymax+0.5) node[above] {$y$};

    % Label the origin
    \node[below left] at (0,0) {$O$};
\end{tikzpicture}





\begin{remark} %about the volume of sublattice

    If we view the lattice $L$ as $\mathbb{Z}^n$ under an action
    of $g \in \text{GL}_n(\mathbb{R})$ as in definition \ref{lattice2}, it is more convenient to define volume use wedge  product.
    Suppose $L$ has rank $n$, then $L$ has a basis $b_1,b_2,\ldots,b_n$ such that
    \[b_i = g \cdot e_i, \quad g \in \text{GL}_n(\mathbb{R}),\]
    where $e_1,e_2,\ldots,e_n$ is the standard basis of $\mathbb{R}^n$. let $\bigwedge^* \mathbb{R}^n$ denote the corresponding exterior algebra. If $\bigwedge^p \mathbb{R}^n$ denotes the $p$th exterior power of $\mathbb{R}^n$, the products $f_v := e_{v_1} \wedge \cdots \wedge e_{v_p}$, where $v$ ranges over the ordered $p$-tuples $(v_1, \ldots, v_p)$ subject to the condition $1 \leq v_1 < \cdots < v_p \leq n$, then form a basis $\{f_v\}$ of $\bigwedge^p \mathbb{R}^n$. In a natural way, $\bigwedge^p \mathbb{R}^n$ permits the Euclidean norm, $\lVert \cdot \rVert$ defined by $\lVert f_v \rVert = 1$ and $\lVert \sum_v \lambda_v f_v \rVert = \left( \sum_v \lambda_v^2 \right)^{1/2}$.

    The group $\mathrm{GL}_n(\mathbb{R})$ operates on the exterior power $\bigwedge^p \mathbb{R}^n$, $p = 1, \ldots, n$, via \[g(e_{v_1} \wedge \cdots \wedge e_{v_p}) = g(e_{v_1}) \wedge \cdots \wedge g(e_{v_p})\] and linear extension.
    So if $M$ is a sublattice of $L$ with basis $l_1,\ldots,l_m$ then the volume
    of $M$ is the length of the vector $l_1 \wedge \ldots \wedge l_m$. In other words, it is the square root
    of the sum of the squares of the determinants of the $m \times m$ minor matrices in the $n \times m$ matrix whose columns are the coordinate of the $l_i$ with respect to any orthonormal basis of $L_\mathbb{R}$.
\end{remark}

%%Parabolic 
\tpoint{ Parabolic $k$-subgroups}\todo{move to chapter II?}
We will first recall what
is a $k-$ parabolic subgroups for the general linear group $\text{GL}_n$ for $n \ge 2$, over an arbitrary
field $k$. Let $e_1,e_2,\ldots,e_n$ be a standard basis for the vector space $k^n$. From linear algebra,
we know that each linear map $T \colon k^n \to k^n$ can be identified with a $n \times n$ matrix. In particular
we obtain an identification between the group $\text{GL}_n(k)$ with $\text{GL}(k^n)$ of $k-$ automorphisms of
$k^n$.
\begin{definition}
    A flag $\mathcal{F}$ of $k^n$ is a chain of linear subspaces
    \[\mathcal{F} \colon 0 \subset F_1 \subset F_2 \subset \ldots \subset F_r \subset k^n\]
    Let $d_i = \dim F_i$, then we call the ordered $r$-tuple $(d_1,d_2,\ldots,d_r)$  the \textit{type} of the flag $\mathcal{F}$.
\end{definition}
A parabolic subgroup of $\text{GL}(k^n)$ is the stabilizer $P_\mathcal{F} = P$ of a flag $\mathcal{F}$.
A parabolic subgroup $P$ is call \textit{minimal} if it stabilizes a flag of type
$(1,2,\ldots,n)$.

Let \( e_1, \ldots, e_n \) be the standard basis for the \( k \)-vector space \( k^n \). For any \( 1 \leq i \leq n \), define \( V_i \) to be \( e_1 + \cdots + e_i \). We call a flag \( \mathcal{V} \) by the chain
\[
    \mathcal{V}: 0 \subset V_{d_1} \subset V_{d_2} \subset \cdots \subset V_{d_s} \subset k^n,
\]
a standard flag in \( k^n \). Let \( d_0 = 0 \) and \( d_{s+1} = n \). We define \( r_j := d_j - d_{j-1} \), where \( j = 1, \ldots, s+1 \). Then \( \rho = (r_1, \ldots, r_{s+1}) \) is an ordered partition of \( n \) into positive integers, i.e., an ordered sequence of positive integers so that \( r_1 + \cdots + r_{s+1} = n \). The corresponding standard parabolic subgroup \( P_{\mathcal{V}} := P_{\rho} \) consists of all matrices in \( \text{GL}_n(k) \) admitting a block decomposition whose diagonal blocks are \( (r_j \times r_j) \)-matrices in \( \text{GL}_{r_j}(k) \), \( j = 1, \ldots, r+1 \), the lower entries are 0, and the other entries are arbitrary. Every parabolic subgroup of \( \text{GL}_n(k) \) is conjugate to a subgroup of this type.


\item For  $\text{GL}_4(\mathbb{R})$, there are seven standard parabolic subgroups for seven partitions
              \[ 4 = 1+ 1+1+1, \quad 4 =1+1+2, \quad 4 = 1+2+1\]
              \[4 =2+1+1, \quad 4 = 1+3, \quad 4 = 2+1, \quad 4 = 3+1\]
              Explicitly, we have the following subgroups
              \begin{align*}
                   & P_{1,1,1,1} = \left\lbrace \begin{bmatrix}
                                                    \ast & \ast & \ast & \ast \\
                                                    0    & \ast & \ast & \ast \\
                                                    0    & 0    & \ast & \ast \\
                                                    0    & 0    & 0    & \ast \\
                                                \end{bmatrix} \right\rbrace, \quad & P_{1,1,2} = \left\lbrace \begin{bmatrix}
                                                                                                                  \ast & \ast & \ast & \ast \\
                                                                                                                  0    & \ast & \ast & \ast \\
                                                                                                                  0    & 0    & \ast & \ast \\
                                                                                                                  0    & 0    & \ast & \ast \\
                                                                                                              \end{bmatrix} \right\rbrace \\
                   & P_{1,2,1} = \left\lbrace \begin{bmatrix}
                                                  \ast & \ast & \ast & \ast \\
                                                  0    & \ast & \ast & \ast \\
                                                  0    & \ast & \ast & \ast \\
                                                  0    & 0    & 0    & \ast \\
                                              \end{bmatrix} \right\rbrace, \quad   & P_{2,1,1} = \left\lbrace \begin{bmatrix}
                                                                                                                  \ast & \ast & \ast & \ast \\
                                                                                                                  \ast & \ast & \ast & \ast \\
                                                                                                                  0    & 0    & \ast & \ast \\
                                                                                                                  0    & 0    & 0    & \ast \\
                                                                                                              \end{bmatrix} \right\rbrace \\
                   & P_{1,2,1} = \left\lbrace \begin{bmatrix}
                                                  \ast & \ast & \ast & \ast \\
                                                  0    & \ast & \ast & \ast \\
                                                  0    & \ast & \ast & \ast \\
                                                  0    & 0    & 0    & \ast \\
                                              \end{bmatrix} \right\rbrace, \quad   & P_{2,1,1} = \left\lbrace \begin{bmatrix}
                                                                                                                  \ast & \ast & \ast & \ast \\
                                                                                                                  \ast & \ast & \ast & \ast \\
                                                                                                                  0    & 0    & \ast & \ast \\
                                                                                                                  0    & 0    & 0    & \ast \\
                                                                                                              \end{bmatrix} \right\rbrace \\
                   & P_{1,3} = \left\lbrace \begin{bmatrix}
                                                \ast & \ast & \ast & \ast \\
                                                0    & \ast & \ast & \ast \\
                                                0    & \ast & \ast & \ast \\
                                                0    & \ast & \ast & \ast \\
                                            \end{bmatrix} \right\rbrace, \quad     & P_{3,1} = \left\lbrace \begin{bmatrix}
                                                                                                                \ast & \ast & \ast & \ast \\
                                                                                                                \ast & \ast & \ast & \ast \\
                                                                                                                \ast & \ast & \ast & \ast \\
                                                                                                                0    & 0    & 0    & \ast \\
                                                                                                            \end{bmatrix} \right\rbrace   \\
                   & P_{2,2} = \left\lbrace \begin{bmatrix}
                                                \ast & \ast & \ast & \ast \\
                                                \ast & \ast & \ast & \ast \\
                                                0    & 0    & \ast & \ast \\
                                                0    & 0    & \ast & \ast \\
                                            \end{bmatrix} \right\rbrace
              \end{align*}
              Clearly $\textbf{MaxParSt} = \left\lbrace P_{1,3},P_{3,1},P_{2,2}\right\rbrace.$