\documentclass[11pt,letterpaper]{article}
\usepackage[lmargin=0.75in,rmargin=0.75in,tmargin=0.75in,bmargin=0.5in]{geometry}

% -------------------
% Packages
% -------------------
\usepackage{
	amsmath,			% Math Environments
	amssymb,			% Extended Symbols
	enumerate,		    % Enumerate Environments
	graphicx,			% Include Images
	lastpage,			% Reference Lastpage
	multicol,			% Use Multi-columns
	multirow,			% Use Multi-rows
  bbm,           % Mathbb for numbers
  amsthm
}
\usepackage[framemethod=TikZ]{mdframed}
\usepackage{tikz, tabularx}
\usepackage{graphics}
\usepackage{hyperref}
\newcolumntype{W}{>{\centering\arraybackslash}X}%Para agilizar las columnas.
% -------------------
% Font
% -------------------
\usepackage[T1]{fontenc}
\usepackage{charter}


% -------------------
% Commands
% -------------------
\newcommand{\homework}[2]{\noindent\textbf{Name: }{} \hfill \textbf{} \\  \textbf{Due date: #2} \hfill \textbf{}\\}

\newcommand{\prob}{\noindent\textbf{Problem. }}
\newcounter{problem}
\newcommand{\problem}{
	\stepcounter{problem}%
	\noindent \textbf{Problem \theproblem. }%
}
\newcommand{\pointproblem}[1]{
	\stepcounter{problem}%
	\noindent \textbf{Problem \theproblem.} (#1 points)\,%
}
\newcommand{\pspace}{\par\vspace{\baselineskip}}
\newcommand{\ds}{\displaystyle}


% -------------------
% Theorem Environment
% -------------------
\mdfdefinestyle{theoremstyle}{%
	frametitlerule=true,
	roundcorner=5pt,
	linecolor=black,
	outerlinewidth=0.5pt,
	middlelinewidth=0.5pt
}
\mdtheorem[style=theoremstyle]{exercise}{\textbf{Problem}}


% -------------------
% Header & Footer
% -------------------
\usepackage{fancyhdr}

\fancypagestyle{pages}{
	%Headers
	\fancyhead[L]{}
	\fancyhead[C]{}
	\fancyhead[R]{}
\renewcommand{\headrulewidth}{0pt}
	%Footers
	\fancyfoot[L]{}
	\fancyfoot[C]{}
	\fancyfoot[R]{page \thepage \, of \pageref{LastPage}}
\renewcommand{\footrulewidth}{0.0pt}
}
\headheight=0pt
\footskip=14pt

\pagestyle{pages}

\DeclareMathOperator{\1}{\mathbbm{1}}

% -------------------
% Content
% -------------------
\begin{document}
\homework{\#}{02/14}


% Question 1
\begin{exercise}
  Let $\chi_0$ be the principal character mod $3$, that is
  \[\chi_0 (n) = \begin{cases}
      1 , \quad \gcd(n,3)=1 \\
      0, \quad \text{otherwise}
    \end{cases}\]
  Define
  \[\chi(n) = \begin{cases}
      1, \quad  & n \equiv 1 \pmod 3 \\
      -1, \quad & n \equiv 2 \pmod 3 \\
      0, \quad  & \text{otherwise}
    \end{cases}\]
  Prove that $\chi_0$ and $\chi$ are completely multiplicative functions. Verify the identity
  \[ \mathbbm{1} = \dfrac{1}{2}(\chi_0 +\chi)\]
\end{exercise}
\begin{proof}
  \hfill

  Let $m,n \in \mathbb{Z}$ be arbitrary integers. We consider the following cases:
  \begin{enumerate}
    \item At least one of $m,n$ is divisible by $3$. WLOG, we can assume $3 \mid m$. Then it is clear that $3 \mid mn$.
          By definition, we must have $\chi_0(m) = \chi(m) =0$ and $\chi_0(mn) =\chi(mn)=0$. Thus we have
          \[\chi(m)\chi(n) = 0\cdot \chi(n) = 0 = \chi(mn) \quad \text{ and }\chi_0(m)\chi_0(n) = 0\cdot \chi(n) = 0 = \chi(mn).\]
    \item Both $m,n$ are coprime to $3$. There will be then two subcases
          \begin{itemize}
            \item $m \equiv n \pmod 3$. Then clearly we have $\chi_0(m) = \chi_0(n)=1$ and $\chi(m)=\chi(n)$. Moreover
                  $mn \equiv 1 \pmod 3$, thus $\chi(mn)=\chi_0(mn)=1$. In particular, we have
                  \[\chi(m)\chi(n) = 1\cdot 1 = 1 = \chi(mn) \quad \text{ and }\chi_0(m)\chi_0(n) = 1\cdot 1 = 1 = \chi(mn).\]
            \item $m \not\equiv n \pmod 3$. We can further assume that $ m \equiv 1 \pmod 3$ and $n \equiv 2 \pmod 3$. clearly
                  \[\chi_0(m)\chi_0(n) = 1\cdot 1 = 1 = \chi_0(mn), \]
                  as $\gcd(mn,3)=1$.

                  On the other hand, $mn \equiv 1 \cdot (-1)\equiv 2 \pmod 3$, thus
                  \[\chi(m)\chi(n) = 1\cdot (-1) = -1 = \chi(mn) \]
          \end{itemize}
          In conclustion, $\chi$ and $\chi_0$ are completely multiplicative. To verify the given identity, we consider
          the following cases
          \begin{enumerate}
            \item $m \not\equiv 1 \pmod 3$: Then $\mathbbm{1}(m) = 0$. On the other hand, we have
                  \[\dfrac{1}{2}\left(\chi(m)+\chi_0(m)\right)= \begin{cases}
                      0+0, \quad m \equiv 0 \pmod 3 \\
                      -1+1, \quad m \equiv 2 \pmod 3
                    \end{cases} = 0\]
            \item $m \equiv 1 \pmod 3$: Then $\mathbbm{1}(m) =1 =1/2(\chi(m)+\chi_0(m))$
          \end{enumerate}
  \end{enumerate}
  Thus we are done.
\end{proof}


% Question 2
\begin{exercise}\label{ex2}
  Let $G$ be a finite abelian group. Show that $\#\hat{G} \le \#G$.
\end{exercise}
\begin{proof}
  First we will prove that the set $S$ of maps $f\colon G \to \mathbb{C}$ has a vector space structure.
  The sum and the scalar multiplication of maps are defined as follows
  \begin{itemize}
    \item $(\chi_1 + \chi_2)(a) := \chi_1(a)+ \chi_2(a)$ for all $a \in G$.
    \item $(c\chi)(a):= c\cdot\chi(a)$ for any $\chi \in S$ and $c \in \mathbb{C}$.
  \end{itemize}
  But we can check all the axioms for a set being a vector space pointwisely, with the zero
  beging the zero map. So $S$ have a a structure of vector space. Let's compute the dimension of $S$ as
  as $\mathbb{C}-$ vector space. Since $G$ is finite, we can assume that, as a set
  \[G = \left\lbrace a_1,\ldots,a_n \right\rbrace\]
  Then we can define the map $f_i$ to be the "dual" of $a_i$ in the following sense: $f_i(a_j) =\delta_{ij}$, i.e.
  $f_i(a_i)=1$ and vanishes at $a_j \ne a_i$. We claim that $\mathfrak{B}= \left\lbrace f_i \right\rbrace$ forms a basis of $S$.
  Indeed, let $f\colon G \to \mathbb{C}$ be arbitrary. Since $G$ is finite, $f$ is determined by its value at each $a_i \in G$.
  Assume that $c_i = f(a_i)$. Then for any $1 \le i \le n$, we have
  \[f(a_i) = c_i = \sum_{i=1}^n c_i f_i(a_i)\]
  In particular, we can rewrite the following identity as
  \[f = c_1f_1+c_2f_2+\ldots+c_nf_n,\]
  which implies $\mathfrak{B}$ spans $S$. On the other hand, if
  \[ 0 = c_1f_1+\ldots+c_nf(n)\]
  Applying both sides to $a_i$ yields
  \[0 = c_if_i(a_i) = c_i\]
  which means $\mathfrak{B}$ is a set of linearly independent vectors. Thus $\dim S = \#\mathfrak{B} = n$.
  In the note, we proved that the character $\chi_1,\chi_2,\ldots,\chi_m$ are mutually orthogonal. In particular, they are
  a subset of $S$ comprising of linearly independent vectors. Thus
  \[m = \#\hat{G} \le n = \# G. \]
  \textbf{Remark:} In fact, if we assume the theorem about the structure of finite abelian group, we can prove that
  the equality always happens, namely $ \#\hat{G} =\# G$
\end{proof}
\newpage






%Question 3
\begin{exercise}
  \label{ex3}
  Consider the finite abelian group $G_9$ consisting of the invertible elements of $\mathbb{Z}/9\mathbb{Z}$. Find all the characteres of $G_9$.
  Make sure you prove that the characters of $G_9$ you find are all distinct and there are no others.
\end{exercise}
\begin{proof}
  Using the Remark in the previous exercise, we predict that there are $6$ characters in total. It can be checked easily that $2$ generates $G_9$, so it is a
  cyclic group of order $6$. Indeed, we have
  \begin{align*}
    2^1 \equiv 2 \pmod 9 \\
    2^2 \equiv 4 \pmod 9 \\
    2^3 \equiv 8 \pmod 9 \\
    2^4 \equiv 7 \pmod 9 \\
    2^5 \equiv 5 \pmod 9 \\
    2^6 \equiv 1 \pmod 9
  \end{align*}
  So the character is defined solely by determining where it sends $[2]$ in $\mathbb{C}$. Let $\chi \in \hat{G_9}$ be
  any character, we then have
  \[\chi(2)^6=\chi(2^6) = \chi(1) = 1 \in \mathbb{C}\]
  thus implies $\chi(2)$ must be a $6-th$ root of unity. There are $6$ choices in total, namely
  \[\chi(2) = e^{\frac{2i\pi k}{6}}, 0 \le k \le 6\]
  Clearly these are 6 distinct characters, and they are all possible characters of $G_9$.
\end{proof}

%Question 4
\begin{exercise}
  Show that there is exactly one real, non-principal Dirichlet character $\chi \pmod 9$. Find $\chi(916)$.
\end{exercise}
\begin{proof}
  Continueing from the previous exercise, with a note that a character is called real character if its image lies entirely in
  $\mathbb{R}$, namely  we have a group homomorphism $\chi\colon G_9 \to \mathbb{R}$. Since we know that all character $\chi \in \hat{G}$ satisfy
  \[|\chi(a)| =1, \text{ for all } a \in G_9,\]
  we can deduce that a real character $\chi$ must satisfy $\chi(G_9) \subset \left\lbrace \pm 1 \right\rbrace$. Since
  we want to find a non-principal character, the only choices is that $\chi(G_9) = \left\lbrace \pm 1\right\rbrace$. As shown
  above, $\chi$ is defined solely by its value at $2$, so we must choose $\chi(2) =-1$. This implies that there is only
  one non-principal, real Dirichlet character over $G_9$.
\end{proof}
\end{document}