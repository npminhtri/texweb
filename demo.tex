% This is a simple sample document.  For more complicated documents take a look in the exercise tab. Note that everything that comes after a % symbol is treated as comment and ignored when the code is compiled.

\documentclass{article} % \documentclass{} is the first command in any LaTeX code.  It is used to define what kind of document you are creating such as an article or a book, and begins the document preamble

\usepackage{amsmath} % \usepackage is a command that allows you to add functionality to your LaTeX code

\title{Normal Families and the Riemann Mapping theorem} % Sets article title
\author{Tri Nguyen} % Sets authors name
\date{\today} % Sets date for date compiled
\usepackage[papersize={216mm,330mm},tmargin=20mm,bmargin=20mm,lmargin=20mm,rmargin=20mm]{geometry}
\usepackage[english]{babel}
\usepackage[utf8]{inputenc}
\usepackage{amsmath,amssymb,mathabx,amsthm}%\for eqref
\usepackage{lscape}
\usepackage{graphicx}
\usepackage[colorinlistoftodos]{todonotes}
\usepackage{fancyhdr}
\pagestyle{fancy}
\fancyhf{}
\newtheorem{definition}{Definition}
\newtheorem{lemma}{Lemma}

\setlength\parindent{0pt}



% The preamble ends with the command \begin{document}
\begin{document} % All begin commands must be paired with an end command somewhere
\maketitle % creates title using information in preamble (title, author, date)

This note is used to list every theorems in chapter 9 of the book \textbf{Complex made simple}.
\section{Quasi-metrics} % creates a section
\begin{definition}
A function $d \colon X \times X \to [0,\infty]$ satisfying the condition
\begin{itemize}
    \item $d(x,x) = 0 (x \in X)$
    \item $d(x,y) = d(y,x) (x,y \in X)$
    \item $d(x,z) \le d(x,y) + d(y,z) (x,y,z)$
\end{itemize} 
then it is called a quasi-metric on space $X$. We can see that $d$ is almost the same as a metric except that $d(x,y)$ can be zero for distinct $x,y$.
\end{definition}

One can construct a metric $\overline{d}$ from quasi-metric $d$, noting that $d$ is an equivalence relation on $X$.

Now we introduction the notion of \textit{concave function}: The function $\psi \colon I \to \mathbb{R}$ is said to be \textit{concave} if 
\begin{align*}
    \psi(tx + (1-t)y) \le t\psi(x) + (1-t)\psi(y),
\end{align*}
for all $x,y \in X$ and $ 0 \le t \le 1$. It can be inferred from the definition that $\psi$ is concave iff $-\psi$ is convex. Now we have the following lemma
\begin{lemma}
    Suppose that $\psi \colon I \to \mathbb{R}$ is concave and $a_1,a_2,b_1,b_2 \in I$ such that $a_1 <b_1, a_2 <b_2, a_2 \ge a_1, b_2 \ge b_1$. Then 
    \[\dfrac{\psi(b_2)- \psi(a_2)}{b_2-a_2} \le \dfrac{\psi(b_1)- \psi(a_1)}{b_1-a_1} \]
\end{lemma}
Geometrically speaking, the slope of the segment joining two points on the graph of $\psi$ decreases as the points moves to the right.
\end{document} % This is the end of the document