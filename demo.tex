% This is a simple sample document.  For more complicated documents take a look in the exercise tab. Note that everything that comes after a % symbol is treated as comment and ignored when the code is compiled.

\documentclass[12pt]{article} % \documentclass{} is the first command in any LaTeX code.  It is used to define what kind of document you are creating such as an article or a book, and begins the document preamble

\usepackage{amsmath} % \usepackage is a command that allows you to add functionality to your LaTeX code

\title{SOME (VERY EASY) EXERCISES IN MATH 681} % Sets article title
\author{Tri Nguyen} % Sets authors name
\date{\today} % Sets date for date compiled
\usepackage[papersize={216mm,330mm},tmargin=20mm,bmargin=20mm,lmargin=20mm,rmargin=20mm]{geometry}
\usepackage[english]{babel}
\usepackage[utf8]{inputenc}
\usepackage{amsmath,amssymb,mathabx,amsthm}%\for eqref
\usepackage{lscape}
\usepackage{graphicx}
\usepackage[colorinlistoftodos]{todonotes}
\usepackage{fancyhdr}
\usepackage{tikz,lipsum,lmodern}
\usepackage[most]{tcolorbox}
\pagestyle{fancy}
\fancyhf{}
\newtheorem{definition}{Definition}
\newtheorem{lemma}{Lemma}
\DeclareMathOperator{\sln}{\mathfrak{sl}_n(\mathbb{R})}
\DeclareMathOperator{\Tr}{\textbf{Tr}}
\DeclareMathOperator{\ad}{ad}
\setlength\parindent{0pt}



% The preamble ends with the command \begin{document}
\begin{document} % All begin commands must be paired with an end command somewhere
\maketitle % creates title using information in preamble (title, author, date)
\begin{tcolorbox}[colback=blue!5!white,colframe=blue!75!black,title=Problem 1]
  Let $\mathfrak{k} = \left\lbrace X \in \sln : X^t = -X \right\rbrace $ and $ \mathfrak{p } = \left\lbrace X \in \sln: X=X^t\right\rbrace$.
  Verify the following inclusion relation:
  \begin{itemize}
    \item $[\mathfrak{k},\mathfrak{k}] \subset \mathfrak{k}$.
    \item  $[\mathfrak{k},\mathfrak{p}] \subset \mathfrak{p}$.
    \item  $[\mathfrak{p},\mathfrak{p}] \subset \mathfrak{k}$.
  \end{itemize}
\end{tcolorbox}
\begin{proof}
  \hfill

  Let $X,Y \in \mathfrak{k}$ be arbitrary. Then we have
  \[ [X,Y]^t= \left(XY-YX\right)^t=Y^tX^t-X^tY^t=YX-XY = -[X,Y]\]
  Thus by definition, $[X,Y] \in \mathfrak{k}$.

  Let $U,V \in \mathfrak{p}$ be arbitrary. Similarly, we have
  \[[U,V]^t = (UV - VU)^t= V^tU^t- U^tV^t = VU - UV = -[U,V]\]
  which also implies that $[\mathfrak{p},\mathfrak{p}] \subset \mathfrak{k}$.

  Lastly, we have
  \[[X,U]^t = (XU-UX)^t = U^tX^t-X^tU^t = -UX+XU=[X,U]\]
  which means  $[\mathfrak{k},\mathfrak{p}] \subset \mathfrak{p}$.
\end{proof}
\begin{tcolorbox}[colback=blue!5!white,colframe=blue!75!black,title=Problem 2]
  This is a verification exercise about choosing the \textbf{right} basis. Let denote $\kappa(.,.)$
  the Killing form. Follows the same notation in previous exercise, let
  \begin{itemize}
    \item $Y_1,\ldots, Y_N$ be a basis of $\mathfrak{p}$ and this basis is indexed by $i,j,k,l$.
    \item $Y_{N+1},\ldots, Y_n$ be a basis of $\mathfrak{k}$ and this basis is indexed by $a,b,c,d$.
  \end{itemize}
  \textit{Fact: } $\kappa$ is positive definite on $\mathfrak{p}$ and negative definite on $\mathfrak{k}$. Furthermore
  we chose the above basis such that
  \[\begin{cases}
      \kappa(Y_i,Y_j) = \delta_{ij}, \\
      \kappa (Y_a,Y_b) = -\delta_{ab}
    \end{cases}\]
  By the above exercise, we have that
  \[\begin{cases}
      [Y_i,Y_j] = \sum_{a} c^a_{ij}Y_a \\
      [Y_a,Y_i] = \sum_{j} c_{ai}^j Y_j
    \end{cases}\]
  Show that $c^a_{ij} = c^i_{aj}$.
\end{tcolorbox}
\begin{proof}
  \hfill

  It is well-known that the Killing form is invariant in the following sense
  \[\kappa([a,b],c) = \kappa(a,[b,c])\]
  Using this invariant property, we will compute $\kappa(Y_a,[Y_i,Y_j])$ in two ways
  \begin{enumerate}
    \item we have:
          \[\kappa\left(Y_a,[Y_i,Y_j]\right) = \kappa\left(Y_a,\sum_b c^b_{ij}Y_b\right) = -c^a_{ij}\]
    \item On the other hand we also have
          \[\kappa(Y_a,[Y_i,Y_j]) = -\kappa([Y_a,Y_j],Y_i) = -\kappa \left(\sum_k c_{aj}^k Y_k,Y_i\right)=-c^i_{aj}\]
  \end{enumerate}
  Comparing both results yield the desired equality.
\end{proof}
\begin{tcolorbox}[colback=blue!5!white,colframe=blue!75!black,title=Problem 3]
  Verify that
  \begin{enumerate}
    \item $\sln  = \mathfrak{k}\oplus\mathfrak{p}$.
    \item $\kappa$ is positive definite on $\mathfrak{p}$ and negative definite on $\mathfrak{k}$.
  \end{enumerate}
\end{tcolorbox}
\begin{proof}
  It is easy to check that for any traceless matrix $X = [c_{ij}]$, we can easily
  find two traceless matrices $A = [a_{ij}]$ and $B = [b_{ij}]$ such that
  \begin{itemize}
    \item $A$ is symmetric and $B$ is skew-symmetric.
    \item $X= A+B$.
  \end{itemize}
  Indeed, in order to determine the matrices $A,B$, it boils down to solve the the system
  \begin{align*}
    \begin{cases}
      a_{ij}+b_{ij}=c_{ij} \\
      a_{ij}-b_{ij} = c{ji}
    \end{cases},
  \end{align*}
  for distinct indices $i,j$. This system is always solvable. It is obvious that the zero matrix
  is the only matrix that is both symmetric and skew-symmetric. Thus
  \[\sln = \mathfrak{k}\oplus\mathfrak{p}\]
  The positive/negative definite part follows directly from the definitions of
  the Killing form and the obvervation that
  \[\kappa(X,X) = \Tr(\ad X \circ \ad X) = \Tr(\ad (X)^t\ad X) = \sum_{i,j}c^2_{ij} \ge 0,\]
  where $X \in \mathfrak{p}$. similarly for $X \in \mathfrak{k}$
  \[\kappa(X,X) = \Tr(\ad X \circ \ad X) = -\Tr(\ad (X)^t\ad X) = -\left(\sum_{i,j}c^2_{ij}\right) \le 0.\]
  Hence we are done.
\end{proof}
\begin{tcolorbox}[colback=blue!5!white,colframe=blue!75!black,title=Problem 4]
  Given a parabolic subgroup $P \in \text{SL}_n(\mathbb{R})$ and assume that
  \[P = M_P \times A_P \times U_P.\]
  Recall that we have the Iwasawa decomposition
  \[ \text{SL}_n(\mathbb{R}) \cong K \times A \times N\]
  Prove that $K \cap P = K \cap M_P$.
\end{tcolorbox}
\begin{proof}
  It is obvious that $K \cap P \supset K \cap M_P$, so we just need to prove the other inclusion. Let $X \in K \cap P$
  be arbitrary, then it has the form
  \[X = \begin{bmatrix}
      A_{n_1} & \star   & \ldots & \star   \\
      0       & A_{n_2} & \ldots & \star   \\
      0       & 0       & \ldots & \star   \\
      \vdots  & \vdots  & \ddots & \vdots  \\
      0       & 0       & \ldots & A_{n_k} \\
    \end{bmatrix},\]
  and satisfies $X^t = X^{-1}$. By comparing the entries, all the entries off the diagonal block matrix $A_{n_i}$
  must be zero. So we have
  \[X = \begin{bmatrix}
      A_{n_1} & 0       & \ldots & 0       \\
      0       & A_{n_2} & \ldots & 0       \\
      0       & 0       & \ldots & 0       \\
      \vdots  & \vdots  & \ddots & \vdots  \\
      0       & 0       & \ldots & A_{n_k} \\
    \end{bmatrix}\]
  But the block matrices themselves also satisfy $A_{n_i}^t = A_{n_i}^{-1}$, thus implies
  $\det(A_{n_i}) = \pm 1$, which also means $X \in M_P.$
\end{proof}
\begin{tcolorbox}[colback=blue!5!white,colframe=blue!75!black,title=Problem 5]
  Show that the map $d_{q+1} \circ d_q$ in section 1.1 of the note is identically zero.
\end{tcolorbox}
\begin{proof}
  Recall that the map $d_q$ is given by
  \begin{align*}
    d_q \colon C^q & \to C^{q+1}   \\
    f              & \mapsto d_q f
  \end{align*}
  where $(d_q f)(x_0,\ldots,x_{q+1}) = \sum (-1)^i f(x_0,\ldots \hat{x_i},\ldots,x_{q+1})$. Here the notation
  $\hat{\cdot}$ means we omit the corresponding variable. Now we have
  \begin{align*}(d_{q+1} \circ d_q)(f)(x_0,\ldots,x_{q+2}) & = \sum (-1)^i(d_q f)((x_0,\ldots \hat{x_i},\ldots,x_{q+2}))                                                                 \\
                                                         & = \sum_{i=0}^{q+2} (-1)^i \left[\sum_{j\ne i} (-1)^{a_{i,j}} f(x_0,\ldots \hat{x_i},\ldots,\hat{x_j},\ldots,x_{q+1})\right] \\
                                                         & =\sum (-1)^{i+a_{i,j}}f(x_0,\ldots \hat{x_i},\ldots,\hat{x_j},\ldots,x_{q+1})
  \end{align*}
  Now we fixes some nonnegative integers $u,v$ and consider two cases
  \begin{itemize}
    \item $i =u> v = j: $ In this case $a_{i,j} = u$, thus the coefficients of $f(x_0,\ldots \hat{x_u},\ldots,\hat{x_v},\ldots,x_{q+1})$ is $(-1)^{u+v}$.
    \item $i = v< u  =j: $ In this cases $a_{i,j} = u-1$,  thus the coefficients of $f(x_0,\ldots \hat{x_u},\ldots,\hat{x_v},\ldots,x_{q+1})$ is $(-1)^{u+v-1}$.
  \end{itemize}
  This implies that for each pair of distinct $(u,v)$, the function $f$ is evaluated at the same points with different signs, hence canceled out.
  In particular, the sum will vanish. Therefore $d_{q+1} \circ d_q \equiv 0$.
\end{proof}
\end{document} % This is the end of the document