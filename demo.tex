\documentclass[11pt,letterpaper]{article}
\usepackage[lmargin=0.75in,rmargin=0.75in,tmargin=0.75in,bmargin=0.5in]{geometry}

% -------------------
% Packages
% -------------------
\usepackage{
	amsmath,			% Math Environments
	amssymb,			% Extended Symbols
	enumerate,		    % Enumerate Environments
	graphicx,			% Include Images
	lastpage,			% Reference Lastpage
	multicol,			% Use Multi-columns
	multirow,			% Use Multi-rows
  bbm,           % Mathbb for numbers
  amsthm
}
\usepackage[framemethod=TikZ]{mdframed}
\usepackage{tikz, tabularx}
\usepackage{graphics}
\newcolumntype{W}{>{\centering\arraybackslash}X}%Para agilizar las columnas.
% -------------------
% Font
% -------------------
\usepackage[T1]{fontenc}
\usepackage{charter}


% -------------------
% Commands
% -------------------
\newcommand{\homework}[2]{\noindent\textbf{Name: }\makebox[3in]{Fill my name here} \hfill \textbf{} \\  \textbf{: #2} \hfill \textbf{}\\}

\newcommand{\prob}{\noindent\textbf{Problem. }}
\newcounter{problem}
\newcommand{\problem}{
	\stepcounter{problem}%
	\noindent \textbf{Problem \theproblem. }%
}
\newcommand{\pointproblem}[1]{
	\stepcounter{problem}%
	\noindent \textbf{Problem \theproblem.} (#1 points)\,%
}
\newcommand{\pspace}{\par\vspace{\baselineskip}}
\newcommand{\ds}{\displaystyle}


% -------------------
% Theorem Environment
% -------------------
\mdfdefinestyle{theoremstyle}{%
	frametitlerule=true,
	roundcorner=5pt,
	linecolor=black,
	outerlinewidth=0.5pt,
	middlelinewidth=0.5pt
}
\mdtheorem[style=theoremstyle]{exercise}{\textbf{Problem}}


% -------------------
% Header & Footer
% -------------------
\usepackage{fancyhdr}

\fancypagestyle{pages}{
	%Headers
	\fancyhead[L]{}
	\fancyhead[C]{}
	\fancyhead[R]{}
\renewcommand{\headrulewidth}{0pt}
	%Footers
	\fancyfoot[L]{}
	\fancyfoot[C]{}
	\fancyfoot[R]{page \thepage \, of \pageref{LastPage}}
\renewcommand{\footrulewidth}{0.0pt}
}
\headheight=0pt
\footskip=14pt

\pagestyle{pages}

\DeclareMathOperator{\1}{\mathbbm{1}}

% -------------------
% Content
% -------------------
\begin{document}


% Question 1
\begin{exercise}
  Let $\chi_0$ be the principal character mod $3$, that is
  \[\chi_0 (n) = \begin{cases}
      1 , \quad \gcd(x,3)=1 \\
      0, \quad \text{otherwise}
    \end{cases}\]
  Define
  \[\chi(n) = \begin{cases}
      1, \quad n \equiv 1 \pmod 3  \\
      -1, \quad n \equiv 2 \pmod 3 \\
      0, \quad \text{otherwise}
    \end{cases}\]
  Prove that $\chi_0$ and $\chi$ are completely multiplicative functions. Verify the identity
  \[ \mathbbm{1} = \dfrac{1}{2}(\chi_0 +\chi)\]
\end{exercise}
\begin{proof}
  \hfill

  Let $m,n \in \mathbb{Z}$ be arbitrary integers. We consider the following cases:
  \begin{enumerate}
    \item At least one of $m,n$ is divisible by $3$. WLOG, we can assume $3 \mid m$. Then it is clear that $3 \mid mn$.
          By definition, we must have $\chi_0(m) = \chi(m) =0$ and $\chi_0(mn) =\chi(mn)=0$. Thus we have
          \[\chi(m)\chi(n) = 0\cdot \chi(n) = 0 = \chi(mn) \quad \text{ and }\chi_0(m)\chi_0(n) = 0\cdot \chi(n) = 0 = \chi(mn).\]
    \item Both $m,n$ are coprime to $3$. There will be then two subcases
          \begin{itemize}
            \item $m \equiv n \pmod 3$. Then clearly we have $\chi_0(m) = \chi_0(n)=1$ and $\chi(m)=\chi(n)$. Moreover
                  $mn \equiv 1 \pmod 3$, thus $\chi(mn)=\chi_0(mn)=1$. In particular, we have
                  \[\chi(m)\chi(n) = 1\cdot 1 = 1 = \chi(mn) \quad \text{ and }\chi_0(m)\chi_0(n) = 1\cdot 1 = 1 = \chi(mn).\]
            \item $m \not\equiv n \pmod 3$. We can further assume that $ m \equiv 1 \pmod 3$ and $n \equiv 2 \pmod 3$. clearly
                  \[\chi_0(m)\chi_0(n) = 1\cdot 1 = 1 = \chi_0(mn), \]
                  as $\gcd(mn,3)=1$.

                  On the other hand, $mn \equiv 1 \cdot (-1)\equiv 2 \pmod 3$, thus
                  \[\chi(m)\chi(n) = 1\cdot (-1) = -1 = \chi(mn) \]
          \end{itemize}
          In conclustion, $\chi$ and $\chi_0$ are completely multiplicative. To verify the given identity, we consider
          the following cases
          \begin{enumerate}
            \item $m \not\equiv 1 \pmod 3$: Then $\mathbbm{1}(m) = 0$. On the other hand, we have
                  \[\dfrac{1}{2}\left(\chi(m)+\chi_0(m)\right)= \begin{cases}
                      0+0, \quad m \equiv 0 \pmod 3 \\
                      -1+1, \quad m \equiv 2 \pmod 3
                    \end{cases} = 0\]
            \item $m \equiv 1 \pmod 3$: Then $\mathbbm{1}(m) =1 =1/2(\chi(m)+\chi_0(m))$
          \end{enumerate}
  \end{enumerate}
  Thus we are done.
\end{proof}


% Question 2
\begin{exercise}
  For $s>1$, define the $L$ function
  \[L(s,\chi):= \sum_{n=1}^\infty \dfrac{\chi(n)}{n^s} = \sum_{k=0}^\infty \dfrac{1}{(3k+1)^s}-\sum_{k=0}^\infty \dfrac{1}{(3k+2)^s} = 1- \dfrac{1}{2^x}+\dfrac{1}{4^x}-\dfrac{1}{5^x}+\ldots \]
  Show that
  \[\dfrac{1}{2} \le L(s,\chi) \le 1\]
  for $s>1$, and hence show that
  \[\dfrac{1}{2} \log L(s,\chi_0) + \dfrac{1}{2}\log L(s,\chi) \to \infty \quad \text{ as } s \to 1^+,\]
  where
  \[L(s,\chi_0):= \sum_{n=1}^\infty \dfrac{\chi_0(n)}{n^s}.\]
\end{exercise}
\begin{proof}
  It can be seen easily that
  \[\sum_{k=1}^\infty \left|\dfrac{\chi(k)}{k^s}\right|< \sum_{k=1}^\infty \dfrac{1}{k^s}<\infty\]
  for $s>1$. Thus the series $L(s,\chi)$ converges absolutely. In particular, we can rearrange the term of the series
  without changing it values. Note that
  \[L(s,\chi) = 1- \dfrac{1}{2^x}+\dfrac{1}{4^x}-\dfrac{1}{5^x}+\ldots = 1-\sum_{k=0}\left(\dfrac{1}{(3k+2)^s}-\dfrac{1}{(3k+4)^s}\right) \le 1,\]
  and
  \[L(s,\chi) = 1- \dfrac{1}{2^x}+\dfrac{1}{4^x}-\dfrac{1}{5^x}+\ldots = 1-\dfrac{1}{2^s}+\sum_{k=1}\left(\dfrac{1}{(3k+1)^s}-\dfrac{1}{(3k+2)^s}\right)>1-\dfrac{1}{2^s}\ge \dfrac{1}{2},\]
  where $s>1$. In conclustion, we have
  \[1/2 \le L(s,\chi) \le 1 \]
  Using formulae $(14)$ for the principal character in the note, we get
  \[ L(s,\chi_0) \ge \dfrac{\Phi(3)}{3^x}\sum_{k=1}^\infty \dfrac{1}{k^s} \longrightarrow \infty,\]
  as $s \to 1^+$. Moreover, $L(s,\chi)$ is bounded as $s \to 1^+$ as showed above, we can conclude that
  \[L(s,\chi_0) \cdot L(s,\chi) \to \infty \quad \text{ as } s \to 1^+\]
  which clearly show that
  \[\dfrac{1}{2} \log L(s,\chi_0) + \dfrac{1}{2}\log L(s,\chi) \to \infty \quad \text{ as } s \to 1^+,\]
  as desired.
\end{proof}


%Question 3
\newpage
\begin{exercise}
  Show that
  \[\dfrac{1}{2}\log L(s,\chi_0)+ \dfrac{1}{2}\log L(s,\chi) = \sum_{p}\sum_{k=1}^\infty \dfrac{\mathbbm{1}(p^k)}{kp^{ks}}\]
  for $s>1$, where the sum extends over all primes $p$.
\end{exercise}
\begin{proof}
  Since $\chi$ and $\chi_0$ are real-valued, we can safely taking the logarithm without the need to taking care of
  complex logarithm. Using the Taylor's expansion for the $\log$, we get
  \[ \log L(s,\chi) = - \sum_{p} \log(1-\chi(p)p^{-s}) = \sum_{p}\sum_{k=1}^\infty \dfrac{\chi(p^k)p^{-ks}}{k}\]
  and
  \[ \log L(s,\chi_0) = - \sum_{p} \log(1-\chi_0(p)p^{-s}) = \sum_{p}\sum_{k=1}^\infty \dfrac{\chi_0(p^k)p^{-ks}}{k}\]
  As shown above, we have that $\mathbbm{1}= \frac{\chi+\chi_0}{2}$, thus
  \[\log L(s,\chi_0)+ \log L(s,\chi) =\sum_{p}\sum_{k=1}^\infty \dfrac{[\chi(p^k)+\chi_0(p^k)]p^{-ks}}{k}= \sum_{p}\sum_{k=1}^\infty 2\cdot \dfrac{\mathbbm{1}(p^k)p^{-ks}}{k},\]
  which implies
  \[\dfrac{1}{2}\log L(s,\chi_0)+ \dfrac{1}{2}\log L(s,\chi) = \sum_{p}\sum_{k=1}^\infty \dfrac{\mathbbm{1}(p^k)}{kp^{ks}},\]
  for $s>1$.
\end{proof}

%Question 4
\begin{exercise}
  Put everything together to conclude that the series $$\displaystyle\sum_{p \equiv 1 \pmod 3} \dfrac{1}{p}$$ diverges.
\end{exercise}
\begin{proof}
  It can be seen easily that the double sum in problem 3 is the Taylor's expansion of the function
  $\log(L(s,\mathbbm{1}))$ with $s>1$. By problem 2 we have
  \[ \log(L(s,\mathbbm{1})) \longrightarrow \infty \quad \text{ as } \quad x \to 1^+\]
  Taking the exponent of $\log(L(s,\mathbbm{1}))$ and use the definition, we get
  \[ \log L(s,\mathbbm{1}) = \sum_{p=3t+1} \dfrac{1}{p^s} + \underbrace{\sum_{p=3t+1}\sum_{k=2}^\infty  \dfrac{1}{kp^{ks}}}_{E(x)}\]
  If we can show that the latter double sum is bounded, then we are done. This sum can be estimated in the exactly the same way
  in the note as follows
  \[\sum_{k=2}^\infty  \dfrac{1}{kp^{ks}} < \sum_{k=2}^\infty \left(\dfrac{1}{p^s}\right)^s= \dfrac{1}{p^s(p-1)} < \dfrac{1}{p-1}-\dfrac{1}{p} \]
  Summing over all primes $p \equiv 1 \pmod 3$, we get
  \[E(x) = \sum_{p=3t+1}\sum_{k=2}^\infty  \dfrac{1}{kp^{ks}} < \sum_{p=3t+1} \left(\dfrac{1}{p-1}-\dfrac{1}{p}\right) < \sum_{n \in \mathbb{Z}_{\ge 1}}\left(\dfrac{1}{n}-\dfrac{1}{n+1}\right)=1\]
  In particular, we have
  \[ \log(L(s,\mathbbm{1})) = \sum_{p=3t+1} \dfrac{1}{p^s} + E(x) \longrightarrow \infty \quad \text{ as } \quad x \to 1^+\]
  As $E(x) \in (0,1)$ letting $s \to 1^+$ yields
  \[\sum_{p \equiv 1 \pmod 3} \dfrac{1}{p} =\infty,\]
  which means the series $\displaystyle\sum_{p \equiv 1 \pmod 3} \dfrac{1}{p}$ diverges, as desired.
\end{proof}
\end{document}