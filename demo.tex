\documentclass[11pt,letterpaper]{article}
\usepackage[lmargin=0.75in,rmargin=0.75in,tmargin=0.75in,bmargin=0.5in]{geometry}

% -------------------
% Packages
% -------------------
\usepackage{
	amsmath,			% Math Environments
	amssymb,			% Extended Symbols
	enumerate,		    % Enumerate Environments
	graphicx,			% Include Images
	lastpage,			% Reference Lastpage
	multicol,			% Use Multi-columns
	multirow,			% Use Multi-rows
  bbm,           % Mathbb for numbers
  amsthm
}
\usepackage[framemethod=TikZ]{mdframed}
\usepackage{tikz, tabularx}
\usepackage{graphics}
\usepackage{hyperref}
\newcolumntype{W}{>{\centering\arraybackslash}X}%Para agilizar las columnas.
% -------------------
% Font
% -------------------
\usepackage[T1]{fontenc}
\usepackage{charter}


% -------------------
% Commands
% -------------------
\newcommand{\homework}[2]{\noindent\textbf{Name: }{} \hfill \textbf{} \\  \textbf{Due date: #2} \hfill \textbf{}\\}

\newcommand{\prob}{\noindent\textbf{Problem. }}
\newcounter{problem}
\newcommand{\problem}{
	\stepcounter{problem}%
	\noindent \textbf{Problem \theproblem. }%
}
\newcommand{\pointproblem}[1]{
	\stepcounter{problem}%
	\noindent \textbf{Problem \theproblem.} (#1 points)\,%
}
\newcommand{\pspace}{\par\vspace{\baselineskip}}
\newcommand{\ds}{\displaystyle}


% -------------------
% Theorem Environment
% -------------------
\mdfdefinestyle{theoremstyle}{%
	frametitlerule=true,
	roundcorner=5pt,
	linecolor=black,
	outerlinewidth=0.5pt,
	middlelinewidth=0.5pt
}
\mdtheorem[style=theoremstyle]{exercise}{\textbf{Problem}}


% -------------------
% Header & Footer
% -------------------
\usepackage{fancyhdr}

\fancypagestyle{pages}{
	%Headers
	\fancyhead[L]{}
	\fancyhead[C]{}
	\fancyhead[R]{}
\renewcommand{\headrulewidth}{0pt}
	%Footers
	\fancyfoot[L]{}
	\fancyfoot[C]{}
	\fancyfoot[R]{page \thepage \, of \pageref{LastPage}}
\renewcommand{\footrulewidth}{0.0pt}
}
\headheight=0pt
\footskip=14pt

\pagestyle{pages}

\DeclareMathOperator{\1}{\mathbbm{1}}
\DeclareMathOperator{\Log}{Log}

% -------------------
% Content
% -------------------
\begin{document}
\homework{\#}{02/14}


% Question 1
\begin{exercise}
  Let $q \ge 2$ be an integer and let $\chi$ be a Dirichlet character mod $q$. Consider the $L$
  \[L(z;\chi) := \sum_{i=1}^\infty \dfrac{\chi(n)}{n^z}\]
  which converges absolutely for $\Re(z)=:x >1$ since
  \[ \sum_{n\ge 1 }|\chi(n)n^{-z}| \ge \sum_{n\ge 1} n^{-x} <\infty,\]
  by the $p-$test from FPM. Show that $L(z;\chi)$ defines a holomorphic function on $X = \left\lbrace z \in \mathbb{C}: \Re(z)>1 \right\rbrace$
  and $L(z;\chi) \ne 0$ for all $z \in X$. Furthermore, show that
  \[\dfrac{L'(z;\chi)}{L(z;\chi)} = - \sum_p \dfrac{\chi(p)\log(p)}{p^z-\chi(p)};\]
  for $z \in X$.
\end{exercise}
\begin{proof}
  \hfill
  Using Weierstrass M-test as in the statement in the problem, we have proved that the sequence
  \[ L_N(z;\chi) = \sum_{n=1}^N \dfrac{\chi(n)}{n^z}\]
  converges uniformly to the function $L(z;\chi)$. In particular, $L(z;\chi)$ is continuous
  over the domain $X$. Clearly the sequence of functions $L_N$ are holomorphic on $X$. Thus we have
  \[\int_{\gamma} L_N(z; \chi)dz = 0,\]
  for piecewise smooth closed curve inside $X$, by Cauchy's theorem. Furthermore, we have
  \[\int_{\partial \Delta} L(z;\chi) = \lim_{N \to \infty} \int_{\partial \Delta} L_N(z,\chi) =0\]
  By Morera's theorem, we can conclude that $L(z,\chi)$ is holomorphic over the domain $X$.

  Clearly the domain $X$ is simply connected. Using proposition 2.3 in the note, we can rewrite the $L-$ function as
  \[L(z;\chi ) = \sum_{i=1}^\infty \dfrac{\chi(n)}{n^z} = \prod_p \dfrac{1}{1-\chi(p)p^{-z}}\]
  Note that over $X= \left\lbrace z \in \mathbb{C}: \Re(z)>1 \right\rbrace$, we have
  \[\dfrac{1}{1-\chi(p)p^{-z}} = 1+ \chi(p)p^{-z}+ (\chi(p)p^{-z})^2+\ldots = 1+ \sum_{k\ge 1} \chi(p^k)p^{-kz}\]
  where
  \[\sum_{k\ge 1} |\chi(p^k)p^{-kz}| \le \sum_{n \ge 1} n^{-\Re z}<\infty\]
  Let denote $u_p(z)= \sum_{k\ge 1} \chi(p^k)p^{-kz}$, then
  \[L(z;\chi) = \prod_p \left(1+u_p(z)\right)\]
  Then $L(z;\chi)=0$ if and only if $u_p(z) =-1$ for some $p$. But this is impossible as
  \[|u_p(z)| \le \sum_{k\ge 1} |\chi(p^k)p^{-kz}| = \sum_{k\ge 1} p^{-k\Re z} = \dfrac{p^{-\Re (z)}}{1-p^{-\Re z}} =\dfrac{1}{p^{\Re z}-1} <1\]
  Thus $L(z;\chi)$ never vanishes on the domain $X$.

  Lastly, we will prove the identity for $\frac{L'(z;\chi)}{L(z;\chi)}$. From the discussion of the note,
  there exists a branch of the logarithm $\log L(z,\chi)$ on $X$ and satisfies
  \[\log L(z;\chi) = \sum_p \Log(1-\chi(p)p^{-z})\]
  We can then define the partial sum $S_m = \sum_{p\le m} \Log(1-\chi(p)p^{-z})$. Clearly this partial sum converges uniformly to
  $\log L(z;\chi)$ as $m \to \infty$ and are holomorphic. Thus
  \[S'_m(z) = -\sum_{p \le m} \dfrac{\chi(p)\log(p)}{p^z-\chi(p)} \to \left[\log(L(z;\chi))\right]' = \dfrac{L'(z;\chi)}{L(z;\chi)}\]
  In particular, for $z \in X$, we have
  \[\dfrac{L'(z;\chi)}{L(z;\chi)}= -\sum_{p } \dfrac{\chi(p)\log(p)}{p^z-\chi(p)},\]
  where the sum extends over all prime number $p$, as desired.
\end{proof}


% Question 2
\begin{exercise}\label{ex2}
  Let $G$ be a finite abelian group. Show that $\#\hat{G} \le \#G$.
\end{exercise}
\begin{proof}
  First we will prove that the set $S$ of maps $f\colon G \to \mathbb{C}$ has a vector space structure.
  The sum and the scalar multiplication of maps are defined as follows
  \begin{itemize}
    \item $(\chi_1 + \chi_2)(a) := \chi_1(a)+ \chi_2(a)$ for all $a \in G$.
    \item $(c\chi)(a):= c\cdot\chi(a)$ for any $\chi \in S$ and $c \in \mathbb{C}$.
  \end{itemize}
  But we can check all the axioms for a set being a vector space pointwisely, with the zero
  beging the zero map. We can verify as follows
  \begin{enumerate}
    \item Additive closure: this is clear from the way we defined the sum of two functions in $S$.
    \item Commutative: For any $a \in G$, we have $\chi_1(a)+\chi_2(a) = \chi_1(a)+\chi_2(a)$. Since this holds for any $a \in G$, we have $\chi_1 + \chi_2 = \chi_2+\chi_1$.
    \item Associative: Similarly, for all $a \in G$
          \[(\chi_1(a) + \chi_2(a))+\chi_3(a) = \chi_1(a)+ (\chi_2(a)+\chi_3(a))\]
          In general
          \[(\chi_1 + \chi_2)+\chi_3 = \chi_1+ (\chi_2+\chi_3)\]
    \item The zero map is just the function $0 \colon G \to \mathbb{C}$ that sends everything to $0$.
    \item For a map $f \in S$, we can define its additive inverse to be $(-f)\colon G \to \mathbb{C}$ such that $(-f)(a) := -f(a)$. Clearly
          \[(f+ (-f))(a) = f(a) + (-f(a)) = 0, \forall a \in G\]
          Thus $f+(-f)$ is the zero map.
    \item The other axioms on the scalar multiplication can be checked pointwisely in the  exact same way.

  \end{enumerate}

  So $S$ have a a structure of vector space.

  Let's compute the dimension of $S$ as
  as $\mathbb{C}-$ vector space. Since $G$ is finite, we can assume that, as a set
  \[G = \left\lbrace a_1,\ldots,a_n \right\rbrace\]
  Then we can define the map $f_i$ to be the "dual" of $a_i$ in the following sense: $f_i(a_j) =\delta_{ij}$, i.e.
  $f_i(a_i)=1$ and vanishes at $a_j \ne a_i$. We claim that $\mathfrak{B}= \left\lbrace f_i \right\rbrace$ forms a basis of $S$.
  Indeed, let $f\colon G \to \mathbb{C}$ be arbitrary. Since $G$ is finite, $f$ is determined by its value at each $a_i \in G$.
  Assume that $c_i = f(a_i)$. Then for any $1 \le i \le n$, we have
  \[f(a_i) = c_i = \sum_{i=1}^n c_i f_i(a_i)\]
  In particular, we can rewrite the following identity as
  \[f = c_1f_1+c_2f_2+\ldots+c_nf_n,\]
  which implies $\mathfrak{B}$ spans $S$. On the other hand, if
  \[0 = c_1f_1+c_2f_2+\ldots+c_nf_n\]
  Applying both sides to $a_i$ yields
  \[0 = c_if_i(a_i) = c_i\]
  which means $\mathfrak{B}$ is a set of linearly independent vectors. Thus $\dim S = \#\mathfrak{B} = n$.
  In the note, we proved that the character $\chi_1,\chi_2,\ldots,\chi_m$ are mutually orthogonal. In particular, they are
  a subset of $S$ comprising of linearly independent vectors. Thus
  \[m = \#\hat{G} \le n = \# G. \]
  \textbf{Remark:} In fact, if we assume the theorem about the structure of finite abelian group, we can prove that
  the equality always happens, namely $ \#\hat{G} =\# G$
\end{proof}
\newpage






%Question 3
\begin{exercise}
  \label{ex3}
  Consider the finite abelian group $G_9$ consisting of the invertible elements of $\mathbb{Z}/9\mathbb{Z}$. Find all the characteres of $G_9$.
  Make sure you prove that the characters of $G_9$ you find are all distinct and there are no others.
\end{exercise}
\begin{proof}
  Using the Remark in the previous exercise, we expect $6$ characters in total. It can be checked easily that $2$ generates $G_9$, so it is a
  cyclic group of order $6$. Indeed, we have
  \begin{align*}
    2^1 \equiv 2 \pmod 9 \\
    2^2 \equiv 4 \pmod 9 \\
    2^3 \equiv 8 \pmod 9 \\
    2^4 \equiv 7 \pmod 9 \\
    2^5 \equiv 5 \pmod 9 \\
    2^6 \equiv 1 \pmod 9
  \end{align*}
  So the character is defined solely by determining where it sends $[2]$ in $\mathbb{C}$. Let $\chi \in \hat{G_9}$ be
  any character, we then have
  \[\chi(2)^6=\chi(2^6) = \chi(1) = 1 \in \mathbb{C}\]
  thus implies $\chi(2)$ must be a $6-th$ root of unity. There are $6$ choices in total, namely
  \[\chi(2) = e^{\frac{2i\pi k}{6}}, 1 \le k \le 6\]
  Clearly these are 6 distinct characters - as they have different values at residue class $[2]$ -  and they are all possible characters of $G_9$.

  We can further compute a character table as follows:
  $\chi(n)\pmod{9}$ \\
  \[\begin{tabular}{c c c c c c c }
      \hline
      $n$         & $1$ & $2$           & $4$          & $5$           & $7$          & $8$  \\
      \hline
      $\chi_1(n)$ & $1$ & $1$           & $1$          & $1$           & $1$          & $1$  \\
      $\chi_2(n)$ & $1$ & $\omega$      & $\omega^{2}$ & $-\omega^{2}$ & $-\omega$    & $-1$ \\
      $\chi_3(n)$ & $1$ & $\omega^{2}$  & $-\omega$    & $-\omega$     & $\omega^{2}$ & $1$  \\
      $\chi_4(n)$ & $1$ & $-1$          & $1$          & $-1$          & $1$          & $-1$ \\
      $\chi_5(n)$ & $1$ & $-\omega$     & $\omega^{2}$ & $\omega^{2}$  & $-\omega$    & $1$  \\
      $\chi_6(n)$ & $1$ & $-\omega^{2}$ & $-\omega$    & $\omega$      & $\omega^{2}$ & $-1$ \\
      \hline
    \end{tabular}\]
  where  $\omega = e^{i\pi/3}$.
\end{proof}

%Question 4
\begin{exercise}
  Show that there is exactly one real, non-principal Dirichlet character $\chi \pmod 9$. Find $\chi(916)$.
\end{exercise}
\begin{proof}
  Continuing from the previous exercise, with a note that a character is called real character if its image lies entirely in
  $\mathbb{R}$, namely  we have a group homomorphism $\chi\colon G_9 \to \mathbb{R}$. Since we know that all character $\chi \in \hat{G}$ satisfy
  \[|\chi(a)| =1, \text{ for all } a \in G_9,\]
  we can deduce that a real character $\chi$ must satisfy $\chi(G_9) \subset \left\lbrace \pm 1 \right\rbrace$. Since
  we want to find a non-principal character, the only choices is that $\chi(G_9) = \left\lbrace \pm 1\right\rbrace$. As shown
  above, $\chi$ is defined solely by its value at $2$, so we must choose $\chi(2) =-1$. This implies that there is only
  one non-principal, real Dirichlet character over $G_9$.

  Note that $916 \equiv 16 = 2^4 \pmod 9$. Thus
  \[ \chi(916) = \chi (2^4) = (\chi(2)^4) = (-1)^4=1.\]
  And we are done.

  \textbf{Remark:} Note that if we already have the full character table as computed in the previous exercise, it can be deduced immediately that
  there is a unique non principal real character. In the given table, it is the character $\chi_4$.
\end{proof}
\end{document}