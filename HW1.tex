\documentclass[12pt]{article} % \documentclass{} is the first command in any LaTeX code.  It is used to define what kind of document you are creating such as an article or a book, and begins the document preamble
\usepackage{amsmath} % \usepackage is a command that allows you to add functionality to your LaTeX code

\usepackage[papersize={216mm,330mm},tmargin=20mm,bmargin=20mm,lmargin=20mm,rmargin=20mm]{geometry}
\usepackage[english]{babel}
\usepackage[utf8]{inputenc}
\usepackage{amsmath,amssymb,mathabx,amsthm}%\for eqref
\usepackage{lscape}
\usepackage{graphicx}
\usepackage[colorinlistoftodos]{todonotes}
\usepackage{fancyhdr}
\usepackage{hyperref} %creat hyperlink
\hypersetup{
    colorlinks=true,
    linkcolor=blue,
    filecolor=magenta,      
    urlcolor=cyan,
    pdftitle={Overleaf Example},
    pdfpagemode=FullScreen,
    } %set up a hyperlink to be in blue 
\newtheorem{theorem}{Theorem}
\newtheorem{definition}{Definition}
\pagestyle{fancy}
\fancyhf{}
\setlength\parindent{0pt} % noindent for the whole document.
\renewcommand{\baselinestretch}{1.2} % increase the distance between line.
\DeclareMathOperator{\frkh}{\mathfrak{h}}
\DeclareMathOperator{\frkg}{\mathfrak{g}}
\DeclareMathOperator{\ad}{ad}

\title{Lie theory - homework 1} % Sets article title
\author{Tri Nguyen - University of Alberta} % Sets authors name
\date{\today} % Sets date for date compiled

% The preamble ends with the command \begin{document}
\begin{document}
\maketitle
\textbf{Problem 1}
\begin{enumerate}
    \item By definition, we only need to check that
          \[[[a,b],c] \in \mathfrak{g} \quad \forall a,b \in \frkh, c \in \frkg,\]
          but this is clear as $\frkh$ is an ideal of $\frkg$, we could use Jacobi's identity to get
          \[[[a,b],c] = [b,[c,a]]+ [a,[b,c]] \in [\frkh,\frkh].\]
    \item Recall that $\mathcal{D}^{k+1} \frkg = [\frkg^k,\frkg^k]$. Clearly $\frkg$ is itself an ideal, so the fact that
          $\mathcal{D}^{k+1}\frkg$ follows immediately from part a and induction on $k$.
    \item In class, we called $\frkg$ semisimple iff it has no nontrivial solvable ideal. Note that abelian ideals are solvable, hence all abelian ideals are zero if $\frkg$ is semisimple.
          Conversely, assume that $\frkg$ is not semisimple, then it has a non trivial solvable ideal $\frkh$. In particular, we have a strictly decreasing chain of ideals as follows:
          \[\frkh = \frkh^{(0)} \supset \frkh^{(1)} \supset \ldots \supset \frkh^{(n)} \supset \frkh^{(n+1)}= (0)\]
          But this implies that $\frkh^{(n)}$ is a non trivial abelian ideal of $\frkg$ by part a.
\end{enumerate}
\textbf{Problem 2}
We compute $\ad x$ with respect to this basis. The other two are computed similarly.
\begin{align}
    \ad x
\end{align}
\end{document}