\documentclass[12pt]{article} % \documentclass{} is the first command in any LaTeX code.  It is used to define what kind of document you are creating such as an article or a book, and begins the document preamble
\usepackage{amsmath} % \usepackage is a command that allows you to add functionality to your LaTeX code

\usepackage[papersize={216mm,330mm},tmargin=20mm,bmargin=20mm,lmargin=20mm,rmargin=20mm]{geometry}
\usepackage[english]{babel}
\usepackage[utf8]{inputenc}
\usepackage{amsmath,amssymb,mathabx,amsthm}%\for eqref
\usepackage{lscape}
\usepackage{graphicx}
\usepackage[colorinlistoftodos]{todonotes}
\usepackage{fancyhdr}
\usepackage{hyperref} %creat hyperlink
\hypersetup{
    colorlinks=true,
    linkcolor=blue,
    filecolor=magenta,      
    urlcolor=cyan,
    pdftitle={Overleaf Example},
    pdfpagemode=FullScreen,
    } %set up a hyperlink to be in blue 
\newtheorem{theorem}{Theorem}
\newtheorem{definition}{Definition}
\pagestyle{fancy}
\fancyhf{}
\setlength\parindent{0pt} % noindent for the whole document.
\renewcommand{\baselinestretch}{1.2} % increase the distance between line.
\DeclareMathOperator{\frkh}{\mathfrak{h}}
\DeclareMathOperator{\Hom}{Hom}
\DeclareMathOperator{\frkg}{\mathfrak{g}}
\DeclareMathOperator{\ad}{ad}
\DeclareMathOperator{\gl}{\mathfrak{\mathfrak{gl}}(n,F)}
\DeclareMathOperator{\sl3}{\mathfrak{sl}_3(F)}


\title{Lie theory - homework 2} % Sets article title
\author{Tri Nguyen - University of Alberta} % Sets authors name
\date{\today} % Sets date for date compiled

% The preamble ends with the command \begin{document}
\begin{document}
\maketitle
\textbf{Problem 1}
\begin{proof}
    By definition of $\frkg$-module, we need to check that:
    \begin{enumerate}
        \item $(ax+by)\cdot f = a(x\cdot f)+ b(y\cdot f)$: But it is easy to see that for any $v \in V$:
              \begin{align*}
                  [(ax+by) \cdot f](v) & = (ax+by)\cdot f(v)- f((ax+by)\cdot v)                       \\
                                       & = a(x\cdot f)(v)+b(y\cdot f)(v) - af(x\cdot v)-b f(y\cdot v) \\
                                       & = a(x\cdot f)(v)+b(y\cdot f)(v)
              \end{align*}
        \item Similarly, we can also check that $x\cdot(af+bg) = a(x\cdot f)+ b(y \cdot g)$.
        \item $[xy]\cdot f = x\cdot (y\cdot f)-y\cdot (x \cdot f)$: Evaluating at $v \in V$, we get
              \begin{align*}
                  ([xy]\cdot f)(v) & = [xy]f(v)-f([xy]\cdot v)                                                       \\
                                   & =x\cdot (y\cdot f(v))-y\cdot(x\cdot f(v))-f(x\cdot (y\cdot v)-y\cdot(x\cdot v)) \\
                                   & = (x\cdot (y\cdot f)(v))-(y\cdot (x\cdot f)(v))                                 \\
                                   & =(x\cdot y\cdot f)(v)- (y\cdot x \cdot f)(v)
              \end{align*}
              This shows that $\Hom(V,W)$ is indeed a $\frkg$-module.
    \end{enumerate}

    Now we want to show that $\Hom(V,W)$ is isomorphic to $V^*\otimes W$ as vector spaces. Since for any
    $v \in V$, $f(v)$ is a scalar for $f \in V^*$, we define a bilinear map from $V^* \times W$ to $\Hom(V,W)$ as follows:
    \begin{align*}
        \overline{f} \colon V^* \times W & \to \Hom(V,W)       \\
        (f,w)                            & \mapsto f(*)\cdot w
    \end{align*}
    Clearly this is a well-defined biliear map, and thus there exists a linear map $f$ such that
    \begin{align*}
        f \colon V^* \otimes W & \to \Hom(V,W)                  \\
        h\otimes w             & \mapsto (v\mapsto h(v)\cdot w)
    \end{align*}
    Assume that $V$ has dimension $n$, then so does $V^*$. Let $\left\lbrace e_i\right\rbrace_{i=1}^n$ be a basis of $V$ and
    denote $\left\lbrace e^*_i\right\rbrace_{i=1}^n$ be its dual basis. Then for any $g \in \Hom(V,W)$ we define a map
    \begin{align*}
        g \colon  \Hom(V,W) & \to V^* \otimes W                 \\
        u                   & \mapsto \sum e_i^* \otimes u(e_i)
    \end{align*}
    Clearly $g$ is also well-defined and we have that
    \[(f\circ g(u))(v) = f\left(\sum e_i^* \otimes u(e_i)\right)(v) = \sum e_i^*(v)\cdot u(e_i)=u(v)\]
    and
    \[g\circ f (h\otimes w)=g(h()\cdot w)=\sum e_i^* \otimes h(e_i)w = (\sum h(e_i)e_i^*)\otimes w=h\otimes w \]
    So $g$ and $f$ are inverse of each other, which implies the desired isomorphism.
\end{proof}
\textbf{Problem 2}
\begin{proof}
    Without lost of generalization, we assume that we are working over algebraically closed field.
    Thus an element $x$ is semisimple if it is diagonalizable. Clearly if $x,y$ are nilpotent, then there are $n,m \in \mathbb{Z}$ such that
    $x^n=0$ and $y^m=0$. Since $xy=yx$ by hypothesis, we have that
    \[(x+y)^{m+n} = \sum_{k=1}^{m+n} x^k y^{m+n-k} = 0\]
    \textbf{Claim: If two matrices are commute, they are simultaneously diagonalizable.}

    Using the above claim, we can choose a basis of $V$ such that, with respect to that basis, $x,y$ are two diagonal matrices.
    Then it is clear that $x+y$ is also diagonal.
    From the obvervation above, if $x = x_s+x_n$ and $y=y_s+y_n$, then we must have
    $x_s+y_s$ is semisimple and $x_n+y_n$ is nilpotent. Moreover
    \[(x_s+y_s)+(x_n+y_n) = x+y\]
    By the uniqueness of decomposition into semisimple and nilpotent part, we conclude that
    \[(x+y)_s=x_s+y_s,\quad (x+y)_n = x_n+y_n\]
    A proof of the claim can be found here: \url{https://math.stackexchange.com/questions/2905474/regarding-a-proof-of-if-a-b-in-m-n-mathbbk-are-diagonalizable-and-commu}
\end{proof}
\textbf{Problem 3}
\begin{proof}
    Given the Killing form is nondegenerate, the corresponding Lie algebra will be semisimple. The proof of this part is identical
    to the proof in class (or in Humphrey's textbook). We will show that the converse is not always true, i.e. a semisimple Lie algebra
    can have degenerate Killing form. Using the hint, we look at the quotient algebra of $\sl3$ of its center over the field $F$ of characteristic 3.

    First we show that the quotient is semisimple. Indeed, let denote this quotient by $L$. Any solvable ideals of $L$ corresponds to a solvable
    ideals of $\sl3$ containing the center. But it is well know that $\sl3$ is semisimple, thus any solvable ideals of $\sl3$ is trivial, thus implies any solvable ideal
    of $L$ is trivial. Hence $L$ is semisimple, by definition.

    Now the Gram matrix with respect to the Killing form over $L$ is given by
    \[\begin{pmatrix}
            12 & -6 & 0 & 0 & 0 & 0 & 0 & 0\cr
            -6 & 12 & 0 & 0 & 0 & 0 & 0 & 0\cr
            0  & 0  & 0 & 0 & 6 & 0 & 0 & 0 \cr
            0  & 0  & 0 & 0 & 0 & 0 & 6 & 0 \cr
            0  & 0  & 6 & 0 & 0 & 0 & 0 & 0 \cr
            0  & 0  & 0 & 0 & 0 & 0 & 0 & 6 \cr
            0  & 0  & 0 & 6 & 0 & 0 & 0 & 0 \cr
            0  & 0  & 0 & 0 & 0 & 6 & 0 & 0
        \end{pmatrix}\]
    which is identially 0 matrix over field of characteristic 3, hence the Killing form is degenerate.

    Below I provided how to compute the first few entries of the Grammian matrix of the Killing form -  is essentially the same for every
    other entries.

    For example, we choose a basis of the Cartan subalgebra of $\sl3$ as
    \begin{align*}
        H_{12}= \begin{pmatrix}
                    1 & 0  & 0 \\
                    0 & -1 & 0 \\
                    0 & 0  & 0
                \end{pmatrix}, \quad
        H_{23} = \begin{pmatrix}
                     0 & 0 & 0  \\
                     0 & 1 & 0  \\
                     0 & 0 & -1
                 \end{pmatrix}
    \end{align*}
    It can be verified that $\ad(H_{12})(E_{ij}) =(\omega_i - \omega_j)E_{ij}$ where $\omega = (1,-1,0)$ for $H_{13}$.
    So as a matrix, with respect to the ordered basis
    \[\left\lbrace H_{12},H_{23}, E_{12},E_{21},E_{13},E_{31|},E_{23},E_{32}\right\rbrace\]
    $\ad H_{12}$ is of the form
    \begin{align*}
        \begin{pmatrix}
            0 & 0 & 0 & 0  & 0 & 0  & 0  & 0\cr
            0 & 0 & 0 & 0  & 0 & 0  & 0  & 0\cr
            0 & 0 & 2 & 0  & 0 & 0  & 0  & 0 \cr
            0 & 0 & 0 & -2 & 0 & 0  & 0  & 0 \cr
            0 & 0 & 0 & 0  & 1 & 0  & 0  & 0 \cr
            0 & 0 & 0 & 0  & 0 & -1 & 0  & 0 \cr
            0 & 0 & 0 & 0  & 0 & 0  & -1 & 0 \cr
            0 & 0 & 0 & 0  & 0 & 0  & 0  & 1
        \end{pmatrix}
    \end{align*}
    In a similar way, we have
    \[\ad H_{23} = \begin{pmatrix}
            0 & 0 & 0  & 0 & 0 & 0  & 0 & 0\cr
            0 & 0 & 0  & 0 & 0 & 0  & 0 & 0\cr
            0 & 0 & -1 & 0 & 0 & 0  & 0 & 0 \cr
            0 & 0 & 0  & 1 & 0 & 0  & 0 & 0 \cr
            0 & 0 & 0  & 0 & 1 & 0  & 0 & 0 \cr
            0 & 0 & 0  & 0 & 0 & -1 & 0 & 0 \cr
            0 & 0 & 0  & 0 & 0 & 0  & 2 & 0 \cr
            0 & 0 & 0  & 0 & 0 & 0  & 0 & -2
        \end{pmatrix}\]
    From this we have that $\kappa(H_{12},H_{12})= \kappa(H_{23},H_{23}) =12$ and $\kappa(H_{13},H_{23})=-6$
\end{proof}
\end{document}