\documentclass[12pt]{article} % \documentclass{} is the first command in any LaTeX code.  It is used to define what kind of document you are creating such as an article or a book, and begins the document preamble

\usepackage{amsmath} % \usepackage is a command that allows you to add functionality to your LaTeX code

\usepackage[papersize={216mm,330mm},tmargin=20mm,bmargin=20mm,lmargin=20mm,rmargin=20mm]{geometry}
\usepackage[english]{babel}
\usepackage[utf8]{inputenc}
\usepackage{amsmath,amssymb,mathabx,amsthm}%\for eqref
\usepackage{lscape}
\usepackage{graphicx}
\usepackage[colorinlistoftodos]{todonotes}
\usepackage{fancyhdr}
\usepackage{hyperref} %creat hyperlink
\hypersetup{
    colorlinks=true,
    linkcolor=blue,
    filecolor=magenta,      
    urlcolor=cyan,
    pdftitle={Overleaf Example},
    pdfpagemode=FullScreen,
    } %set up a hyperlink to be in blue 
\newtheorem{theorem}{Theorem}
\newtheorem{definition}{Definition}
\pagestyle{fancy}
\fancyhf{}
\setlength\parindent{0pt} % noindent for the whole document.
\renewcommand{\baselinestretch}{1.2} % increase the distance between line.

\title{Conformal equivalence between annuli} % Sets article title
\author{Tri Nguyen - University of Alberta} % Sets authors name
\date{\today} % Sets date for date compiled

% The preamble ends with the command \begin{document}
\begin{document} % All begin commands must be paired with an end command somewhere
\maketitle % creates a title using the information in the preamble (title, author, date)
Disclaimer: This is just a rewritten version of the blog here:
\href{http://cykenleung.blogspot.com/2012/04/conformal-equivalence-of-annuli.html}{see here}.

First, we define an annulus in a complex plane
\begin{definition}
  An annulus in $\mathbb{C}$ is the set
  \[A(r,R) = \left\lbrace z \in \mathbb{C}: r < |z| <R \right\rbrace\]
\end{definition}
Given two annuli $A_1 = A(r_1, R_1)$ and $A_2 = A(r_2, R_2)$, one may ask under which conditions
that two annuli are biholomorphic. It turns out that in a complex plane, the biholomorphic relation is defined using only
the ratio $r_1/r_2$ and $R_1/R_2$. This is shown in the following theorem

\begin{theorem}
  $A_1$ is biholomorphic to $A_2$ if and only if $\dfrac{R_1}{r_1} = \dfrac{R_2}{r_2}$.
\end{theorem}
\begin{proof}
  First suppose that $\dfrac{R_1}{R_2} = \dfrac{r_1}{r_2} =k$. Then clearly the linear map $f(z)=kz$
  is a biholomorphic map and $f(A_1) = A_2$. Thus $A_1$ is biholomorphic to $A_2$.

  Conversely, assume that $A_1$ is biholomorphic to $A_2$ under a map $f$. By scaling, we could further assume that $r_1=r_2=1$. Thus now we need to show that $R_1=R_2$.
  Fix some $1<r<R_2$ and let $C = \left\lbrace z \in A_2 \colon |z|=r\right\rbrace$.

  Since $f^{-1}$ is a continuous map, $f^{-1}(C)$ is a compact set. Thus we can find
  an $m>0$ such that $|x| \ge m$ for all $x \in f^{-1}(C)$. Choose $\delta>0$ small enough
  such that $1+\delta<m$. This implies the annulus $A_3 = A(1,1+\delta) \cap f^{-1}(C)=\emptyset$. Let
  $V = f(A_3)$. Since $A_3$ is connected set, so is $f(A_3)$. Thus $V$ is either inside
  $A_2\setminus A(1,r)$ or $A(1,r)$. By replacing $f$ with $R_2/f$, we can reduce to consider the former case.

  \textbf{Claim:} $|f(z_n)| \to 1$ whenever $|z_n| \to 1$.

  \textit{Proof:} Clearly we have $z_n = f^{-1}(f(z_n))$. If $f(z_n)$ converges to some points
  in $A_2$, $z_n$ must then converges to some points in $A_1$, contradicting the hypothesis that
  $|z_n| \to 1$. Thus $|f(z_n)|$ must converge to either $1$ or $R_2$, but the latter case is excluded as $V \subset A(1,r)$.

  In the same manner, we also have the following claim:

  \textbf{Claim:} $|f(z_n)| \to R_2$ whenever $|z_n| \to R_1$.

  Now set $\alpha = \log(R_1)/\log(R_2)$ and define a new function $g \colon A_1 \to \mathbb{R}$ by
  \[g(z) = \log|f(z)|^2 - \alpha\log|z|^2 = 2(\log|f(z)| - \alpha\log|z|).\]
  This is a harmonic function since $\log(|f|^2)$ is a harmonic over $\mathbb{C}$, where $f$ is a non vanishing analytic function. Indeed, we have
  \[\Delta\left(\log(|f|^2)\right)= \Delta\left(\log(f)+\log(\overline{f})\right)
    = 4\dfrac{\partial}{\partial z}\dfrac{\partial}{\partial \overline{z}}\log(f)+4\dfrac{\partial}{\partial \overline{z}}\dfrac{\partial}{\partial z}\log(\overline{f})=0\]

  Using the above claims, $g$ can be
  extended continuously to $\overline{A_1}$ and $g(z)=0$ for all $z \in \partial A_1$. By maximum
  modulus theorem for harmonic function, $g$ must be identically zero on the whole disk $\overline{A_1}$.
  In particular
  \[ 0 = \dfrac{\partial g}{\partial z} = \dfrac{f'(z)}{f(z)}-\dfrac{\alpha}{z}.\]

  Let $\gamma = \gamma(t)= ce^{it}$ for some $1<c<R_1$. Then we have
  \[\alpha = \dfrac{1}{2\pi i}\int_{\gamma}\dfrac{\alpha}{z} =\dfrac{1}{2\pi i}\int_{\gamma}\dfrac{f'(z)}{f(z)} \]

  By the Argument Principle, $\alpha$ must be an integer. A consequence is that
  \[\dfrac{d}{dz}(z^{-\alpha}f(z))= -\alpha z^{-\alpha-1}f(z)+z^{-\alpha}f'(z) = z^{-\alpha-1}(zf'(z)-\alpha f(z))=0.\]

  This forces $z^{-\alpha}f(z)=K$ for some constant $K$ on $A_1$. Thus $f(z)=Kz^{\alpha}$. As f is injective, $\alpha=1$ by the fundamental theorem of Algebra.
\end{proof}

Let $u = \cos(\theta)$ then $du=-\sin(\theta)$. Then we have
\[\int\frac{\cos^2(\theta)}{\sin(\theta)}d\theta = \int \frac{\cos^2(\theta)}{\sin^2(\theta)}\sin(\theta)d\theta = \int\frac{\cos^2(\theta)}{1-\cos^2(\theta)}\sin(\theta)d\theta = \frac{u^2}{1-u^2}du \]
\end{document} % This is the end of the document