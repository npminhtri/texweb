\documentclass{article} % \documentclass{} is the first command in any LaTeX code.  It is used to define what kind of document you are creating such as an article or a book, and begins the document preamble

\usepackage{amsmath} % \usepackage is a command that allows you to add functionality to your LaTeX code

\usepackage[papersize={216mm,330mm},tmargin=20mm,bmargin=20mm,lmargin=20mm,rmargin=20mm]{geometry}
\usepackage[english]{babel}
\usepackage[utf8]{inputenc}
\usepackage{amsmath,amssymb,mathabx,amsthm}%\for eqref
\usepackage{lscape}
\usepackage{graphicx}
\usepackage[colorinlistoftodos]{todonotes}
\usepackage{fancyhdr}
\newtheorem{theorem}{Theorem}
\newtheorem{definition}{Definition}
\pagestyle{fancy}
\fancyhf{}
\setlength\parindent{0pt}


\title{Conformal equivalence between annuli} % Sets article title
\author{Tri Nguyen} % Sets authors name
\date{\today} % Sets date for date compiled

% The preamble ends with the command \begin{document}
\begin{document} % All begin commands must be paired with an end command somewhere
\maketitle % creates a title using the information in the preamble (title, author, date)
\noindent \textbf{Introduction}
First we give a definition for an annulus in complex plane:
\begin{definition}
  An annulus in $\mathbb{C}$ is the set
  \[A(r,R) = \left\lbrace z \in \mathbb{C}: r < |z| <R \right\rbrace\]
\end{definition}
Given two annuli $A_1 = A(r_1, R_1)$ and $A_2 = A(r_2, R_2)$, one may ask under which conditions
that two annuli are biholomorphic. It turns out that in a complex plane, the biholomorphic relation is defined using only
the ratio $r_1/r_2$ and $R_1/R_2$. This is shown in the following theorem

\begin{theorem}
  $A_1$ is biholomorphic to $A_2$ if and only if $\dfrac{R_1}{r_1} = \dfrac{R_2}{r_2}$.
\end{theorem}
\begin{proof}
  First suppose that $\dfrac{R_1}{R_2} = \dfrac{r_1}{r_2} =k$. Then clearly the linear map $f(z)=kz$
  is a biholomorphic map and $f(A_1) = A_2$. Thus $A_1$ is biholomorphic to $A_2$.

  Conversely, assume that $A_1$ is biholomorphic to $A_2$. By scaling if necessary, we could
  further assume that $r_1=r_2=1$. Thus now we need to show that $R_1=R_2$. Fix some $1<r<R_2$.

\end{proof}
\end{document} % This is the end of the document