\documentclass[12pt]{article} % \documentclass{} is the first command in any LaTeX code.  It is used to define what kind of document you are creating such as an article or a book, and begins the document preamble

\usepackage{amsmath} % \usepackage is a command that allows you to add functionality to your LaTeX code

\usepackage{blindtext}
\usepackage{titlesec} %Creat table of contents. 

\usepackage[papersize={216mm,330mm},tmargin=20mm,bmargin=20mm,lmargin=20mm,rmargin=20mm]{geometry}
\usepackage[english]{babel}
\usepackage[utf8]{inputenc}
\usepackage{amsmath,amssymb,mathabx,amsthm}%\for eqref
\usepackage{lscape}
\usepackage{graphicx}
\usepackage{tikz-cd}
\usepackage[colorinlistoftodos]{todonotes}
\usepackage{fancyhdr}
\usepackage{hyperref} %creat hyperlink
\hypersetup{
    colorlinks=true,
    linkcolor=blue,
    filecolor=magenta,      
    urlcolor=cyan,
    pdftitle={Overleaf Example},
    pdfpagemode=FullScreen,
    } %set up a hyperlink to be in blue 
\newtheorem{theorem}{Theorem}
\newtheorem{definition}{Definition}
\pagestyle{fancy}
\fancyhf{}

\DeclareMathOperator{\Aff}{\mathbb{A}^n}
\DeclareMathOperator{\Hom}{Hom}
\DeclareMathOperator{\Sp}{Spec}
\setlength\parindent{0pt} % noindent for the whole document.
\renewcommand{\baselinestretch}{1.2} % increase the distance between line.

\title{Introduction to Schemes} % Sets article title
\author{Lecturer: Maria Iakerson} % Sets authors name
\date{\today} % Sets date for date compiled

% The preamble ends with the command \begin{document}
\begin{document} % All begin commands must be paired with an end command somewhere
\maketitle % creates a title using the information in the preamble (title, author, date)
\tableofcontents
\section{Why Schemes?}
\subsection{Summary of affine varieties}
Let $k$ be an algebraic closed field. The main idea of classical algebraic geometry is that we
have a correspondence
\begin{align*}
      \left\lbrace\text{subsets of $k^n$ cut} \text{out by polynomials} \right\rbrace & \leftrightarrow
      \left\lbrace \text{finitely generated reduced
      $k-$ algebras }\right\rbrace                                                                                     \\
      \text{Geometry}                                                                 & \leftrightarrow \text{Algebra}
\end{align*}
In particular, the above correspondence can be given as follows:
\begin{itemize}
      \item $I \subset k[x_1,\ldots,x_n]$ ideal: Then we define \[X:=Z(I) = \left\lbrace a \in k^n| f(a) =0 \quad\forall f \in I\right\rbrace\]
            sometimes we use the notation $V(I)$ instead of $Z(I)$. This kind of set is an affine variety.
      \item $\Aff$: $n-$ dimensional affine space. As a set, it is just $k^n$, but we equip this set with
            \textit{Zariski} topology - where the closed subsets are generated by $Z(I)$.
      \item $I(X):=\left\lbrace f \in k[x_1,\ldots,x_n]| f(x) =0 \quad \forall x \in X\right\rbrace$. Then the quotient replacing
            $k[X]:= \dfrac{k[x_1,\ldots,x_n]}{I(X)}$ is called \textit{coordinate ring} of $X$.
      \item $k[X]$ parametrizes functions on $X$:
            \[x \in X \rightsquigarrow \mathfrak{m}_x:=\ker(ev_x: k[X] \to k)\]
            and $\forall f \in k[x]$ gives
            \begin{align*}
                  f \colon X & \to \mathbb{A}^1 = k                                \\
                  x          & \mapsto f(x) = \overline{f} \in k[x]/\mathfrak{m}_x
            \end{align*}
      \item Hilbert's weak Nullstellensatz:
            \[\left\lbrace\text{ points of } X\right\rbrace \leftrightarrow \left\lbrace \text{ maximal ideals of }k[X]\right\rbrace\]
      \item Hilbert Nullstellensatz: $I(Z(I))=\sqrt{I}:=\left\lbrace f: f^n \in I \text{ for some } n\right\rbrace$.
      \item Morphisms: given $X$ and $Y \in \Aff$ a morphism between two affine varieties is given by
            $\varphi=(f_1,\ldots,f_n)$. This morphism induces a $k-$ algebra homomorphism
            \[\varphi^* \colon k[Y] \to k[X],\quad \varphi^*(\psi) = \psi \circ \varphi, \qquad \qquad
                  \begin{tikzcd}
                        X \arrow[r, "\varphi"] \arrow[dr, dashrightarrow, "\varphi^*f",swap] & Y \arrow[d, "f"]\\ & \mathbb{A}^1 \end{tikzcd}\]

            so $\Hom(X,Y) = \Hom(k[Y], k[X])$ - which gives the equivalence of categories as stated in the beginning.
\end{itemize}
\subsection{Why varieties are not good enough?}
Some possible reasons are:
\begin{enumerate}
      \item embedding into $\Aff$ shouldn't be part of the data. It would be nice
            to have an intrinsic definition, since you can embed the same variety in different spaces.
      \item for non-algebraic closed field, the Nullstellensatz doesn't work: $I = (x^2+y^2+1)$ is a prime ideal in $\mathbb{R}[x,y]$, hence is a radical ideal.
            But $Z(I)= \varnothing$, so $I(Z(I)) = \mathbb{R}[x,y].$
      \item We can ask, on which topological space is $\mathbb{R}[x,y]/(x^2+y^2+1)$ naturally a functions space? Or $\mathbb{Z}[x]$? Or $\mathbb{Z}$?
            This leads to the idea of considering all possible rings.
      \item Nilpotent arises naturally when deforming varieties, so ignoring them is not a good option.
\end{enumerate}
\newpage
\section{The prime spectrum}
In the last sections, we have an equivalence between categories
\[\text{ affine varieties over $k=\overline{k}$} \cong_{op} \text{ reduced finitely generated $k-$ algebras}\]
Now we want to extend this equivalent relation as follows:
\[\text{ affine schemes } \cong_{op} \text{ commutative ring with unit}\]
This generalization allows to study arithmetic phenomena by geometric methods, by taking rings to be $\mathbb{Z},\mathbb{Z}_p$, etc.

\begin{definition}
      Given a ring $R$, its spectrum is defined to be
      \[\Sp R: = \left\lbrace \mathfrak{p} | \mathfrak{p} \subset R \text{ is a prime ideal}\right\rbrace\]
\end{definition}
This way ,$x \in \Sp R \leftrightsquigarrow \mathfrak{p}_x \in R$.

\textbf{NB:} in general, we cannot think about $f \in R$ as functions with values in a fixed field $k$. However, there is a more general notion.
\begin{definition}
      Let $x \in \Sp R$ correspond to $\mathfrak{p} \subset R$. The \textbf{residue field} of $x$
      (or $\mathfrak{p}$) is
      \[\kappa(x) = \kappa(\mathfrak{p}):= R_\mathfrak{p}/\mathfrak{p}R_\mathfrak{p}\]
\end{definition}
Every $f \in R$ has a "value"
\[f(x):=f \mod \mathfrak{p}_x \in \kappa(x), \quad \forall x \in \Sp R\]
and the codomain depends on the choice of $x$. By definition, $f(x)=0$ if $f \in \mathfrak{p}_x$. The moral of
the story is that $\Sp R$ will be the space on which $R$ is the ring of functions: the affine scheme corresponding to $R$.
\end{document} % This is the end of the document