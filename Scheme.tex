\documentclass[12pt]{article} % \documentclass{} is the first command in any LaTeX code.  It is used to define what kind of document you are creating such as an article or a book, and begins the document preamble
\usepackage[papersize={216mm,330mm},tmargin=20mm,bmargin=20mm,lmargin=20mm,rmargin=20mm]{geometry}
\usepackage[english]{babel}
\usepackage[utf8]{inputenc}
\usepackage{amsmath,amssymb,mathabx,amsthm}%\for eqref
\usepackage{lscape}
\usepackage{graphicx}
\usepackage{tikz}\usetikzlibrary{arrows.meta,calc} %library tikz


\usepackage{pgfplots}
\pgfplotsset{compat=1.15}
\usepackage{mathrsfs}
\usetikzlibrary{arrows}
\usepackage[framemethod=TikZ]{mdframed}
\usepackage{tikz, tabularx}
\usepackage[table,x11names]{xcolor}
\usepackage{graphics}
\usepackage{hyperref}
\newcolumntype{W}{>{\centering\arraybackslash}X}%Para agilizar las columnas.
% -------------------
% Font
% -------------------
\usepackage[T1]{fontenc}
\usepackage{charter}
\usepackage{color,soul} %package for highlining
\usepackage[colorinlistoftodos]{todonotes}
\usepackage{fancyhdr}
\usepackage{hyperref} %creat hyperlink
\hypersetup{
    colorlinks=true,
    linkcolor=blue,
    filecolor=magenta,      
    urlcolor=cyan,
    pdftitle={Overleaf Example},
    pdfpagemode=FullScreen,
    } %set up a hyperlink to be in blue 
\newtheorem{theorem}{Theorem}
\newtheorem{definition}{Definition}[section]
\newtheorem{prop}[definition]{Proposition}
\newtheorem{lemma}[definition]{Lemma}
\newtheorem*{remark}{Remark}

\pagestyle{fancy}
\cfoot{\thepage} % this is for the page numbering
\setlength\parindent{0pt} % noindent for the whole document.
\renewcommand{\baselinestretch}{1.1} % increase the distance between line.

% -------------------
% Commands
% -------------------
\newcommand{\homework}[2]{\noindent\textbf{Name: Kyra Xian
}{} \hfill \textbf{} \\  \textbf{Due date: #2} \hfill \textbf{}\\}

\newcommand{\prob}{\noindent\textbf{Problem. }}
\newcounter{problem}
\newcommand{\problem}{
	\stepcounter{problem}%
	\noindent \textbf{Problem \theproblem. }%
}
\newcommand{\pointproblem}[1]{
	\stepcounter{problem}%
	\noindent \textbf{Problem \theproblem.} (#1 points)\,%
}
\newcommand{\pspace}{\par\vspace{\baselineskip}}
\newcommand{\ds}{\displaystyle}
% -------------------
% Theorem Environment
% -------------------
\mdfdefinestyle{theoremstyle}{%
	frametitlerule=true,
	roundcorner=5pt,
	linecolor=black,
	outerlinewidth=0.5pt,
	middlelinewidth=0.5pt
}
\mdtheorem[style=theoremstyle]{exercise}{\textbf{Problem}}
\DeclareMathOperator{\sl2c}{\mathfrak{sl}(2,\mathbb{C})}
\DeclareMathOperator{\rsl2c}{\mathfrak{sl}(2,\mathbb{C})_{\mathbb{C}\mid \mathbb{R}}}
\DeclareMathOperator{\lieg}{\mathfrak{g}}
\DeclareMathOperator{\s2r}{\mathfrak{sl}(2,\mathbb{R})}

\title{SOLUTION FOR THE BONUS QUESTION} % Sets article title
\date{\today} % Sets date for date compiled

% The preamble ends with the command \begin{document}
\begin{document} % All begin commands must be paired with an end command somewhere
\maketitle
In this report, I will give a proof for the bonus question for the class MATH 538.
First we recall the question
\begin{exercise}
      Consider the Lie algebra $\lieg = \sl2c$ viewed as a real Lie algebra. We will denote this
      Lie algebra by $\rsl2c$. Show that
      \begin{enumerate}
            \item $\lieg$ is simple as a real Lie algebra.
            \item $\rsl2c$ is not a split real Lie algebra, i.e. it has no split Cartan subalgebra.
      \end{enumerate}
\end{exercise}
\begin{proof}
      \hfill \\

      \begin{enumerate}
            \item First we will show that $\rsl2c$ is semisimple. Indeed, we have
                  \[\rsl2c \otimes \mathbb{C} = (\mathfrak{sl}(2,\mathbb{R}) \otimes_\mathbb{R} \mathbb{C}) \otimes_\mathbb{R} \mathbb{C} = \s2r \otimes_\mathbb{R}(\mathbb{C}\otimes_\mathbb{R}\mathbb{C}) \]
                  Now note that $\mathbb{C} \otimes_\mathbb{R} \mathbb{C} \cong \mathbb{C} \oplus \mathbb{C}$ as rings, so
                  we clearly reduce to
                  \[\rsl2c \otimes \mathbb{C} \cong \s2r \otimes_\mathbb{R}(\mathbb{C}\oplus \mathbb{C}) \cong \sl2c \oplus \sl2c\]
                  This clearly shows that $\rsl2c$ is semi-simple. Now we show that it is in fact simple.
                  Assume not, then $\rsl2c$ has a nontrivial simple ideal $I$. Then for any $\lambda \in \mathbb{C}\setminus \left\lbrace 0 \right\rbrace$, clearly $\lambda(I)$ is also an ideal as
                  \[I \supset [I,\lambda(I)] = \lambda[I,I] =\lambda(I) \ne 0\]
                  Clearly $\dim I = \dim \lambda(I)$, this implies that $I = \lambda(I)$. That means when we allow complex scalar multiplication
                  the ideal $I$ stays the same. Thus $I$ is in fact an ideal of $\sl2c$. But since $\sl2c$ is simple
                  as a complex Lie algebra, this implies $I=\sl2c$. The restriction of scalar doesn't change anything, which implies
                  $\rsl2c$ is simple.
            \item Assume not, then $\rsl2c$ is a split real Lie algebra. From the previous part, we regconized that
                  $\rsl2c$ is a real form of the semisimple Lie algebra $\lieg$. Assume that
                  the real Lie algebra $\rsl2c$ is split, then we have a root space decomposition
                  \[\rsl2c = \mathfrak{h}\oplus \bigoplus_{\alpha \in \Delta} \mathfrak{g}_\alpha\]
                  where $\Delta$ is the set of root with respect to the split Cartan subalgebra $\mathfrak{h}$.
                  We know that the root space $\mathfrak{g}_\alpha$ has dimension 1, so we can rewwrite the above as
                  \[\rsl2c = \mathfrak{h}\oplus \bigoplus_{\alpha \in \Delta} \mathbb{R}x_\alpha\]
                  Under the base changing, we get
                  \[\sl2c \oplus \sl2c = \rsl2c \otimes \mathbb{C} = \left(\mathfrak{h}\oplus \bigoplus_{\alpha \in \Delta} \mathbb{R}x_\alpha\right)\otimes \mathbb{C} =\left(\mathfrak{h}\otimes \mathbb{C}\right) \oplus\bigoplus_{\alpha \in \Delta}\mathbb{C}x_\alpha\]
      \end{enumerate}
      In particular, the root system doesn't change after the base changing, and we can see that the corresponding
      root system to the given Lie algebra is of type $A_1 \times A_1$. But this is a reducible root system, so the
      corresponding split real Lie algebra must also reducible. We will show that
      if a root system corresponding to a semisimple Lie algebra is reducible, then the corresponding Lie algebra
      is not semisimple.

      Indeed, assume that $\Delta = \Delta_1 \sqcup \Delta_2$, then clearly the subalgebra
      \[\mathfrak{g}_1 = \bigoplus_{\alpha \in \Delta_1^+} \left(\mathfrak{g}_\alpha+\mathfrak{g}_{-\alpha} + [\mathfrak{g}_\alpha,\mathfrak{g}_{-\alpha}]\right)\]
      is a proper ideal of $\lieg$, as
      \begin{align*}
            [\lieg_\alpha,\lieg_\beta]= \begin{cases}
                                              0                     & \quad \text{ if $\alpha + \beta \notin \Delta$, in particular $\alpha \in \Delta_1, \beta \in \Delta_2$} \\
                                              \lieg_{\alpha+\beta}, & \quad \text{ otherwise}
                                        \end{cases}.
      \end{align*}
      So the real Lie algebra $\rsl2c$, if it is split, must be semisimple and has a nontrivial simple ideal. But as we proved above
      $\rsl2c$ is a simple real Lie algebra, so this proves $\rsl2c$ can't not be a split
\end{proof}

\end{document} % This is the end of the document
