\documentclass[11pt]{amsart}
\usepackage{amsmath}
\usepackage{amssymb}
\usepackage{amscd}
\usepackage{amsthm}

\usepackage[all]{xy}

%graphics
%\usepackage{graphicx}
%\usepackage{pstricks}
%\usepackage{pst-plot}

%\setlength{\textwidth}{4.7in}
%\setlength{\textheight}{7.5in}

\newtheorem{thm}{Theorem}[section]
\newtheorem{lem}[thm]{Lemma}
\newtheorem{cor}[thm]{Corollary}
\newtheorem{conj}[thm]{Conjecture}
\newtheorem{prop}[thm]{Proposition}
\newtheorem{question}[thm]{Question}
\newtheorem{claim}[thm]{Claim}
\newtheorem{fact}[thm]{Fact}

\theoremstyle{remark}

\newtheorem{rem}[thm]{Remark}

\theoremstyle{definition}
\newtheorem{defn}[thm]{Definition}

\numberwithin{equation}{section}

\allowdisplaybreaks[4]

\DeclareMathOperator{\rank}{rank}
\DeclareMathOperator{\codim}{codim}
\DeclareMathOperator{\Ord}{Ord}
\DeclareMathOperator{\Var}{Var}
\DeclareMathOperator{\Ext}{Ext}
\DeclareMathOperator{\Pic}{Pic}
\DeclareMathOperator{\Spec}{Spec}
\DeclareMathOperator{\Jac}{Jac}
\DeclareMathOperator{\Div}{Div}
\DeclareMathOperator{\sgn}{sgn}
\DeclareMathOperator{\supp}{supp}
\DeclareMathOperator{\Hom}{Hom}
\DeclareMathOperator{\Sym}{Sym}
\DeclareMathOperator{\nilrad}{nilrad}
\DeclareMathOperator{\Ann}{Ann}
\DeclareMathOperator{\Proj}{Proj}
\DeclareMathOperator{\Aut}{Aut}

\begin{document}

\vfuzz0.5pc
\hfuzz0.5pc % Don't bother to report 
% overfull boxes if overage is < 6pt

\newcommand{\claimref}[1]{Claim \ref{#1}}
\newcommand{\thmref}[1]{Theorem \ref{#1}}
\newcommand{\propref}[1]{Proposition \ref{#1}}
\newcommand{\lemref}[1]{Lemma \ref{#1}}
\newcommand{\coref}[1]{Corollary \ref{#1}}
\newcommand{\remref}[1]{Remark \ref{#1}}
\newcommand{\conjref}[1]{Conjecture \ref{#1}}
\newcommand{\questionref}[1]{Question \ref{#1}}
\newcommand{\defnref}[1]{Definition \ref{#1}}
\newcommand{\secref}[1]{Sec. \ref{#1}}
\newcommand{\ssecref}[1]{\ref{#1}}
\newcommand{\sssecref}[1]{\ref{#1}}

%\def \d#1{\displaystyle{#1}}
%\def \mult{\mathop{\mathrm{mult}}\nolimits}
%\def \br{\mathop{\mathrm{br}}\nolimits}
%\def \rank{\mathop{\mathrm{rank}}\nolimits}

%\def \codim{\mathop{\mathrm{codim}}\nolimits}
%\def \Ord{\mathop{\mathrm{Ord}}\nolimits}
%\def \Var{\mathop{\mathrm{Var}}\nolimits}
%\def \Ext{\mathop{\mathrm{Ext}}\nolimits}
%\def \Pic{\mathop{\mathrm{Pic}}\nolimits}
%\def \Spec{\mathop{\mathrm{Spec}}\nolimits}
%\def \Jac{{\mathrm{Jac}}}

%\def \Div{{\mathrm{Div}}}
%\def \sgn{{\mathrm{sgn}}}
%\def \Hom{{\mathrm{Hom}}}
%\def \Sym{{\mathrm{Sym}}}
%\def \nilrad{{\mathrm{nilrad}}}
%\def \Supp{{\mathrm{Supp}}}
%\def \Ann{{\mathrm{Ann}}}

\newcommand{\RED}{{\mathrm{red}}}
\newcommand{\tors}{{\mathrm{tors}}}
\newcommand{\eq}{\Leftrightarrow}

\newcommand{\mapright}[1]{\smash{\mathop{\longrightarrow}\limits^{#1}}}
\newcommand{\mapleft}[1]{\smash{\mathop{\longleftarrow}\limits^{#1}}}
\newcommand{\mapdown}[1]{\Big\downarrow\rlap{$\vcenter{\hbox{$\scriptstyle#1$}}$}}
\newcommand{\smapdown}[1]{\downarrow\rlap{$\vcenter{\hbox{$\scriptstyle#1$}}$}}

\newcommand{\A}{{\mathbb A}}
\newcommand{\I}{{\mathcal I}}
\newcommand{\J}{{\mathcal J}}
\newcommand{\CO}{{\mathcal O}}
\newcommand{\C}{{\mathcal C}}
\newcommand{\BC}{{\mathbb C}}
\newcommand{\BQ}{{\mathbb Q}}
\newcommand{\m}{{\mathcal M}}
\newcommand{\h}{{\mathcal H}}
\newcommand{\Z}{{\mathcal Z}}
\newcommand{\BZ}{{\mathbb Z}}
\newcommand{\W}{{\mathcal W}}
\newcommand{\Y}{{\mathcal Y}}
\newcommand{\T}{{\mathcal T}}
\newcommand{\BP}{{\mathbb P}}
\newcommand{\CP}{{\mathcal P}}
\newcommand{\G}{{\mathbb G}}
\newcommand{\BR}{{\mathbb R}}
\newcommand{\D}{{\mathcal D}}
\newcommand {\LL}{{\mathcal L}}
\newcommand{\f}{{\mathcal F}}
\newcommand{\E}{{\mathcal E}}
\newcommand{\BN}{{\mathbb N}}
\newcommand{\N}{{\mathcal N}}
\newcommand{\K}{{\mathcal K}}
\newcommand{\R} {{\mathbb R}}
\newcommand{\PP} {{\mathbb P}}
\newcommand{\BF}{{\mathbb F}}
\newcommand{\closure}[1]{\overline{#1}}
\newcommand{\EQ}{\Leftrightarrow}
\newcommand{\imply}{\Rightarrow}
\newcommand{\isom}{\cong}
\newcommand{\embed}{\hookrightarrow}
\newcommand{\tensor}{\mathop{\otimes}}
\newcommand{\wt}[1]{{\widetilde{#1}}}
\newcommand{\ol}{\overline}
\newcommand{\ul}{\underline}

\newcommand{\bs}{{\backslash}}
\newcommand{\CS}{{\mathcal S}}
\newcommand{\CA}{{\mathcal A}}
\newcommand{\X}{{\mathcal X}}
\newcommand{\Q} {{\mathbb Q}}
\newcommand{\F} {{\mathcal F}}
\newcommand{\sing}{{\text{sing}}}
\newcommand{\U} {{\mathcal U}}
\newcommand{\B}{{\mathcal B}}

\title{Bloch's Theorem}

\maketitle

%\author{Xi Chen}

%\address{Department of Mathematics\\
%South Hall, Room 6607\\
%University of California\\
%Santa Barbara, CA 93106}
%\email{xichen@math.ucsb.edu}
%
%\author{}
%
%\address{Department of Mathematics\\
%South Hall, Room 6607\\
%University of California\\
%Santa Barbara, CA 93106}
%\email{}
%
\date{\today}

\begin{lem}\label{LEM000900}
    Let $f$ be analytic in $\Delta = \{|z| < 1\}$ with $f(0) = 0$ and $f'(0) = 1$. If $|f(z)|\le M$ for all $z\in \Delta$, then
    $f(\Delta)$ contains the disk $|w| \le (\sqrt{M+1} - \sqrt{M})^2$.
\end{lem}

\begin{proof}
    By Schwartz's lemma, we implicitly have $M \ge 1$. Let $f(z) = z + \sum_{n=2}^\infty a_n z^n$.
    Using CIF, we have
    $|a_n| \le M$ for all $n$. Therefore,
    \begin{equation}\label{E000900}
        \begin{split}
            |f(z)| & \ge |z| - \sum_{n=2}^\infty |a_n| |z|^n
            \\
                   & = r - \frac{M r^2}{1-r}
        \end{split}
    \end{equation}
    for $|z| = r < 1$. Obviously, we can maximize the RHS of \eqref{E000900} by taking
    \begin{equation}\label{E000100}
        r = \rho = 1 - \sqrt{\frac{M}{M+1}}
    \end{equation}
    and correspondingly, $|f(z)| \ge (\sqrt{M+1} - \sqrt{M})^2$ for $|z| = \rho$.

    For all $|w| < (\sqrt{M+1} - \sqrt{M})^2$, $|f(z) - (f(z) - w)| = |w| \le |f(z)|$ for all $|z| = \rho$. Therefore,
    $f(z) - w$ and $f(z)$ have the same number of zeros in $|z| < \rho$. It follows that
    $f(\Delta)$ contains the disk $|w| \le (\sqrt{M+1} - \sqrt{M})^2$.
\end{proof}

Obviously, by ``scaling'', we have the following:

\begin{lem}\label{LEM000102}
    Let $f$ be an analytic function on $D = \{|z-a|<R\}$. If $|f(z) - f(a)| \le M$ for all $z\in D$, then
    $f(D)$ contains the disk $|w - f(a)| \le (\sqrt{M + |f'(a) R|} - \sqrt{M})^2$.
\end{lem}

\begin{lem}\label{LEM000101}
    An analytic function $f(z)$ on $\Delta$ is 1-1 if $|f'(z) - M| < |M|$ for all
    $z\in \Delta$ and a constant $M\in \BC$.
\end{lem}

\begin{proof}
    Let $z_1$ and $z_2$ be two distinct points in $\Delta$ and let $\gamma$ be the line joining
    $z_1$ and $z_2$. Then
    \begin{equation}\label{E000102}
        \begin{split}
            |f(z_1) - f(z_2)| & = \left|\int_\gamma f'(z) dz\right|                                         \\
                              & = \left| \int_\gamma M - (M - f'(z)) dz\right|                              \\
                              & \ge \left|\int_\gamma M dz\right| - \left|\int_\gamma (f'(z) - M) dz\right| \\
                              & \ge |M| \int_\gamma |dz| - \int_\gamma |f'(z) - M| |dz| > 0
        \end{split}
    \end{equation}
    and hence $f(z)$ is 1-1. Note: the third line is triangle inequality.
\end{proof}

\begin{thm}[Bloch's Theorem]\label{THM000900}
    Let $f(z)$ be an analytic function on $\Delta$ satisfying $f'(0) = 1$. Then there is a positive
    constant $B$ (called {\it Bloch's constant\/}), independent of $f$, such that
    there exists a disk $S \subset \Delta$ where $f$ is 1-1 and whose image $f(S)$ contains a disk
    of radius $B$. In particular, $B > 1/72$.
\end{thm}

\begin{proof}
    Obviously, it is enough to show this for $f'(z)$ bounded on $\Delta$.

    Let
    \begin{equation}\label{E000103}
        m(r, g) = \max_{|z| = r} |g(z)|.
    \end{equation}
    We let $0\le r_0 < 1$ be the largest number \footnote[1]{Here Xi Chen defined using $\max$, but I think $\sup$ would be more accurate.} such that $(1-r_0) m(r_0, f') = 1$.
    Such $r_0$ exists since $f'(z)$ is bounded on $\Delta$ and $m(r,g)$ is continuous. Indeed,
    assume that $r_n \to r$ as $n \to \infty$ and denote $k_n=r_n/r$. We could assume that
    $m(r_n,g) =|g(z_n)|$ for every nonnegative integer $n$. Then clearly we have
    \[|g(k_nz)| \le g(z_n) \le g(z)= m(r,g)\]
    Since $g$ is continuous, $g(k_nz) \to g(z)$, thus $g(z_n)$ also tends to $g(z)$. This implies
    $m(r,g)$ is continuous.

    Then
    $(1-r) m(r, f') < 1$ for $r > r_0$ and hence
    \begin{equation}\label{E000104}
        |f'(z)| \le \frac{1}{1 - |z|}
    \end{equation}
    for $|z| \ge r_0$. And by principle of maximum modulus, we have
    \begin{equation}\label{E000105}
        |f'(z)| \le m(r_0, f') = \frac{1}{1 - r_0}
    \end{equation}
    for $|z| \le r_0$. In conclusion,
    \begin{equation}\label{E000106}
        |f'(z)| \le \frac{1}{1 - \max(r_0, |z|)}
    \end{equation}
    for all $z\in \Delta$.

    Let $a\in \Delta$ be a number such that $|a| = r_0$ and $|f'(a)| = 1/(1-r_0)$.

    For $0 < \rho < 1-r_0$ and $|z - a| \le \rho$, we have
    \begin{equation}\label{E000107}
        |f'(z) - f'(a)| \le \frac{1}{1 - r_0} + \frac{1}{1 - r_0 - \rho}
    \end{equation}
    and hence
    \begin{equation}\label{E000108}
        |f'(z) - f'(a)| \le \frac{|z - a|}{\rho}
        \left(\frac{1}{1 - r_0} + \frac{1}{1 - r_0 - \rho}\right)
    \end{equation}
    by Schwartz's lemma \footnote[2]{See the end of the note.}. Therefore,
    $|f'(z) - f'(a)| < |f'(a)|$ for $z$ in the disk
    \begin{equation}\label{E000110}
        S = \left\{|z - a| < \frac{\rho(1-r_0-\rho)}{2(1-r_0) - \rho}\right\}.
    \end{equation}
    The radius of $S$ is founded by solving for $f'(a)>$ RHS \ref{E000108}.

    By \lemref{LEM000101}, $f$ is 1-1 on $S$. Obviously, the radius of $S$ is maximized when we set
    $\rho = (2-\sqrt{2})(1-r_0)$ and correspondingly,
    \begin{equation}\label{E000111}
        S = \left\{|z-a| < (3-2\sqrt{2})(1 - r_0)\right\}.
    \end{equation}

    Moreover, since
    \begin{equation}\label{E000109}
        |f(z) - f(a)| \le \ln\left(\frac{\sqrt{2}+1}{2}\right)
    \end{equation}
    for $z\in S$ by \eqref{E000106}, we conclude that $f(S)$ contains a disk of radius
    \begin{equation}\label{E000112}
        \left(\sqrt{\ln\left(\frac{\sqrt{2}+1}{2}\right) + (3 - 2\sqrt{2})}
        - \sqrt{\ln\left(\frac{\sqrt{2}+1}{2}\right)}\right)^2 > \frac{1}{72}.
    \end{equation}
\end{proof}

\begin{rem}\label{REM000900}
    The key to the proof of Bloch's theorem is the existence of $a\in \Delta$ and positive constants $C_1$ and $C_2$ such that
    $|f'(z)| \le C_2 |f'(a)|$ for all $|z-a| \le C_1/|f'(a)|$.
\end{rem}

Supplementary notes: A variant of Schwarz's lemma
\begin{thm}[A variant of Schwarz's lemma]

    Let $f \colon \left\lbrace |z| \le R \right\rbrace  \to \mathbb{C}$ such that $|f(z)| \le A$ for all $z$ and $f(0)=0$.
    Then
    \[f(z) \le \dfrac{A|z|}{R}\]
\end{thm}
\begin{proof}
    Let define the function $g(y):=\dfrac{f(Ay)}{R}$. Then clearly $g \colon \Delta \to \Delta$ and
    satisfying $g(0)=0$. By the usual Schwarz's lemma, we must have $|g(y)| \le |y|$. Change $y \to \frac{z}{R}$, we get the desired inequality.
\end{proof}
\end{document}