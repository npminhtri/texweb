\documentclass[11pt]{amsart}

\usepackage[toc,page]{appendix}

\usepackage{etex}
\usepackage{amsmath}
\usepackage{amssymb}
\usepackage{amscd}
\usepackage{amsthm}
\usepackage{hyperref}
\usepackage{colortbl}

\usepackage{mathtools}
\usepackage{extarrows}
\usepackage{ifthen}

\usepackage[all,cmtip]{xy}

\usepackage{tikz}
\usepackage{pgfplots}
%\pgfplotsset{compat=1.9}
\usepackage{tikz-cd}

%graphics
%\usepackage{graphicx}
%\usepackage{pstricks}
%\usepackage{pst-plot}

%\setlength{\textwidth}{4.7in}
%\setlength{\textheight}{7.5in}

\newtheorem{thm}{Theorem}[section]
\newtheorem{lem}[thm]{Lemma}
\newtheorem{cor}[thm]{Corollary}
\newtheorem{conj}[thm]{Conjecture}
\newtheorem{prop}[thm]{Proposition}
\newtheorem{question}[thm]{Question}
\newtheorem{claim}[thm]{Claim}
\newtheorem{fact}[thm]{Fact}
\newtheorem{remark}[thm]{Remark}

\theoremstyle{remark}

\newtheorem{rem}[thm]{Remark}
\newtheorem{example}[thm]{Example}
\newtheorem{prob}{Problem}[section]

\theoremstyle{definition}
\newtheorem{defn}[thm]{Definition}

\numberwithin{equation}{section}

\allowdisplaybreaks[4]

%\usepackage[usenames,dvipsnames]{color}

\begin{document}

\vfuzz0.5pc
\hfuzz0.5pc % Don't bother to report
% overfull boxes if overage is < 6pt

\newcommand{\comment}[1]{}

\ifthenelse{\equal{1}{1}}{
    \newcommand{\blue}[1]{{\color{blue}#1}}
    \newcommand{\green}[1]{{\color{green}#1}}
    \newcommand{\red}[1]{{\color{red}#1}}
    \newcommand{\cyan}[1]{{\color{cyan}#1}}
    \newcommand{\magenta}[1]{{\color{magenta}#1}}
    \newcommand{\yellow}[1]{{\color{yellow}#1}} % not easy to read!
}{
    \newcommand{\blue}[1]{}
    \newcommand{\green}[1]{}
    \newcommand{\red}[1]{}
    \newcommand{\cyan}[1]{}
    \newcommand{\magenta}[1]{}
    \newcommand{\yellow}[1]{} % not easy to read!
}

\newcommand{\claimref}[1]{Claim \ref{#1}}
\newcommand{\thmref}[1]{Theorem \ref{#1}}
\newcommand{\propref}[1]{Proposition \ref{#1}}
\newcommand{\lemref}[1]{Lemma \ref{#1}}
\newcommand{\coref}[1]{Corollary \ref{#1}}
\newcommand{\remref}[1]{Remark \ref{#1}}
\newcommand{\conjref}[1]{Conjecture \ref{#1}}
\newcommand{\questionref}[1]{Question \ref{#1}}
\newcommand{\defnref}[1]{Definition \ref{#1}}
\newcommand{\secref}[1]{\S \ref{#1}}
\newcommand{\ssecref}[1]{\ref{#1}}
\newcommand{\sssecref}[1]{\ref{#1}}

%\def \d#1{\displaystyle{#1}}
%\def \mult{\mathop{\mathrm{mult}}\nolimits}
%\def \br{\mathop{\mathrm{br}}\nolimits}
%\def \rank{\mathop{\mathrm{rank}}\nolimits}

%\def \codim{\mathop{\mathrm{codim}}\nolimits}
%\def \Ord{\mathop{\mathrm{Ord}}\nolimits}
%\def \Var{\mathop{\mathrm{Var}}\nolimits}
%\def \Ext{\mathop{\mathrm{Ext}}\nolimits}
%\def \Pic{\mathop{\mathrm{Pic}}\nolimits}
%\def \Spec{\mathop{\mathrm{Spec}}\nolimits}
%\def \Jac{{\mathrm{Jac}}}

%\def \Div{{\mathrm{Div}}}
%\def \sgn{{\mathrm{sgn}}}
%\def \Hom{{\mathrm{Hom}}}
%\def \Sym{{\mathrm{Sym}}}
%\def \nilrad{{\mathrm{nilrad}}}
%\def \Supp{{\mathrm{Supp}}}
%\def \Ann{{\mathrm{Ann}}}

\newcommand{\RED}{{\mathrm{red}}}
\newcommand{\tors}{{\mathrm{tors}}}
\newcommand{\eq}{\Leftrightarrow}

\newcommand{\mapright}[1]{\smash{\mathop{\longrightarrow}\limits^{#1}}}
\newcommand{\mapleft}[1]{\smash{\mathop{\longleftarrow}\limits^{#1}}}
\newcommand{\mapdown}[1]{\Big\downarrow\rlap{$\vcenter{\hbox{$\scriptstyle#1$}}$}}
\newcommand{\smapdown}[1]{\downarrow\rlap{$\vcenter{\hbox{$\scriptstyle#1$}}$}}

\newcommand{\A}{{\mathbb A}}
\newcommand{\I}{{\mathcal I}}
\newcommand{\J}{{\mathcal J}}
\newcommand{\CO}{{\mathcal O}}
\newcommand{\C}{{\mathcal C}}
\newcommand{\BC}{{\mathbb C}}
\newcommand{\BQ}{{\mathbb Q}}
\newcommand{\m}{{\mathcal M}}
\newcommand{\h}{{\mathcal H}}
\newcommand{\Z}{{\mathcal Z}}
\newcommand{\BZ}{{\mathbb Z}}
\newcommand{\W}{{\mathcal W}}
\newcommand{\Y}{{\mathcal Y}}
\newcommand{\T}{{\mathcal T}}
\newcommand{\BP}{{\mathbb P}}
\newcommand{\CP}{{\mathcal P}}
\newcommand{\G}{{\mathbb G}}
\newcommand{\BR}{{\mathbb R}}
\newcommand{\D}{{\mathcal D}}
\newcommand{\DD}{{\mathcal D}}
\newcommand{\LL}{{\mathcal L}}
\newcommand{\f}{{\mathcal F}}
\newcommand{\E}{{\mathcal E}}
\newcommand{\BN}{{\mathbb N}}
\newcommand{\N}{{\mathcal N}}
\newcommand{\K}{{\mathcal K}}
\newcommand{\R} {{\mathbb R}}
\newcommand{\PP}{{\mathbb P}}
\newcommand{\Pp}{{\mathbb P}}
\newcommand{\BF}{{\mathbb F}}
\newcommand{\QQ}{{\mathcal Q}}
\newcommand{\VV}{{\mathcal V}}
\newcommand{\closure}[1]{\overline{#1}}
\newcommand{\EQ}{\Leftrightarrow}
\newcommand{\imply}{\Rightarrow}
\newcommand{\isom}{\cong}
\newcommand{\embed}{\hookrightarrow}
\newcommand{\tensor}{\mathop{\otimes}}
\newcommand{\wt}[1]{{\widetilde{#1}}}
\newcommand{\ol}{\overline}
\newcommand{\ul}{\underline}

\newcommand{\bs}{{\backslash}}
\newcommand{\CS}{{\mathcal S}}
\newcommand{\CA}{{\mathcal A}}
\newcommand{\Q}{{\mathbb Q}}
\newcommand{\F}{{\mathcal F}}
\newcommand{\sing}{{\text{sing}}}
\newcommand{\U} {{\mathcal U}}
\newcommand{\B}{{\mathcal B}}
\newcommand{\X}{{\mathcal X}}
\newcommand{\g}{{\mathcal G}}

% Javier-Macro
\newcommand{\ECS}[1]{E_{#1}(X)}
\newcommand{\CV}[2]{{\mathcal C}_{#1,#2}(X)}

\newcommand{\rank}{\mathop{\mathrm{rank}}\nolimits}
\newcommand{\codim}{\mathop{\mathrm{codim}}\nolimits}
\newcommand{\Ord}{\mathop{\mathrm{Ord}}\nolimits}
\newcommand{\Var}{\mathop{\mathrm{Var}}\nolimits}
\newcommand{\Ext}{\mathop{\mathrm{Ext}}\nolimits}
\newcommand{\EXT}{\mathop{{\mathcal E}\mathrm{xt}}\nolimits}
\newcommand{\Pic}{\mathop{\mathrm{Pic}}\nolimits}
\newcommand{\Spec}{\mathop{\mathrm{Spec}}\nolimits}
\newcommand{\Jac}{\mathop{\mathrm{Jac}}\nolimits}
\newcommand{\Div}{\mathop{\mathrm{Div}}\nolimits}
\newcommand{\sgn}{\mathop{\mathrm{sgn}}\nolimits}
\newcommand{\supp}{\mathop{\mathrm{supp}}\nolimits}
\newcommand{\Hom}{\mathop{\mathrm{Hom}}\nolimits}
\newcommand{\Sym}{\mathop{\mathrm{Sym}}\nolimits}
\newcommand{\nilrad}{\mathop{\mathrm{nilrad}}\nolimits}
\newcommand{\Ann}{\mathop{\mathrm{Ann}}\nolimits}
\newcommand{\Proj}{\mathop{\mathrm{Proj}}\nolimits}
\newcommand{\mult}{\mathop{\mathrm{mult}}\nolimits}
\newcommand{\Bs}{\mathop{\mathrm{Bs}}\nolimits}
\newcommand{\Span}{\mathop{\mathrm{Span}}\nolimits}
\newcommand{\IM}{\mathop{\mathrm{Im}}\nolimits}
\newcommand{\im}{\mathop{\mathrm{im}}\nolimits}
\newcommand{\Hol}{\mathop{\mathrm{Hol}}\nolimits}
\newcommand{\End}{\mathop{\mathrm{End}}\nolimits}
\newcommand{\CH}{\mathop{\mathrm{CH}}\nolimits}
\newcommand{\Exec}{\mathop{\mathrm{Exec}}\nolimits}
\newcommand{\SPAN}{\mathop{\mathrm{span}}\nolimits}
\newcommand{\birat}{\mathop{\mathrm{birat}}\nolimits}
\newcommand{\cl}{\mathop{\mathrm{cl}}\nolimits}
\newcommand{\rat}{\mathop{\mathrm{rat}}\nolimits}
\newcommand{\Bir}{\mathop{\mathrm{Bir}}\nolimits}
\newcommand{\Rat}{\mathop{\mathrm{Rat}}\nolimits}
\newcommand{\aut}{\mathop{\mathrm{aut}}\nolimits}
\newcommand{\Aut}{\mathop{\mathrm{Aut}}\nolimits}
\newcommand{\eff}{\mathop{\mathrm{eff}}\nolimits}
\newcommand{\nef}{\mathop{\mathrm{nef}}\nolimits}
\newcommand{\amp}{\mathop{\mathrm{amp}}\nolimits}
\newcommand{\DIV}{\mathop{\mathrm{Div}}\nolimits}
\newcommand{\Bl}{\mathop{\mathrm{Bl}}\nolimits}
\newcommand{\Cox}{\mathop{\mathrm{Cox}}\nolimits}
\newcommand{\NE}{\mathop{\mathrm{NE}}\nolimits}
\newcommand{\NM}{\mathop{\mathrm{NM}}\nolimits}
\newcommand{\Gal}{\mathop{\mathrm{Gal}}\nolimits}
\newcommand{\coker}{\mathop{\mathrm{coker}}\nolimits}
\newcommand{\ch}{\mathop{\mathrm{ch}}\nolimits}
\newcommand{\Res}{\mathop{\mathrm{Res}}\nolimits}
\newcommand{\Log}{\mathop{\mathrm{Log}}\nolimits}

\title{Math 506 Homework 1-3 Solutions}

\maketitle

\section{Homework 1}

\begin{prob}
    Use the definition of $H_1$ to prove the following: Let $D_1$ and $D_2$ be two connected open sets in $\BC$. If $H_1(D_1) = H_1(D_2) = 0$ and
    $D_1\cap D_2$ is connected, then $H_1(D_1\cup D_2) = 0$.
    Hint: Show that every closed curve $\gamma$ in $D_1\cup D_2$ is homologous to the sum
    $\sum \gamma_\alpha$, where each $\gamma_\alpha$ is either a closed curve in $D_1$ or a closed curve in $D_2$.
\end{prob}

\begin{proof}
    Let $\gamma$ be parameterized by $\gamma: [0,1]\to D = D_1\cup D_2$.
    Let us consider $\gamma^{-1}(D_1)$ and $\gamma^{-1}(D_2)$. Each is a disjoint union of intervals open in $[0,1]$ and their union covers $[0,1]$. By compactness of $[0,1]$, we can find a finite open cover of $[0,1]$ in the form of
    \[
        [0,1] = [a_0, b_0) \cup (a_1, b_1) \cup ... \cup (a_{2n}, b_{2n}]
    \]
    with $0 = a_0 < a_1 < b_0 < a_2 < b_1 < a_3 < b_2 < ... < b_{2n}=1$ such that $(a_k, b_k)\subset \gamma^{-1}(D_1)$ if $k$ is even and $(a_k,b_k) \subset \gamma^{-1}(D_2)$ if $k$ is odd. Here we assume that $\gamma(0) = \gamma(1)\in D_1$ WLOG. So
    \[
        [0,1] = [0,c_1] \cup [c_1, c_2]\cup ... \cup [c_{2n}, 1]
        \text{ for } c_m = \frac{1}2(b_{m-1} + a_m)
    \]
    with $\gamma(c_m)\in D_1\cap D_2$ for $1\le m \le 2n$ and $\gamma([c_{m-1}, c_m])\subset D_k$ if $2\mid (m-k)$, where we let $c_0=0$ and $c_{2n+1} = 1$.

    We let $\gamma_m$ be the curve $\gamma: [c_{m-1}, c_m]\to D$. Then $\gamma_m\subset D_k$ if $2\mid (m-k)$. For each pair $\{c_i, c_{2n+1-i}\}$, we choose a continuous curve $\sigma_i: [0,1]\to D_1\cap D_2$ such that $\sigma_i(0) = \gamma(c_i)$ and $\sigma_i(1) = \gamma(c_{2n+1-i})$. This is possible since $D_1\cap D_2$ is connected. Then
    \[
        \gamma = (\gamma_1 + \sigma_1 + \gamma_{2n+1}) +
        \sum_{i=2}^{n}
        (\gamma_i + \sigma_{i} + \gamma_{2n+2-i} - \sigma_{i-1})
        + (\gamma_{n+1} - \sigma_n)
    \]
    in $H_1(D)$. Clearly, each of
    \[
        \gamma_1 + \sigma_1 + \gamma_{2n+1},\ \gamma_i + \sigma_{i} + \gamma_{2n+2-i} - \sigma_{i-1},\ \gamma_{n+1} - \sigma_n
    \]
    lies entirely in one of $D_1$ and $D_2$. So they are homologous to $0$ since $H_1(D_1) = H_1(D_2) = 0$. So $\gamma = 0$ in $H_1(D)$.
\end{proof}

\begin{prob}
    Find all entire functions $f(z)$ satisfying
    \[
        f(z_1 + z_2) = f(z_1) f(z_2) \text{ for all } z_1, z_2\in \BC.
    \]
    Do there exist nonconstant entire functions $f(z)$ satisfying
    \[
        f(z_1 z_2) = f(z_1) + f(z_2) \text{ for all } z_1, z_2\in \BC?
    \]
    Justify your answer.
\end{prob}

\begin{proof}
    If $f(z)$ has a zero at $z_0$, then $f(z) = f(z_0) f(z-z_0) = 0$ for all $z$.

    Suppose that $f(z)$ is nowhere vanishing. Then $f'(z) /f(z)$ has a complex anti-derivative $g(z)$ on $\BC$. Then
    \[
        \frac{(e^{g(z)})'}{e^{g(z)}} = g'(z) = \frac{f'(z)}{f(z)}
        \Rightarrow \left(
        \frac{f(z)}{e^{g(z)}}
        \right)' = 0
    \]
    for all $z\in \BC$. Therefore, $f(z) \equiv c e^{g(z)}$ for some constant $c\ne 0$. We may choose $g(z)$ such that $c = 1$. So $f(z) \equiv e^{g(z)}$ for some entire function $g(z)$. Then
    \[
        \begin{aligned}
             & \quad 1 = \frac{f(z_1) f(z_1)}{f(z_1 + z_2)} = \exp (g(z_1) + g(z_2) - g(z_1 + z_2)) \\
             & \Rightarrow g(z_1) + g(z_2) - g(z_1 + z_2) \in \{2n\pi i: n\in \BZ\}
        \end{aligned}
    \]
    for all $z_1,z_2\in \BC$.
    And since $g(z_1) + g(z_2) - g(z_1+z_2): \BC\times \BC\to \BC$ is continuous, we must have
    \[
        g(z_1) + g(z_2) - g(z_1 + z_2) = 2n\pi i
    \]
    for all $z_1, z_2\in \BC$ and some $n\in \BZ$. Differentiating it with respect to $z_1$, we obtain
    \[
        g'(z_1) = g'(z_1 + z_2)
    \]
    for all $z_1, z_2\in \BC$. Hence $g'(z) \equiv a$ and $g(z) \equiv az + 2n\pi i$. So $f(z) \equiv \exp(az)$.

    In conclusion, either $f(z)\equiv 0$ or $\exp(az)$ for some constant $a\in \BC$.
\end{proof}

\begin{prob}
    Show that if $f$ and $g$ are analytic functions on a region $G$ (i.e. a connected open set in $\BC$) such that $\overline{f} g$ is analytic on $G$, then either $f$ is constant or $g\equiv 0$.
\end{prob}

\begin{proof}
    If $g\equiv 0$, we are done. Otherwise, $D = \{g(z)\ne 0\}$ is a dense open subset of $G$. Then $f$ and $\overline{f} = (\overline{f} g)/g$ are analytic on $D$. By Cauchy-Riemann equations,
    \[
        \frac{\partial f}{\partial \overline{z}}
        = \frac{\partial \overline{f}}{\partial \overline{z}} = 0
        \Rightarrow \frac{\partial f}{\partial \overline{z}}
        = \overline{\frac{\partial f}{\partial z}} = 0
        \Rightarrow \frac{\partial f}{\partial \overline{z}}
        = \frac{\partial f}{\partial z} = 0
        \Rightarrow \frac{\partial f}{\partial x}
        = \frac{\partial f}{\partial y} = 0
    \]
    on $D$ and hence $f(z)\equiv c$ is constant on $D$. By continuity, $f(z)\equiv c$ on $G$ since $D$ is dense in $G$.
\end{proof}

\begin{prob}
    Let $D \subset \BC$ be a region and let $f(z)$
    be a meromorphic functions on $D$ (i.e. the quotient of two analytic functions on $D$). Show that if
    \[
        a_0(z) + a_1(z) f(z) + a_2(z) (f(z))^2 + ... + a_{n-1}(z) (f(z))^{n-1} + (f(z))^n \equiv 0
    \]
    for some analytic functions $a_0(z), a_1(z), ..., a_{n-1}(z)$ on $D$,
    then $f(z)$ is analytic on $D$.
\end{prob}

\begin{proof}
    Otherwise, $f(z)$ has a pole at $z_0\in D$. Then
    $f(z) = (z-z_0)^{-m} g(z)$ in a disk $\Delta = \{|z-z_0| < r\}$ for some $m\in \BZ^+$ and some analytic function $g(z)$ in $\Delta$ satisfying $g(z_0)\ne 0$. Then
    \[
        \sum_{r=0}^{n-1} a_r(z) (z-z_0)^{m(n-r)} (g(z))^{r} + (g(z))^n = 0
    \]
    in $\Delta$. Setting $z = z_0$, we obtain $g(z_0) = 0$. Contradiction. So $f(z)$ is analytic on $D$.
\end{proof}

\begin{prob}
    Suppose that the power series $\sum a_n z^n$ has radius of convergence $1$.
    If $\sum a_n$ converges to $A$, show that
    \[
        \lim_{r\to 1^-} \sum a_n r^n = A.
    \]
    Use this to show that
    \[
        \log 2 = \sum_{n=1}^\infty \frac{(-1)^{n+1}}{n}.
    \]
\end{prob}

\begin{proof}
    Let
    \[
        A_n = \sum_{m=0}^n a_m.
    \]
    Then
    \[
        \sum_{n=0}^\infty a_n z^n = \sum_{n=0}^\infty (A_n - A_{n-1}) z^n
        = \sum_{n=0}^\infty A_n z^n(1-z)
    \]
    for $|z| < 1$. For $0 < r < 1$,
    \[
        \begin{aligned}
            \left|
            \sum a_n r^n - A
            \right| & = \left|\sum_{n=0}^\infty A_n r^n(1-r) - \sum_{n=0}^\infty A r^n(1-r)\right| \\
                    & \le \sum_{n=0}^\infty |A_n - A|r^n(1-r)
            \\
                    & = \sum_{n=0}^{N-1} |A_n - A|r^n(1-r) + \sum_{n=N}^\infty |A_n - A|r^n(1-r)   \\
                    & \le 2M(1-r^N) + \varepsilon_N r^N
        \end{aligned}
    \]
    for all $N\in \BZ^+$, $\varepsilon_N = \sup\{|A_n-A|: n\ge N\}$
    and $M = \sup |A_n|$. Therefore,
    \[
        \limsup_{r\to 1^-} \left|
        \sum a_n r^n - A
        \right| \le \varepsilon_N
    \]
    for all $N\in \BZ^+$. And since $\varepsilon_N\to 0$ as $N\to\infty$,
    \[
        \limsup_{r\to 1^-} \left|
        \sum a_n r^n - A
        \right| = 0 \Rightarrow \lim_{r\to 1^-} \sum a_n r^n = A.
    \]

    Let $\Log(z)$ be the principal branch of $\log z$. Then
    \[
        \Log (1+z) = \sum_{n=1}^\infty \frac{(-1)^{n+1}}{n} z^n
    \]
    for $|z| < 1$. Since $\{1/n\}$ is decreasing and $\lim_{n\to\infty} 1/n = 0$,
    \[
        \sum_{n=1}^\infty \frac{(-1)^{n+1}}{n}
    \]
    converges. Therefore,
    \[
        \sum_{n=1}^\infty \frac{(-1)^{n+1}}{n} = \lim_{r\to 1^-} \Log(1+r) = \ln 2.
    \]
\end{proof}

\begin{prob}
    Show that if $f: \BC\to \BC$ is a continuous function and $f(z)$ is analytic
    on $\BC\backslash \{\text{Re}(z) = 0\}$, then $f(z)$ is entire.
\end{prob}

\begin{proof}
    By Morera's Theorem, it suffices to show that $\int_{\gamma} f(z) dz = 0$ for all triangles $\gamma$.

    Since $f(z)$ is analytic on $\{\text{Re}(z) > 0\}$, $\int_{\gamma} f(z) dz = 0$
    for all continuous closed curves $\gamma$ contained in $\{\text{Re}(z) > 0\}$. For every continuous closed curve $\gamma\in \{\text{Re}(z) \ge 0\}$,
    \[
        \begin{aligned}
            \int_{\gamma} f(z) dz & = \lim_{\varepsilon\to 0^+} \int_{\gamma_{\varepsilon}} f(z) dz = 0
        \end{aligned}
    \]
    where $\gamma_{\varepsilon}(t) = \gamma(t) + \varepsilon\in \{\text{Re}(z) > 0\}$ for $\varepsilon > 0$. In conclusion,
    \[
        \int_{\gamma} f(z)dz = 0
    \]
    for all continuous closed curves $\gamma\subset \{\text{Re}(z) \ge 0\}$. Similarly,
    \[
        \int_{\gamma} f(z)dz = 0
    \]
    for all continuous closed curves $\gamma\subset \{\text{Re}(z) \le 0\}$.

    For every triangle $\gamma$, it is easy to see that
    \[
        \int_{\gamma} f(z) dz = \int_{\gamma_1} f(z) dz + \int_{\gamma_2} f(z) dz
    \]
    for some closed polygons $\gamma_1$ and $\gamma_2$ satisfying that
    $\gamma_1\subset \{\text{Re}(z) \le 0\}$ and $\gamma_2\subset \{\text{Re}(z) \ge 0\}$. Therefore, $\int_{\gamma} f(z)dz = 0$ and $f(z)$ is entire.
\end{proof}

\begin{prob}
    Let $f(z)$ and $g(z)$ be two analytic functions on an open set $D$. Show that if $f(z)$ and $g(z)$ have finitely many zeros in $D$ and they do not have common zeros, then there exist analytic functions $a(z)$ and $b(z)$ on $D$ such that $a(z) f(z) + b(z) g(z) \equiv 1$ on $D$.
\end{prob}

\begin{proof}
    Let $p_1,p_2,...,p_n$ be the zeros of $f(z)$ with multiplicities $m_1,m_2,...,m_n$, respectively. We claim that there exists a polynomial $b(z)$ in $z$ of degree $\deg b(z) < m_1 + m_2 + ... + m_n$ such that $1-b(z)g(z)$ has zeros at $p_1,p_2,...,p_n$ of multiplicities at least $m_1,m_2,...,m_n$.

    Let $h(z) = 1/g(z)$. Since $g(z)$ does not vanish at $p_j$,
    $h(z)$ is analytic at $p_j$ for $j=1,2,...,n$. By Chinese Remainder Theorem,
    there exists $b(z)\in \BC[z]$ of $\deg b(z) < m_1 + m_2 + ... + m_n$ such that
    \[
        b(z) \equiv \sum_{l=0}^{m_j-1} \frac{h^{(l)}(p_j)}{l!} (z-p_j)^l
        \ \ (\text{mod}\ (z-p_j)^{m_j})
    \]
    for $j=1,2,...,n$. Therefore, $h(z) - b(z)$ has zeros at $p_j$ of multiplicities at least $m_j$. The same holds for
    $1 - b(z)g(z) = g(z)(h(z) - b(z))$. So
    \[
        a(z) = \frac{1-b(z)g(z)}{f(z)}
    \]
    is analytic on $D$. We are done.
\end{proof}

\section{Homework 2}

\begin{prob}
    Let $f(z)$ be an entire function with two periods $\lambda_1$ and $\lambda_2$, i.e.,
    \[
        f(z) = f(z+\lambda_1) = f(z+\lambda_2)
    \]
    for all $z\in \BC$. If $\lambda_1$ and $\lambda_2$ are linearly independent over $\BQ$, then $f(z)$ must be constant.
\end{prob}

\begin{proof}
    Suppose that $\lambda_1$ and $\lambda_2$ are linearly independent over $\BR$. Then every complex number $z$ is a linear combination of $\lambda_1$ and $\lambda_2$ over $\BR$. That is,
    \[
        z = c_1 \lambda_1 + c_2\lambda_2
    \]
    for some real numbers $c_1$ and $c_2$. Let
    \[
        m_1 = \lfloor c_1\rfloor \text{ and } m_2 = \lfloor c_2\rfloor
    \]
    be the largest integers less than or equal to $c_1$ and $c_2$, respectively. Then
    \[
        f(z) = f(z - m_1 \lambda_1 - m_2\lambda_2)
        = f((c_1-m_1) \lambda_1 + (c_2 - m_2)\lambda_2).
    \]
    Since $0 < c_1 - m_1 \le 1$ and $0 < c_2 - m_2 \le 1$,
    \[
        |(c_1-m_1) \lambda_1 + (c_2 - m_2)\lambda_2|
        \le (c_1 - m_1) |\lambda_1| + (c_2 - m_2)|\lambda_2|
        \le |\lambda_1| + |\lambda_2|.
    \]
    Let $M$ be the maximum of $|f(z)|$ on $\{|z|\le |\lambda_1| + |\lambda_2|\}$. Then
    \[
        |f(z)|
        = |f((c_1-m_1) \lambda_1 + (c_2 - m_2)\lambda_2)| \le M
    \]
    for all $z\in \BC$. So $f(z)$ is constant by Louville.

    Suppose that $\lambda_1$ and $\lambda_2$ are linearly independent over $\BQ$. If they are linearly independent over $\BR$, then we are done. Otherwise, $\lambda_1$ and $\lambda_2$ are linearly independent over $\BQ$ and dependent over $\BR$. That is, $\lambda = \lambda_1/\lambda_2\in \BR\backslash \BQ$. Namely, it is an irrational real number.

    We claim that for every $\varepsilon > 0$, there exist integers $m_1$ and $m_2$ such that
    \[
        0 < |m_1\lambda - m_2| < \varepsilon
    \]
    Fixing a positive integer $n$, let us consider
    \[
        a_k = k \lambda - \lfloor k\lambda \rfloor
    \]
    for $k=0,1,2,...,n$. These are $n+1$ numbers in the interval $[0,1]$. By Pigeon Hole principle, there exist $a_k$ and $a_l$ such that $0\le k\ne l\le n$ and
    \[
        |a_k - a_l| = |(k-l) \lambda -(\lfloor k\lambda \rfloor
        - \lfloor l\lambda \rfloor)| \le \frac{1}n
    \]
    Let $m_1 = k-l$ and $m_2 = \lfloor k\lambda \rfloor
        - \lfloor l\lambda \rfloor$. Then
    \[
        |m_1 \lambda - m_2| \le \frac{1}n.
    \]
    This proves our claim.

    For every positive integer $n$, there exist integers $m_1$ and $m_2$ such that
    \[
        0 < |m_1\lambda - m_2| \le \frac{1}n
    \]
    So
    \[
        0 < |m_1 \lambda_1 - m_2 \lambda_2| = |\lambda_2(m_1\lambda - m_2)| \le \frac{|\lambda_2|}n.
    \]
    Let $z_n = m_1\lambda_1 - m_2\lambda_2$.
    Since $f(z_n) = f(m_1 \lambda_1 - m_2 \lambda_2) = f(0)$, we conclude that there exists a sequence $\{z_n\}$ such that
    \[
        0 < |z_n| \le \frac{|\lambda_2|}n \text{ and }
        f(z_n) = f(0).
    \]
    This means that the set $\{z: f(z) = f(0)\}$ has a cluster point at $0$. So $f(z) \equiv f(0)$.
\end{proof}

\begin{prob}
    Let $f(z)$ be an entire function. Show that $f(x)\in \BR$ for all $x\in \BR$ if and only if $f^{(n)}(0)\in \BR$ for all $n=0,1,2,...$.
\end{prob}

\begin{proof}
    If $f^{(n)}(0)$ is real, then
    \[
        f(z) = \sum_{n=0}^\infty \frac{f^{(n)}(0)}{n!} z^n
    \]
    for all $z\in \BR$ and hence $f(x)\in \BR$ for all $x\in \BR$.

    Suppose that $f(x)\in \BR$ for all $x\in \BR$. We can prove by induction that
    $f^{(n)}(z)\in \BR$ for $z\in \BR$. By Cauchy-Riemann equations,
    \[
        f'(z) = \frac{\partial f}{\partial x} = f_x(z).
    \]
    For $z\in \BR$, since $f(z)\in \BR$, $f_x(z)\in \BR$. Therefore, $f'(z)\in \BR$ for all $z\in \BR$. Then, inductively, we have $f''(z), ..., f^{(n)}(z), ...\in \BR$ for all $z\in \BR$.
\end{proof}

\begin{prob}\label{MATH506HW123PROB23}
    Let $f_1(z)$ and $f_2(z)$ be two analytic functions on $D = \{|z| < 1\}$. Suppose that $f_1(0) = f_2(0)$, $f_2$ is biholomorphic and
    $f_1(D) \subset f_2(D)$. Show that
    \[
        |f_1'(0)| \le |f_2'(0)|.
    \]
    Find a necessary and sufficient condition for the equality to hold.
\end{prob}

\begin{proof}
    Let us consider $g(z) = f_2^{-1} \circ f_1(z): D\to D$, which is well defined since $f_1(D)\subset f_2(D)$.

    Since $f_1(0) = f_2(0)$, $g(0) = 0$. Applying Schwartz Lemma to $g$, we obtain $|g'(0)| \le 1$ and hence
    \[
        |g'(0)| = \frac{|f_1'(0)|}{|f_2'(0)|} \le 1 \Rightarrow |f_1'(0)| \le |f_2'(0)|.
    \]
    By Schwartz Lemma, the equality holds if and only if $g(z) = cz$ for some $|c| = 1$, i.e.,
    $f_1(z) = f_2(cz)$ for all $z$.
\end{proof}

\begin{prob}
    Let $f(z)$ be a holomorphic function on $D=\{|z| < 1\}$.
    If $f(0) = 0$, show that the series
    \[
        \sum_{n=1}^\infty f(z^n)
    \]
    uniformly converges on every compact subset of $D$.
\end{prob}

\begin{proof}
    It suffices to show that the series converges uniformly on the closed disk $\{|z|\le r\}$ for all $r < 1$. Let
    \[
        f(z) = \sum_{n=1}^\infty a_n z^n
    \]
    on $D$. Since $\sum a_n z^n$ has radius of convergence at least $1$, for every $0 < R < 1$, there exists a constant $M$ such that $|a_n| \le M R^{-n}$ for all $n$. We choose
    some $r < R < 1$. Then
    \[
        \begin{aligned}
            |f(z^n)| & = \left|\sum_{m=1}^\infty  a_m z^{mn}\right| \le \sum_{m=1}^\infty |a_m| |z|^{mn} \\
                     & \le MR^{-n} r^{mn} = \frac{M r^n}{R - r^n} \le \frac{Mr^n}{R-r}
        \end{aligned}
    \]
    for all $|z|\le r$. Clearly,
    \[
        \sum_{n=1}^\infty \frac{Mr^n}{R-r} = \frac{M}{R-r} \sum_{n=1}^\infty r^n
    \]
    converges and hence $\sum f(z^n)$ converges uniformly on $\{|z|\le r\}$.
\end{proof}

\begin{prob}
    Show that for a complex polynomial $f(z)$ of degree $n$, the function
    $M(r)/r^n$ is nonincreasing for $r\in (0,\infty)$, where
    \[
        M(r) = \max_{|z| \le r} |f(z)|.
    \]
\end{prob}

\begin{proof}
    Let $g(z) = z^n f(z^{-1})$. Then by Maximum Modulus,
    \[
        \max_{|z| \le 1/r} |g(z)| = \max_{|z| = 1/r} |g(z)| = \frac{1}{r^n}
        \max_{|z| = r} |f(z)| = \frac{M(r)}{r^n}.
    \]
    Hence
    \[
        \max_{|z| \le 1/r_1} |g(z)| \ge \max_{|z| \le 1/r_2} |g(z)|
        \Rightarrow \frac{M(r_1)}{r_1^n}\ge \frac{M(r_2)}{r_2^n}
    \]
    for all $0 < r_1 < r_2$.
\end{proof}

\begin{prob}
    Let $D = \{r\le |z|\le R\}$ for some $0<r<R$. Show that there exists a positive constant $\varepsilon$, depending on $r$ and $R$, such that
    \[
        \Big|\Big| f(z) - \frac{1}{z} \Big|\Big|_D = \max_{z\in D} \left|f(z) - \frac{1}z\right| \ge \varepsilon
    \]
    for all entire functions $f(z)$.
\end{prob}

\begin{proof}
    By Maximum Modulus,
    \[
        \begin{aligned}
            \Big|\Big| f(z) - \frac{1}{z} \Big|\Big|_D & \ge \max_{|z| = r} \left|f(z) - \frac{1}z\right| = \frac{1}r \max_{|z| = r} |zf(z) - 1|
            \\
                                                       & \ge \frac{1}r |0f(0) - 1| = \frac{1}r.
        \end{aligned}
    \]
\end{proof}

\begin{prob}
    Let $a$ be a complex number satisfying $|a| > 5/2$. Show that the power series
    \[
        F(z) = \sum_{n=0}^\infty \frac{z^n}{a^{n^2}}
    \]
    defines an entire function which does not vanish on the boundary of the annulus
    \[
        |a^{2n-2}| < |z| < |a^{2n}|
    \]
    and has exactly one zero inside the annulus for $n=1,2,...$.
\end{prob}

\begin{proof}
    We apply Rouch\'e's theorem to $f(z)$ and $f_n(z) = -a^{-n^2} z^n$ in $|z| < |a^{2n}|$.

    For $|z| = |a^{2n}|$,
    \[
        \begin{aligned}
            \left|\frac{z^{m-1} a^{-(m-1)^2}}{z^{m} a^{-m^2}}\right| & = a^{2m-2n-1}\le |a|^{-3} \text{ if } m\le n-1 \text{ and} \\
            \left|\frac{z^{m+1} a^{-(m+1)^2}}{z^{m} a^{-m^2}}\right| & = a^{2n-2m-1}\le |a|^{-3} \text{ if } m\le n+1
        \end{aligned}
    \]
    Therefore,
    \begin{equation}\label{MATH41117FFRE007}
        \begin{aligned}
            |f(z) + f_n(z)|
             & = \left|
            \sum_{m=0}^{n-1} a^{-m^2} z^m + \sum_{m=n+1}^\infty
            a^{-m^2} z^m
            \right|
            \\
             & \le \sum_{m=0}^{n-1} |a^{-m^2} z^m| + \sum_{m=n+1}^\infty
            |a^{-m^2} z^m|                                                                \\
             & = |a^{-(n-1)^2} z^{n-1}| \sum_{m=0}^{n-1}
            \left|\frac{z^m a^{-m^2}}{z^{n-1} a^{-(n-1)^2}}\right|
            \\
             & \quad + |a^{-(n+1)^2} z^{n+1}| \sum_{m=n+1}^{\infty}
            \left|\frac{z^m a^{-m^2}}{z^{n+1} a^{-(n+1)^2}}\right|                        \\
             & < |a^{n^2-1}| \sum_{m=0}^\infty |a|^{-3m}
            +  |a^{n^2-1}| \sum_{m=0}^\infty |a|^{-3m}                                    \\
             & = \frac{2|a|^{n^2-1}}{1-|a|^{-3}} = |f_n(z)|\frac{2|a|^{-1}}{1 - |a|^{-3}}
            \\
             & = |f_n(z)| \frac{2}{|a| - |a|^{-2}}
            < |f_n(z)| \frac{2}{(5/2) - (5/2)^{-2}}                                       \\
             & = \frac{100}{117} |f_n(z)| < |f_n(z)|
        \end{aligned}
    \end{equation}
    for $|z| = |a^{2n}|$ and $|a| > 5/2$. In conclusion, we have
    \begin{equation}\label{MATH41117FFRE008}
        |f(z) + f_n(z)| < |f(z)| + |f_n(z)|
    \end{equation}
    for $|z| = |a^{2n}|$ and all $n=0,1,2,...$. By Rouch\'e's Theorem, $f(z)$ and $f_n(z)$ have the same number of zeros in $|z| < |a^{2n}|$, counted with multiplicity. Therefore, $f(z)$ has exactly $n$ zeros in $|z| < |a^{2n}|$, counted with multiplicity. This holds for all $n\in \BN$.

    Finally, since $f(z)$ has $n$ zeros in $|z| < |a^{2n}|$ and $n-1$ zeros in $|z| < |a^{2n-2}|$, it has exactly one zero in
    $|a^{2n-2}| \le |z| < |a^{2n}|$. By \eqref{MATH41117FFRE008}, $f(z)\ne 0$ for $|z| = |a^{2n}|$ and all $n\in \BN$. Therefore, $f(z)$ has exactly one zero in $|a^{2n-2}| < |z| < |a^{2n}|$.
\end{proof}

\begin{prob}
    For an entire function $f(z)$, we let
    \[
        M(r) = \max_{|z| \le r} |f(z)|.
    \]
    Let $f(z)$ be an entire function with
    \[
        \limsup_{r\to\infty} \frac{\log M(r)}{r} = l.
    \]
    Show that the infinite series
    \[
        F(z) = \sum_{n=0}^\infty f^{(n)}(z)
    \]
    converges if $l < 1$ and diverges if $l > 1$.
\end{prob}

\begin{proof}
    Suppose that $l < 1$. So there exists $\lambda < 1$ such that $|f(z)| \le c e^{\lambda |z|}$ for some constant $c > 0$ and all $z$. By Cauchy Integral Formula,
    \begin{equation}\label{HWSE009}
        |f^{(n)}(z)| = \left|\frac{n!}{2\pi i} \int_{|w| = R} \frac{f(w)}{(w-z)^{n+1}} dw\right|
        \le \frac{c (n!) R}{(R - |z|)^{n+1}} e^{\lambda R}
    \end{equation}
    for all $n\in \BN$.

    We fix $r > 0$ and want to show that
    \begin{equation}\label{HWSE011}
        \sum_{n=0}^\infty |f^{(n)}(z)| < \infty
    \end{equation}
    in $\{|z| \le r\}$. We choose $R = r + \lambda^{-1} n$. Then
    \begin{equation}\label{HWSE010}
        |f^{(n)}(z)|
        \le c e^{\lambda r}\left(1 + \frac{\lambda r}{n}\right)  \lambda^n e^{n}  \frac{n!}{n^n}
    \end{equation}
    for all $n\ge 1$ and $|z| \le r$ by \eqref{HWSE009}. So it suffices to show the convergence
    of the series
    \begin{equation}\label{HWSE000}
        \sum_{n=1}^\infty c e^{\lambda r}\left(1 + \frac{\lambda r}{n}\right)  \lambda^n e^{n}  \frac{n!}{n^n} = \sum_{n=1}^\infty a_n
    \end{equation}
    which follows from the ratio test:
    \begin{equation}\label{HWSE001}
        \lim_{n\to\infty}
        \frac{a_{n+1}}{a_n} = \lim_{n\to\infty} \lambda e \left(\frac{n^n}{(n+1)^n}\right) = \lambda < 1.
    \end{equation}
    When $l > 1$, if $F(z)$ converges for $z = 0$, then
    \begin{equation}\label{HWSE002}
        \lim_{n\to\infty} f^{(n)}(0) = 0 \Rightarrow |f^{(n)}(0)| \le c
    \end{equation}
    for a constant $c$ and all $n$. Then
    \begin{equation}\label{HWSE003}
        |f(z)| = \left|\sum_{n=0}^\infty \frac{f^{(n)}(0)}{n!} z^n\right|
        \le \sum_{n=0}^\infty \frac{c |z|^n}{n!} = c e^{|z|}
    \end{equation}
    which contradicts
    \begin{equation}\label{HWSE004}
        \limsup_{r\to\infty} \frac{\log M(r)}{r} = l > 1.
    \end{equation}
\end{proof}

\begin{prob}
    Let $f(z)$ be an entire function with $M(r)$ defined in the previous problem.
    Show that if there is a constant $0 < \alpha < 1$ such that
    \[
        \lim_{r\to\infty} \frac{M(\alpha r)}{M(r)} > 0,
    \]
    then $f(z)$ is a polynomial and the above limit is $\alpha^n$ with $n=\deg f$.
\end{prob}

\begin{proof}
    Since the limit
    \begin{equation}\label{HWSE005}
        \lim_{r\to\infty} \frac{M(\alpha r)}{M(r)} > 0,
    \end{equation}
    exists, there exists a constant $c > 0$ such that
    $M(\alpha r) \ge c M(r)$ for all $r \ge 1$. Therefore,
    \begin{equation}\label{HWSE013}
        M(\alpha^n r) \ge c^n M(r) \Rightarrow c^{-n} M(1) \ge M(\alpha^{-n}).
    \end{equation}
    By Cauchy Integral Formula, we have
    \begin{equation}\label{HWSE014}
        \begin{aligned}
            |f^{(m)}(0)| & = \left|\frac{m!}{2\pi i} \int_{|z| = \alpha^{-n}} \frac{f(z)}{z^{m+1}} dz\right|
            \le (m!) M(\alpha^{-n}) \alpha^{mn}
            \\
                         & \le (m!) M(1) \left(\frac{\alpha^m}{c}\right)^n
        \end{aligned}
    \end{equation}
    for all $m$ and $n$. Then for all $m$ satisfying $\alpha^m < c$, $f^{(m)}(0) = 0$ by taking
    $n\to\infty$ in \eqref{HWSE014}. Therefore, $f(z)$ is a polynomial.

    If $f(z)$ is a polynomial of degree $n$, then
    \begin{equation}\label{HWSE015}
        \lim_{z\to\infty} \left|
        \frac{f(z)}{z^n}
        \right|
        = c \Rightarrow \lim_{r\to\infty}
        \frac{M(r)}{r^n} = c \Rightarrow \lim_{r\to\infty}
        \frac{M(\alpha r)}{M(r)} = \alpha^n.
    \end{equation}
\end{proof}

\begin{prob}
    Let $f(z)$ be an analytic function on $\{|z| < 1\}$. If
    $f(0) = 0$ and $|f(z)| < 1$ for all $z\in D$, show that
    \[
        |f''(0)| \le 2 - 2 |f'(0)|^2.
    \]
    Hint: Apply Schwartz's Lemma to the function
    \[
        \frac{g(z) - g(0)}{1 - \overline{g(0)} g(z)}
    \]
    for $g(z) = z^{-1} f(z)$.
\end{prob}

\begin{proof}
    By Schwartz's Lemma, $|g(z)| < 1$ for $|z|<1$ and $g(z) = z^{-1}f(z)$ unless $f(z) = cz$ for $|c| = 1$, where the inequality is obvious.

    Therefore, $|h(z)| < 1$ for $|z|<1$ and
    \[
        h(z) = \frac{g(z) - g(0)}{1 - \overline{g(0)} g(z)}.
    \]
    And since $h(0) = 0$, we conclude that
    \[
        |h'(0)| = \frac{|g'(0)|}{1 - |g(0)|^2} \le 1\Rightarrow
        |g'(0)| \le 1 - |g(0)|^2
    \]
    by Schwartz's Lemma.

    Suppose that
    \[
        f(z) = \sum_{n=1}^\infty a_n z^n.
    \]
    Then
    \[
        g(z) = \sum_{n=0} a_{n+1} z^n
    \]
    and hence
    \[
        g^{(n)}(0) = (n!) a_{n+1} = \frac{(n!) f^{(n+1)}(0)}{(n+1)!} = \frac{f^{(n+1)}(0)}{n+1}.
    \]
    Therefore,
    \[
        |g'(0)| \le 1 - |g(0)|^2
        \Rightarrow |f''(0)| \le 2 - 2 |f'(0)|^2.
    \]
\end{proof}

\section{Homework 3}

\begin{prob}
    We call a map $f:X\to Y$ {\em proper} if $f^{-1}(K)$ is compact for all compact sets $K\subset Y$. Then an entire function $f: \BC \to \BC$ is proper if and only if $f(z)$ is a nonconstant polynomial in $z$.
\end{prob}

\begin{proof}
    Suppose that $f$ is proper. Let $K = \{|w|\le 1\}$. Since $f$ is proper, $f^{-1}(K)$ is compact. Therefore, $f^{-1}(K) \subset \{|z| \le R\}$ and hence
    \[
        f(\{|z|> R\})\cap K = \emptyset.
    \]
    So $f(\{|z|>R\})$ cannot be dense in $\BC$. By Casorati-Weierstrass, $f(z)$ has at worst a pole at $\infty$ and hence $f(z)$ is a polynomial. Clearly, $f(z)$ cannot be constant; otherwise, $f^{-1}(c) = \BC$ is not compact for some $c$.

    Suppose that $f(z) = a_0 + a_1 z + ... + a_n z^n$ is a nonconstant polynomial. To show that $f$ is proper, it suffices to show that $f^{-1}(K_r)$ is bounded for all $K_r = \{|w|\le r\}$. Since
    \[
        \lim_{z\to\infty} f(z) = \infty,
    \]
    there exists $R > 0$ such that $|f(z)| > r$ for all $|z| > R$. It follows that
    \[
        f^{-1}(K_r) \subset \{|z|\le R\}.
    \]
\end{proof}

\begin{prob}
    Prove the following variation of Rouch\'e's Theorem:
    Let $\gamma$ be a continuous closed curve homologous to $0$ in an open set $D\subset \BC$ and let $f(z)$ and $g(z)$ be two analytic functions on $D$ satisfying
    \[
        |f(z) + c_1 g(z)| > |f(z) + c_2g(z)|
    \]
    for some constants $c_1, c_2\in \BC$ satisfying $|c_1| \le |c_2|$ and all $z$ on $\gamma$. Then $f(z)$ and $g(z)$ have the same number of zeros in the interior of $\gamma$, counted with multiplicities, i.e.,
    \[
        \sum_{f(p) = 0} \nu(\gamma, p) \mult_p f
        = \sum_{g(q) = 0} \nu(\gamma, q) \mult_q g
    \]
    where $\nu(\gamma,z_0)$ is the winding number of $\gamma$ at $z_0$.
\end{prob}

\begin{proof}
    Let $h(z) = f(z)/g(z)$. Applying Argument Principle to $h(z)$ on $\gamma$,
    \[
        \begin{aligned}
            \nu(h\circ \gamma, 0) & = \frac{1}{2\pi} \int_{\gamma}\frac{h'(z)}{h(z)} dz
            = \frac{1}{2\pi i} \int_{\gamma} \frac{f'(z)}{f(z)} dz \frac{1}{2\pi} - \frac{1}{2\pi i}\int_{\gamma}\frac{g'(z)}{g(z)} dz
            \\
                                  & = \sum_{f(p) = 0} \nu(\gamma, p) \mult_p f
            - \sum_{g(q) = 0} \nu(\gamma, q) \mult_q g.
        \end{aligned}
    \]
    It suffices to show that $\nu(h\circ \gamma, 0) = 0$. By our hypothesis,
    \[
        |h(z) + c_1| > |h(z) + c_2|
    \]
    for all $z\in \gamma$ and hence
    \[
        h\circ \gamma \subset G = \big\{|w+c_1| > |w + c_2|\big\}.
    \]
    Note that $0\not\in G$ since $|c_1| \le |c_2|$. And $G$ is a half plane and hence simply connected. Therefore, $\nu(h\circ \gamma, 0) = 0$.
\end{proof}

\begin{prob}
    Compute the integral
    \[
        \int_0^\infty \frac{dx}{1 + x^r}
    \]
    for $r > 1$.
\end{prob}

\begin{proof}[Solution]
    Let us first assume that $r = p/q$ is rational for some positive integer $p$ and $q$ such that $\gcd(p,q) = 1$. Since $r > 1$, $p>q$.
    Then
    \[
        \int_0^\infty \frac{dx}{1+x^r}
        = \int_0^\infty \frac{dx}{1+t^{p/q}}
        = \int_0^\infty \frac{q t^{q-1}}{1+t^p} dt
    \]
    after the substitution $x = t^q$.

    Let $\alpha = \exp(2\pi i/p)$ and let us consider the complex integral
    \begin{equation}\label{MATH41117FFRE000}
        \int_{\gamma} \frac{q z^{q-1}}{1+z^p} dz
        = \left( \int_{\gamma_1} + \int_{\gamma_2} + \int_{\gamma_3} \right) \frac{q z^{q-1}}{1+z^p} dz
    \end{equation}
    along the curve $\gamma = \gamma_1 + \gamma_2 + \gamma_3$ given by
    \[
        \left\{
        \begin{aligned}
            \gamma_1(t) & = t \text{ for } 0\le t \le R              \\
            \gamma_2(t) & = R e^{it} \text{ for } 0 \le t \le 2\pi/p \\
            \gamma_3(t) & = (R-t) \alpha \text{ for } 0 \le t \le R
        \end{aligned}
        \right.
    \]
    for some large $R$.

    For $\gamma_2$, we have
    \begin{equation}\label{MATH41117FFRE001}
        \begin{aligned}
             & \left|\int_{\gamma_2} \frac{q z^{q-1}}{1+z^p} dz\right|
            \le \left(\frac{2\pi R}{p}\right) \frac{q R^{q-1}}{R^p - 1}
            = \frac{2\pi R^q}{r(R^p - 1)}
            \\
             & \Rightarrow \lim_{R\to\infty} \int_{\gamma_2} \frac{q z^{q-1}}{1+z^p} dz = 0
        \end{aligned}
    \end{equation}
    since $p > q$.

    For $\gamma_1$, we have
    \begin{equation}\label{MATH41117FFRE002}
        \int_{\gamma_1} \frac{q z^{q-1}}{1+z^p} dz
        = \int_0^R \frac{q t^{q-1}}{1+t^p} dt.
    \end{equation}

    For $\gamma_3$, we have
    \begin{equation}\label{MATH41117FFRE003}
        \begin{aligned}
            \int_{\gamma_3} \frac{q z^{q-1}}{1+z^p} dz
             & = -\int_0^R \frac{q \alpha^{q}(R-t)^{q-1}}{1+\alpha^p (R-t)^p} dt
            \\
             & = -\alpha^{q} \int_0^R \frac{q (R-t)^{q-1}}{1+ (R-t)^p} dt
            \\
             & = -\alpha^{q} \int_0^R \frac{q t^{q-1}}{1+ t^p} dt
        \end{aligned}
    \end{equation}
    since $\alpha^p = 1$.

    Combining \eqref{MATH41117FFRE000}-\eqref{MATH41117FFRE003}, we obtain
    \begin{equation}\label{MATH41117FFRE004}
        \lim_{R\to\infty} \int_{\gamma} \frac{q z^{q-1}}{1+z^p} dz
        = (1-\alpha^{q}) \int_0^\infty \frac{q t^{q-1}}{1+ t^p} dt
    \end{equation}

    The roots of $1+z^p$ are $\exp((2n+1)\pi i/p )$ for $0\le n < p$; among them, only $\beta = \exp(\pi i/p)$ lies inside the curve $\gamma$. Therefore, by Residue Theorem,
    \begin{equation}\label{MATH41117FFRE005}
        \begin{aligned}
            \int_{\gamma} \frac{q z^{q-1}}{1+z^p} dz
             & = 2\pi i \Res \left(\frac{q z^{q-1}}{1+z^p}, \beta\right)
            \\
             & = 2\pi i \left.
            \frac{q z^{q-1}}{(1+z^p)'}
            \right|_\beta = 2\pi i \left(\frac{q}{p}\right) \beta^{q-p}
            = - \frac{2\pi i \beta^q}{r}
        \end{aligned}
    \end{equation}
    where $qz^{q-1}(1+z^p)^{-1}$ has a simple pole at $\beta$ since $1+z^p$ has a zero at $\beta$ of multiplicity one.

    Combining \eqref{MATH41117FFRE004} and \eqref{MATH41117FFRE005}, we obtain
    \[
        \begin{aligned}
            \int_0^\infty \frac{q t^{q-1}}{1+ t^p} dt
             & = -\left(\frac{2\pi i}{r}\right)
            \frac{\beta^q}{1-\alpha^q}
            = -\left(\frac{2\pi i}{r}\right)
            \frac{\beta^q}{1-\beta^{2q}}                   \\
             & = -\left(\frac{2\pi i}{r}\right)
            \frac{1}{\beta^{-q} - \beta^q}
            \\
             & = -\left(\frac{2\pi i}{r}\right)
            \frac{1}{\exp(-q \pi i /p) - \exp(q \pi i /p)} \\
             & = -\left(\frac{2\pi i}{r}\right) \frac{1}{
                (-2i)\sin(q \pi/p)} = \frac{\pi}{r \sin(\pi/r)}
        \end{aligned}
    \]
    where we notice that $\alpha = \beta^2$.
    Therefore,
    \begin{equation}\label{MATH41117FFRE006}
        \int_0^\infty \frac{dx}{1 + x^r} = \frac{\pi}{r \sin(\pi/r)}
    \end{equation}
    for all rational numbers $r>1$.
    It is not hard to prove that
    the function
    \[
        F(r) = \int_0^\infty \frac{dx}{1 + x^r}
    \]
    is continuous for $r > 1$. Therefore,
    \[
        F(r) \equiv \frac{\pi}{r \sin(\pi/r)}
    \]
    \eqref{MATH41117FFRE006} holds for all real numbers $r>1$.
\end{proof}

\begin{prob}
    Find $\Aut(\BC^*) = \Aut(\BC -\{0\})$ and $\Aut(\BC - \{0, 1\})$.
\end{prob}

\begin{proof}
    Let us prove the following lemma:

    \begin{lem}\label{MATH506HW123LEM000}
        For a finite set $S$ of points on $\BC$, every univalent function $f(z)$ on $\BC\backslash S$ is a linear fractional transformation with singularity in $S$.
    \end{lem}

    By the above lemma, if $f\in \Aut(\BC^*)$, then $f(z) = az+b$ or $a + bz^{-1}$ for some constants $a,b\in \BC$. And since
    $0\not\in f(\BC^*)$, it is easy to see
    \[
        \Aut(\BC^*) = \{bz: b\ne 0\} \cup \left\{
        \frac{b}z: b\ne 0
        \right\}.
    \]
    Similar, $f\in \Aut(\BC-\{0,1\})$ must be one of the following:
    \[
        az +b,\ a + \frac{b}z,\ a+ \frac{b}{z-1}
    \]
    And since $0,1\not\in f(\BC-\{0,1\})$, it is easy to see
    \[
        \Aut(\BC -\{0,1\}) = \left\{
        z, 1-z, \frac{1}{z}, 1 - \frac{1}{z}, \frac{1}{1-z}, \frac{z}{z - 1}
        \right\}.
    \]
    It remains to prove the lemma.

    First, we show that $f$ has at worst poles at $S\cup \{\infty\}$. We choose a closed disk $D\subset \BC\backslash S$ of positive radius. By Open Mapping, $f(D)$ contains a nonempty open set $G$. Since $f$ is $1$-$1$,
    \[
        f(\BC\backslash D) \cap G = \emptyset.
    \]
    Therefore, $f(\{0<|z-p|<\varepsilon\}) \cap G = \emptyset$ for some $\varepsilon > 0$ and $p\in S$ as long as
    \[
        \{|z-p|<\varepsilon\} \cap D = \emptyset.
    \]
    By Casorati-Weierstrass, $f(z)$ has at worst poles at $S$.

    Similarly, $f(\{|z| > R\}) \cap G=\emptyset$ as long as
    $D\subset \{|z| \le R\}$. So $f(z)$ has at worst poles at $\infty$. In conclusion, $f(z)$ is an analytic function on $\BC\backslash S$ with at worst poles at $S\cup \{\infty\}$. So $f(z)$ has to be a rational function
    $f(z)$ with poles in $S\cup \{\infty\}$.

    Second, we prove that $f(z)$ has simple poles at every singularity among $S\cup \{\infty\}$.
    Otherwise, suppose that $f(z)$ has a pole of order $m\ge 2$ at $p$. Then
    there exists an open neighborhood $U$ of $p$ such that
    $f(z)\ne 0$ in $U^*=U\backslash \{p\}$. So $f: U^*\to \BC^*$ is analytic and $1$-$1$. Consequently, $g(z) = 1/f(z)$ is also $1$-$1$ on $U^*$. Since $g(z)$ has a removable singularity at $p$ and $g(p) = 0$, $g(z)$ extends to a univalent function on $U$. So $g'(p)\ne 0$. But $g(z)$ has a zero of multiplicity $m\ge 2$ at $p$. Contradiction.

    Finally, we prove that $f(z)$ has at most one pole among $S\cup \{\infty\}$. Otherwise, suppose that $f(z)$ has two poles $p\ne q$. We choose $U_p$ and $U_q$ to be open neighborhoods of $p$ and $q$, respectively, such that $U_p\cap U_q = \emptyset$ and $f(z)\ne 0$ on
    $U_p^*\cup U_q^*$ for $U_p^* = U_p\backslash \{p\}$ and $U_q^* = U_q\backslash \{q\}$. As before, $g(z) = 1/f(z)$ is $1$-$1$ on $U_p^*\sqcup U_q^*$ and extends to an analytic function on $U_p\sqcup U_q$ with $g(p) = g(q) = 0$. By Open Mapping, $g(U_p)\cap g(U_q)$ contains an open disk $D$ with $0\in D$. Thus, for every $w\in D\backslash \{0\}$, there exist $z_p\in U_p^*$ and $z_q\in U_q^*$ such that
    $g(z_p) = g(z_q) = w$. This contradicts the fact that $g$ is $1$-$1$ on $U_p^*\sqcup U_q^*$.

    In conclusion, $f(z)$ has at most one simple pole and hence a linear fractional transformation.
\end{proof}

\begin{prob}
    Let $\lambda_1, \lambda_2\ne 0,1$ be two complex numbers. Show that $\BC - \{0, 1,\lambda_1\}$ and $\BC - \{0, 1, \lambda_2\}$ are biholomorphic if and only if
    \[
        \lambda_1 \in \left\{
        \lambda_2, \frac{1}{\lambda_2}, 1 - \lambda_2, 1 - \frac{1}{\lambda_2}, \frac{1}{1-\lambda_2}, \frac{\lambda_2}{\lambda_2 - 1}
        \right\}.
    \]
    In other words, they are biholomorphic if and only if there exists $f\in \Aut(\BC - \{0, 1\})$ such that $\lambda_1 = f(\lambda_2)$.
\end{prob}

\begin{proof}
    By Lemma \ref{MATH506HW123LEM000}, $f$ must be one of the following:
    \[
        az +b,\ a + \frac{b}z,\ a+ \frac{b}{z-1},\ a+\frac{b}{z-\lambda_1}
    \]
    And since $0,1,\lambda_2\not\in f(\BC - \{0,1,\lambda_1\})$, we conclude
    \[
        \begin{aligned}
            f(z) & = z,\ 1-z,\ \frac{z}{\lambda_1},\ 1-\frac{z}{\lambda_1},\ \frac{1-z}{1-\lambda_1},\ \frac{z-\lambda_1}{1- \lambda_1},                                                \\
                 & \quad \frac{1}{z},\ 1 - \frac{1}{z},\ \frac{\lambda_1}{z}, \ 1 - \frac{\lambda_1}{z},\ \frac{\lambda_1(z-1)}{(\lambda_1 - 1)z},\ \frac{z-\lambda_1}{z(1-\lambda_1)},
            \\
                 & \quad\frac{z(1-\lambda_1)}{z-\lambda_1},\ \frac{\lambda_1(z-1)}{z-\lambda_1},\ \frac{z}{z-\lambda_1},\
            \frac{\lambda_1}{\lambda_1 - z},\ \frac{z-1}{z-\lambda_1} \text{ or } \frac{1-\lambda_1}{z - \lambda_1}.
        \end{aligned}
    \]
    It is easy to see that $\lambda_2$ is one of the limits of $f(z)$ as $z\to 0,1,\lambda_1,\infty$. Then it follows
    \[
        \lambda_2 \in \left\{
        \lambda_1, \frac{1}{\lambda_1}, 1 - \lambda_1, 1 - \frac{1}{\lambda_1}, \frac{1}{1-\lambda_1}, \frac{\lambda_1}{\lambda_1 - 1}
        \right\}
    \]
    which is equivalent to
    \[
        \lambda_1 \in \left\{
        \lambda_2, \frac{1}{\lambda_2}, 1 - \lambda_2, 1 - \frac{1}{\lambda_2}, \frac{1}{1-\lambda_2}, \frac{\lambda_2}{\lambda_2 - 1}
        \right\}.
    \]
\end{proof}

\begin{prob}
    Let $D = \{|z| < 1\}$ and $H(D)$ be the space of holomorphic functions on $D$. Show that $F\subset H(D)$ is normal if and only if there is a sequence $\{M_n\}$ of positive constants such that
    $\limsup \sqrt[n]{M_n} \le 1$ and $|a_n|\le M_n$ for all $n$ and
    all $f(z) = \sum_{n=0}^\infty a_n z^n \in F$.
\end{prob}

\begin{proof}
    Suppose that $F$ is normal. Let
    \[
        M_n = \sup_{f\in F} \frac{|f^{(n)}(0)|}{n!} = \sup \left\{
        |a_n|: \sum a_m z^m \in F
        \right\}.
    \]
    Since $F$ is normal, $\{f^{(n)}(z): f\in F
        \}$ is normal for all $n\in \BN$. So the set $\{|f^{(n)}(0)|: f\in F\}$ is uniformly bounded. Consequently, $M_n <\infty$ for all $n$. By the definition of $M_n$, $|a_n| \le M_n$ for all $n$ and $\sum a_m z^m \in F$.

    For all $0< r<1$, $F$ is uniformly bounded on $\{|z|\le r\}$. Hence there exists $C>0$ such that $|f(z)| \le C$ for all $f\in F$ and $|z|\le r$. Then
    \[
        \frac{|f^{(n)}(0)|}{n!} = \left|
        \frac{1}{2\pi i} \int_{|z|=r} \frac{f(z)}{z^{n+1}} dz
        \right| \le \frac{C}{r^n}
    \]
    for all $n$ and all $f\in F$. Then $M_n \le C/r^n$ and
    \[
        \limsup \sqrt[n]{M_n} \le \frac{1}r \limsup \sqrt[n]{C} = \frac{1}r
    \]
    for all $0< r<1$. Therefore, $\limsup \sqrt[n]{M_n}\le 1$.

    On the other hand, suppose that there exists such a sequence $\{M_n\}$. Since $\limsup \sqrt[n]{M_n}\le 1$, for every $0<R<1$, there exists $C$ such that $M_n \le C R^{-n}$ for all $n$. Then
    \[
        |f(z)| \le \sum_{n=0}^\infty M_n r^n \le \sum_{n=0}^\infty CR^{-n}r^n =\frac{CR}{R-r}
    \]
    for all $f\in F$ and $|z|\le r < R$. So $F$ is uniformly bounded on $\{|z|\le r\}$ for all $r < 1$. Consequently, $F$ is normal.
\end{proof}

\begin{prob}
    Let $G$ be a connected open set in $\BC$ and $H(G)$ be the space of holomorphic functions on $G$. For a sequence $\{f_n\}\subset H(G)$ of one-to-one functions which converge
    to some $f\in H(G)$ locally uniformly, show that $f$ is either one-to-one or a constant function.
\end{prob}

\begin{proof}
    It suffices to show that for every $c\in \BC$, either $f(z)\equiv c$ or $f(z) - c$ has at most one zero in $G$. Otherwise, suppose that $f(z)\not\equiv c$ and $f(z) - c$ has two zeros $z_1\ne z_2$ in $G$.

    We choose $r>0$ such that $K = \{|z - z_1| \le r\}\sqcup \{|z - z_2| \le r\}\subset G$ and $f(z)\ne c$ for all $z\in K\backslash \{z_1,z_2\}$. Let
    \[
        M = \min_{z\in \partial K} |f(z) - c| = \min\left(
        \min_{|z-z_1|=r} |f(z) - c|, \min_{|z-z_2|=r} |f(z) - c|
        \right).
    \]
    Since $f_n$ converges to $f$ uniformly on $K$, there exists $N$ such that
    \[
        ||f - f_n||_K < M
    \]
    for all $n > N$. By Rouch\'e's Theorem, since
    \[
        |(f(z)-c) - (f_n(z) - c)| \le ||f-f_n||_K < M \le |f(z)-c|
    \]
    for $n>N$, $|z-z_j|=r$ and $j=1,2$, $f(z)-c$ and $f_n(z)-c$ has the same number of zeros in $|z-z_j| < r$. Therefore, $f_n(z) - c$ has at least two zeros for $n>N$, which contradicts the hypothesis that $f_n$ is $1$-$1$.
\end{proof}

\begin{prob}
    Let $G_1, G_2\subsetneq \BC$ be simply connected open sets
    and $f: G_1\to G_2$ be a biholomorphic map from $G_1$ to $G_2$. Suppose that $f(z_1) = z_2$. Show that for every one-to-one holomorphic map $g: G_1\to G_2$ satisfying $g(z_1) = z_2$,
    $|g'(z_1)| \le |f'(z_1)|$.
\end{prob}

\begin{proof}
    By Riemann Mapping Theorem, there exist biholomorphic maps $s_j: G_j \to D$
    for $D= \{|z|< 1\}$ and $j=1,2$. We can choose $s_j$ such that $s_j(z_j) = 0$ for $j=1,2$.

    By Problem \ref{MATH506HW123PROB23},
    \[
        \begin{aligned}
            |(s_2\circ g \circ s_1^{-1})'(0)|\le |(s_2\circ f \circ s_1^{-1})'(0)|
             & \Rightarrow \left|\frac{s_2'(z_2) g'(z_1)}{s_1'(z_1)}\right|
            \le \left|\frac{s_2'(z_2) f'(z_1)}{s_1'(z_1)}\right|
            \\
             & \Rightarrow |g'(z_1)| \le |f'(z_1)|.
        \end{aligned}
    \]
\end{proof}

\begin{prob}
    Let $f(z)$ and $g(z)$ be entire functions such that
    $e^{f(z)},  e^{g(z)}$ and $1$ are linearly dependant over $\BC$, i.e., there exist $c_1, c_2, c_3\in \BC$, not all zero,
    such that $c_1 e^{f(z)} + c_2 e^{g(z)} + c_3 = 0$ for all $z$.
    Then $f(z), g(z)$ and $1$ are linearly dependent over $\BC$.
\end{prob}

\begin{proof}
    If one of $c_1,c_2,c_3$ vanishes, then it is obvious that $f(z),g(z)$ and $1$ are linearly dependent over $\BC$. Otherwise, suppose that $c_1,c_2,c_3\ne 0$. Then
    \[
        e^{f(z)} = -\frac{c_2}{c_1} e^{g(z)} - \frac{c_3}{c_1}\not\in \left\{0, -\frac{c_3}{c_1}\right\}
    \]
    for all $z\in \BC$. By Picard's Little Theorem, $e^{f(z)}$ is constant and hence $f(z)$ is constant. So $f(z)$ and $1$ are linearly dependent over $\BC$.
\end{proof}

\begin{prob}
    Let $f(x,y)$ and $g(x,y)$ be real-valued harmonic functions on $\BR^2$ such that
    $e^{f(x,y)}$, $e^{g(x,y)}$ and $1$ are linearly dependant over $\BR$. Then
    $f(x,y)$, $g(x,y)$ and $1$ are linearly dependent over $\BR$.
\end{prob}

\begin{proof}
    Suppose that $c_1 e^{f(x,y)} + c_2 e^{g(x,y)} + c_3 = 0$ for some $c_1,c_2,c_3\in \BR$, not all zero, and all $(x,y)\in \BR^2$.

    If one of $c_1,c_2,c_3$ vanishes, it is obvious that $f(x,y),g(x,y)$ and $1$ are linearly dependent over $\BR$. Otherwise, suppose that $c_1,c_2,c_3\ne 0$.

    WLOG, suppose that $c_1 > 0$. If $c_2 > 0$, then
    \[
        c_1 e^{f(x,y)} = c_3 - c_2 e^{g(x,y)} < c_3\Rightarrow f(x,y) < \ln c_3 - \ln c_1
    \]
    and hence $f(x,y)$ is constant by Louiville's Theorem on harmonic functions over $\BR^2$. Suppose that $c_2 < 0$. If $c_3 > 0$, then
    \[
        -c_2 e^{g(x,y)} = c_1 e^{f(x,y)} + c_3 > c_3\Rightarrow g(x,y) > \ln c_3 - \ln(-c_2)
    \]
    and hence $g(x,y)$ is constant. If $c_3 < 0$, then
    \[
        c_1 e^{f(x,y)} = -c_2 e^{g(x,y)} - c_3 > -c_3\Rightarrow f(x,y) > \ln(-c_3) - \ln c_1
    \]
    and hence $f(x,y)$ is constant. In conclusion, $f(x,y), g(x,y)$ and $1$ are linearly dependent over $\BR$.
\end{proof}
\end{document}