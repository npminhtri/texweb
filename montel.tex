\documentclass[12pt]{article} % \documentclass{} is the first command in any LaTeX code.  It is used to define what kind of document you are creating such as an article or a book, and begins the document preamble

\usepackage{amsmath} % \usepackage is a command that allows you to add functionality to your LaTeX code

\usepackage[papersize={216mm,330mm},tmargin=20mm,bmargin=20mm,lmargin=20mm,rmargin=20mm]{geometry}
\usepackage[english]{babel}
\usepackage[utf8]{inputenc}
\usepackage{amsmath,amssymb,mathabx,amsthm}%\for eqref
\usepackage{faktor}
\usepackage{lscape}
\usepackage{graphicx}
\usepackage[colorinlistoftodos]{todonotes}
\usepackage{fancyhdr}
\usepackage{hyperref} %creat hyperlink
\hypersetup{
    colorlinks=true,
    linkcolor=blue,
    filecolor=magenta,      
    urlcolor=cyan,
    pdftitle={Overleaf Example},
    pdfpagemode=FullScreen,
    } %set up a hyperlink to be in blue 
\newtheorem{theorem}{Theorem}
\newtheorem{definition}{Definition}
\newtheorem{lemma}[theorem]{Lemma}
\newtheorem{corollary}[theorem]{Corollary}
\DeclareMathOperator{\pj1}{\mathbb{P}^1(\mathbb{Q})}
\DeclareMathOperator{\sl2}{SL_2(\mathbb{Z})}
\pagestyle{fancy}
\fancyhf{}
\renewcommand{\baselinestretch}{1.2} % increase the distance between line.

\title{Montel theorem and some related results} % Sets article title
\author{Tri Nguyen - University of Alberta} % Sets authors name
\date{\today} % Sets date for date compiled

% The preamble ends with the command
\begin{document}
\maketitle % creates a title using the information in the preamble (title, author, date)
In this expository note, I will try to explain explicitly how to compactify $\Gamma\backslash \mathbb{H}$ by adding points in two ways.
\section{Some preparations}
We will always denote $\Gamma$ a subgroup of the group $SL_2(\mathbb{Z})$ of finite index, and this group acts on
the upper half complex plane $\mathbb{H}$ by
\[\begin{bmatrix}
        a & b \\
        c & d
    \end{bmatrix} \circ z := \dfrac{az+b}{cz+d} \]
When $z$ tends to infinity, we have
\[\lim_{z \to \infty} \dfrac{az+b}{cz+d}= \frac{a}{c},\]
so we add the rational line to define the action of this group at $\infty$. In particular, we consider the set
\[\overline{\mathbb{H}} = \mathbb{H} \cup \mathbb{P}^1(\mathbb{Q})\]
Note that on the projective rational line, we define the action to be the multiplication of a $2 \times 2$ matrix with a $2\times 1$ vector.
Then under this action, we have the following lemma
\begin{lemma}
    $SL_2(\mathbb{Z})$ acts transitively on $\mathbb{P}^1(\mathbb{Q})$.
\end{lemma}
\begin{proof}
    For each point in $\pj1$, we can choose the representative to be of the form $[a:b]$, where
    $\gcd(a,b)=1$. Then there exists $x,y \in \mathbb{Z}$ such that
    \[ax - by = 1\]
    Thus we get the following equality
    \[ \begin{bmatrix}
            b  & a \\
            -x & y \\
        \end{bmatrix}\begin{bmatrix}
            a \\
            b
        \end{bmatrix} = \begin{bmatrix}
            0 \\
            1
        \end{bmatrix}\]
    This implies any points in $\pj1$ can be moved to $[0:1]$, and thus the action is transitive.
\end{proof}

\begin{corollary}
    If $\Gamma$ is a subgroup of finite index in $\sl2$ then $\Gamma \backslash \pj1$ has only finite orbits.
\end{corollary}
\section{Compactification of $\Gamma \backslash \mathbb{H}$ by adding points.}
We introduction a topology on $\overline{\mathbb{H}}$. For the usual upper half plane, the topology are the usual metric
topology on $\mathbb{C}$, and we only define the system of neighborhood of $r \in \pj1$.

Let $S(c, \omega)$ be the circle that touches the real line at $\omega = p/q$ and has the radius $\frac{c}{2q^2}$.
Then the collection of circle $D(c,\omega ) = \bigcup_{0<c' \le c} S(c',\omega)$ is called \textit{Farey disk}. Let $c \to 0$, these disks
define a neighborhood of $\omega$. The Farey disks at $\infty$ is defined to be the region
\[ D(T,\infty ) = \left\lbrace z: \Im z \ge T \right\rbrace\]
\end{document}