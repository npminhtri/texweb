\documentclass[12pt]{article} % \documentclass{} is the first command in any LaTeX code.  It is used to define what kind of document you are creating such as an article or a book, and begins the document preamble
\usepackage{amsmath} % \usepackage is a command that allows you to add functionality to your LaTeX code

\usepackage[papersize={216mm,330mm},tmargin=20mm,bmargin=20mm,lmargin=20mm,rmargin=20mm]{geometry}
\usepackage[english]{babel}
\usepackage[utf8]{inputenc}
\usepackage{amsmath,amssymb,mathabx,amsthm}%\for eqref
\usepackage{lscape}
\usepackage{graphicx}
\usepackage[colorinlistoftodos]{todonotes}
\usepackage{fancyhdr}
\usepackage{hyperref} %creat hyperlink
\hypersetup{
    colorlinks=true,
    linkcolor=blue,
    filecolor=magenta,      
    urlcolor=cyan,
    pdftitle={Overleaf Example},
    pdfpagemode=FullScreen,
    } %set up a hyperlink to be in blue 
\newtheorem{theorem}{Theorem}
\newtheorem{definition}{Definition}
\pagestyle{fancy}
\fancyhf{}
\setlength\parindent{0pt} % noindent for the whole document.
\renewcommand{\baselinestretch}{1.2} % increase the distance between line.
\DeclareMathOperator{\frkh}{\mathfrak{h}}
\DeclareMathOperator{\frkg}{\mathfrak{g}}
\DeclareMathOperator{\ad}{ad}
\DeclareMathOperator{\gl}{\mathfrak{\mathfrak{gl}}(n,F)}


\title{Lie theory - homework 1} % Sets article title
\author{Tri Nguyen - University of Alberta} % Sets authors name
\date{\today} % Sets date for date compiled

% The preamble ends with the command \begin{document}
\begin{document}
\maketitle
\textbf{Problem 1}
\begin{enumerate}
    \item By definition, we only need to check that
          \[[[a,b],c] \in \mathfrak{g} \quad \forall a,b \in \frkh, c \in \frkg,\]
          but this is clear as $\frkh$ is an ideal of $\frkg$, we could use Jacobi's identity to get
          \[[[a,b],c] = [b,[c,a]]+ [a,[b,c]] \in [\frkh,\frkh].\]
    \item Recall that $\mathcal{D}^{k+1} \frkg = [\frkg^k,\frkg^k]$. Clearly $\frkg$ is itself an ideal, so the fact that
          $\mathcal{D}^{k+1}\frkg$ follows immediately from part a and induction on $k$.
    \item In class, we called $\frkg$ semisimple iff it has no nontrivial solvable ideal. Note that abelian ideals are solvable, hence all abelian ideals are zero if $\frkg$ is semisimple.
          Conversely, assume that $\frkg$ is not semisimple, then it has a non trivial solvable ideal $\frkh$. In particular, we have a strictly decreasing chain of ideals as follows:
          \[\frkh = \frkh^{(0)} \supset \frkh^{(1)} \supset \ldots \supset \frkh^{(n)} \supset \frkh^{(n+1)}= (0)\]
          But this implies that $\frkh^{(n)}$ is a non trivial abelian ideal of $\frkg$ by part a.
\end{enumerate}
\textbf{Problem 2}
We compute $\ad x$ with respect to this basis. The other two are computed similarly.
\begin{align}
     & \ad x (x) = [x,x]=0                           \\
     & \ad x (y) = [x,y] = xy-yx = \begin{bmatrix}
                                       1 & 0  \\
                                       0 & -1
                                   \end{bmatrix} = h \\
     & \ad x(h) = \begin{bmatrix}
                      0 & -2 \\
                      0 & 0
                  \end{bmatrix}=-2x
\end{align}
So with respecto to the basis $\left\lbrace x,h,y\right\rbrace$, the linear map $\ad x$ correspond to the matrix
\[ \ad x =\begin{bmatrix}
        0 & -2 & 0 \\
        0 & 0  & 1 \\
        0 & 0  & 0
    \end{bmatrix}\]
similarly, we could find that
\[\ad y = \begin{bmatrix}
        0  & 0 & 0 \\
        -1 & 0 & 0 \\
        0  & 2 & 0
    \end{bmatrix} ,  \ad h = \begin{bmatrix}
        2 & 0 & 0  \\
        0 & 0 & 0  \\
        0 & 0 & -2
    \end{bmatrix}\]
\textbf{Problem 4}
\begin{enumerate}
    \item The given matrix algebra $L$ is generated by 3 following linearly independent elements
          \[ a =\begin{bmatrix}
                  0 & 1 & 0 \\
                  0 & 0 & 0 \\
                  0 & 0 & 0
              \end{bmatrix}, \quad b = \begin{bmatrix}
                  0 & 0 & 1 \\
                  0 & 0 & 0 \\
                  0 & 0 & 0
              \end{bmatrix}, \quad c=\begin{bmatrix}
                  0 & 0 & 0 \\
                  0 & 0 & 1 \\
                  0 & 0 & 0
              \end{bmatrix} \]
          Moreover, it is easy to check that, with respect to the usual lie bracket over matrix algebra, we have
          \[[a,c] = b, \quad [a,b]=[b,c] = 0\]
          Since $L$ and $V$ has the same dimension, they are isomorphic as vector spaces. Thus, there exists a linear map $f \colon V \to L$ such that
          $f(x)=a, f(y)=c$ and $f(z)=b$. Hence
          \[f([x,y]) = b = [a,c] = [f(x),f(y)]\]
          Thus $f$ is a Lie algebra isomorphism.
    \item From the definition, we could see that the derived algebra of $V$ is generated by 1 element $z$, hence abelian, and hence the second term
          in lower central series vanishes.

\end{enumerate}
\textbf{Problem 3}

Since the matrix $x$ has $n$ distinct eigenvalues and has size $n \times n$, $x$ has $n$ linearly independent eigenvectors.

We denote $v_i$ be the vector such that $x\cdot v_i = a_iv_i$. Then $\left\lbrace v_i\right\rbrace_i^n$ can be chosen as a basis of $\mathbb{R}^n$.
With respect to this basis, $x$ can be associated with the diagonal vector, with $a_i$'s on the diagonal. Let $\left\lbrace e_{ij}\right\rbrace_{i=1}^n$ be the standard basis of $\gl$.
It can be verified that
\[\ad x(e_{ij}) = xe_{ij} -e_{ij}x = a_i -a_j, \quad \forall 1 \le i,j \le n\]
This implies that the standard basis $\left\lbrace e_{ij}\right\rbrace_{i=1}^n$ is the full set of eigenvectors of $\ad x$, and the corresponding eigenvalues are scalars $a_i-a_j$.
\end{document}