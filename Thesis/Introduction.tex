\chapter{Introduction}
Semi-stable lattices are first appeared in the paper of Stuhler \cite{MR424707}. He was inspired 
by some work of Harder and Narasimhan on semi-stable vector bundles over curves \cite{}, and studied the analogy 
in the context of lattice theory. The notion of semi-stability as well as canonical filtration were then introduced by him.
Stuhler also discovered the connection between semi-stability and Minkowski's reduction theory. In particular, he showed 
that, if a lattice $(L,q)$ has a Minkowski reduced basis $\{e_1,\ldots e_n\}$ such that 
$q(e_{i+1})>\alpha\cdot q(e_{i})$ for some suitable $\alpha$ then the set 
\[0 \subset \mathbb{Z}e_1 \subset \mathbb{Z}e_1 \oplus \mathbb{Z}e_1 \subset \ldots \subset \mathbb{Z}e_1\oplus\mathbb{Z}e_2\oplus \ldots \oplus \mathbb{Z}e_n\]
forms a canonical plot for $L$. We refer to the original paper \cite{MR424707} or the book \cite{MR4505757} for a detailed treatment of Stuhler's result. In 1978, Grayson reproduced the results of Stuhler
and used this to prove some theorems about arithmetic groups. The difference is that, he associated each lattice to a graphical data called 
\textit{canonical plot}, but "the  effect on the clarity of arguments is dramatic." \cite{MR2127941}. He also recovered 
some result in classical reduction theory without the need to use the notion of Siegel's fundamental domain. Along these lines, there are various result on semistability, such as 
\begin{enumerate}
    \item In \cite{MR4505757}, the author extended the original paper of Stuhler to study the shortest vector problem. In particular, he gave a relation between
    for the first successive minima and the slope of the lattice. 
    \item In \cite{MR2127941}, Bill Casselman studied the partition of arithmetic group using semi-stability. His paper hinted a connection between the partition using semistablity and Arthur's combinatorial lemma in the theory of automorphic forms.
    \item In a series of 3 papers \cite{MR3519540}, \cite{MR3268755} and \cite{MR3605031}, Uri Shapira et.al studied the relations semi-stable lattices in the context of homogenous dynamics. One of the 
    main results involves the relation between semi-stable lattices and well-rounded lattice, which then simplify a conjecture of Minkowski on the geometry of number.     
\end{enumerate}
This thesis has two goals: The first is to give a survey of the geometric picture of semi-stable lattice, following \cite{MR780079} and \cite{MR2127941}. 
The second is to compare the semi-stability in geometric sense, following Grayson, with semi-stability in Lie theoretic sense as introduced in the papers 
\cite{MR3969872} and \cite{patnaik2025borel}. Ultimately, we will prove that these two notions of semi-stability are essentially the same. 
\subsection*{Outline of the thesis}
\begin{itemize}
    \item In chapter II, we go over the theory of two dimensional lattice. We also introduce the semistability for lattices, encoded by the upper half plane model. Using the graphical data, we showed that semi-stability and $\rho-$semi-stability are the same.
    \item In Chapter III, we recalled the necessary Lie theory for $\mathrm{SL}_n$ and $\mathrm{GL}_n$. They will be needed later in introducing the notion of $\rho$-semi-stability. 
    \item In chapter IV, we defined lattices in higher rank and related properties. Then semi-stabled latticed and canonical filtration are studied in detailed, following the papers \cite{MR780079} and \cite{MR2127941}. At the end of this chapter, $\rho$-semi-stability along with some of its basic properties were given. When the lattice is not semi-stable, each lattice is attached to a canonical filtration. Similarly, if a point $x \in \GLn$ is not $\rho$-semi-stable, we associated $x$ to a canonical pair $\mathrm{cp}(x) = (P,\delta)$.
    \item In the last chapter, we give two main theorems: The first theorem involves the equivalence between semi-stability in Grayson's sense and $\rho$-semi-stability. Furthermore, we showed that the canonical filtration and the canonical pair essentially encode the same data.
\end{itemize}