\usepackage{amsmath} % \usepackage is a command that allows you to add functionality to your LaTeX code

\usepackage[papersize={216mm,330mm},tmargin=20mm,bmargin=20mm,lmargin=20mm,rmargin=20mm]{geometry}
\usepackage[english]{babel}
\usepackage[utf8]{inputenc}
\usepackage{amsmath,amssymb,mathabx,amsthm}%\for eqref
\usepackage{mathabx}
\usepackage{lscape}
\usepackage{graphicx}
\usepackage{tikz}\usetikzlibrary{arrows.meta,calc} %library tikz
\usepackage{subcaption}


\usepackage{pgfplots}
\pgfplotsset{compat=1.15}
\usepackage{mathrsfs}
\usetikzlibrary{arrows}
\usepackage{pstricks}
\usepackage{multido}
\usepackage{pst-plot}
\usepackage{rank-2-roots}


\usepackage{color,soul} %package for highlining
\usepackage[colorinlistoftodos]{todonotes}
\usepackage{fancyhdr}
\usepackage{hyperref} %creat hyperlink
\hypersetup{
    colorlinks=true,
    linkcolor=blue,
    filecolor=magenta,      
    urlcolor=cyan,
    pdftitle={Overleaf Example},
    pdfpagemode=FullScreen,
    } %set up a hyperlink to be in blue 
\newtheorem{definition}{Definition}[section]
\newtheorem{theorem}[definition]{Theorem}
\newtheorem{prop}[definition]{Proposition}
\newtheorem{lemma}[definition]{Lemma}
\newtheorem{example}[definition]{Example}

\newtheorem*{remark}{Remark}
\lhead{}%clear the lead head



\pagestyle{fancy}
\cfoot{\thepage} % this is for the page numbering
\setlength\parindent{0pt} % noindent for the whole document.
\renewcommand{\baselinestretch}{1.2} % increase the distance between line.

\DeclareMathOperator{\SLn}{\text{SL}_n(\mathbb{R})}
\DeclareMathOperator{\GLn}{\text{GL}_n(\mathbb{R})}
\DeclareMathOperator{\slnz}{SL_n(\mathbb{Z})}
\DeclareMathOperator{\glnz}{GL_n(\mathbb{Z})}
\DeclareMathOperator{\vol}{vol}
\DeclareMathOperator{\rk}{rank}
\DeclareMathOperator{\Hom}{Hom}
\DeclareMathOperator{\fg}{\mathfrak{g}}
\DeclareMathOperator{\fh}{\mathfrak{h}}
\DeclareMathOperator{\SO}{SO_n(\mathbb{R})}
\DeclareMathOperator{\On}{O_n(\mathbb{R})}
\DeclareMathOperator{\slnr}{\mathfrak{sl}_n(\mathbb{R})}
\DeclareMathOperator{\Ad}{\text{Ad}}
\DeclareMathOperator{\SLR}{SL_2(\mathbb{R})}
\DeclareMathOperator{\slz}{SL_2(\mathbb{Z})}
\DeclareMathOperator{\sl2}{\mathfrak{sl}_2(\mathbb{R})}
\DeclareMathOperator{\SO2}{SO_2(\mathbb{R})}
\DeclareMathOperator{\uH}{\mathfrak{H}}
