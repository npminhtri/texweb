

\chapter{BASIC LIE THEORY FOR  $\SLn\& \GLn$}
% Sets article title
In this chapter, we review the basic theory of roots and weights for the Lie groups $\SLn$ and $ \GLn$.
\section{Lie algebras}
Throughout this section, the ground field is $k=\mathbb{R}$.
\subsection{Definition}
A Lie algebra $\mathfrak{g}$ is a vector space over $k$ such that the Lie bracket
\begin{align*}
    [\cdot,\cdot] \colon \mathfrak{g} \times \mathfrak{g} & \to \mathfrak{g} \\
    (x,y   )                                              & \mapsto [x,y]
\end{align*}
satisfies the following axioms
\begin{enumerate}
    \item $[x,y]$ is bilinear.
    \item $[x,x]=0$ for all $x \in \mathfrak{g}$.
    \item $[x,[y,z]]+ [z,[x,y]] + [y,[z,x]] =0$
\end{enumerate}
The last property is called \textit{Jacobi's identity}.
\begin{example}
    The first example is the set of all matrices of size $n \times n$, denoted by $\glnr$. We
    define the Lie bracket by
    \[[A,B] :=AB-BA,\]
    for arbitrary matrices $A,B \in \glnr$. It is clearly that $[A,B]$ is bilinear and $[A,A]=0$ for all matrices $A$.
    We need to verify the Jacobi's identity. Note that
    \begin{align*}
        [A,[B,C]] & = A(BC-CB)-(BC-CB)A     \\
                  & = ABC - ACB - BCA + CBA
    \end{align*}
    It follows that
    \begin{align*}
          & [A,[B,C]]+[C,[A,B]] + [B,[C,A]] \\
        = & ABC - ACB - BCA + CBA           \\
        + & CAB-CBA-ABC+BAC                 \\
        + & BCA-BAC-CAB+ABC                 \\
        = & 0
    \end{align*}
    Therefore $\glnr$ is a Lie algebra.
\end{example}
\begin{example}\label{slnr-example}
    Another example is the subspace $\slnr$ of $\glnr$, defined by
    \[\slnr = \left\lbrace A \in \glnr : tr(A) = 0\right\rbrace\]
    The Lie bracket over $\slnr$ is just the restriction of the Lie bracket over $\glnr$. Indeed, for
    any $A,B \in \slnr$ we have
    \[tr[A,B] = tr(AB-BA)= tr(AB)-tr(BA)=0\]
    Hence the restriction of the Lie bracket over $\glnr$ are a Lie bracket over $\slnr$. The 3 axioms for the Lie bracket is automatically satisfied.
\end{example}

\subsection{Cartan Subalgebras}
First we need the notion of Cartan subalgebras
\begin{definition}
    For any Lie algebra $\mathfrak{g}$, a subalgebra $\mathfrak{h}$ of $\fg$ is said to be a \textit{Cartan algebra} if it is
    \begin{itemize}
        \item $\fh$ is a nilpotent subalgebra.
        \item It is self normalizing. In particular, we have $\fh = \left\lbrace x \in \fg : [x,\fg] \subset \fg\right\rbrace$.
    \end{itemize}
\end{definition}
A priori, it is not clear whether such a subalgebra exists for any Lie algebras. However, we have the following theorem
\begin{theorem}
    Let $L$ be a Lie algebra of dimension $n$ over a field $F$ of at least $n-1$ elements. Then $L$ has a Cartan subalgebra.
\end{theorem}
A proof for this theorem can be found in \cite[Corollary 1.2]{barnes1967cartan}. In particular, any finite dimensional Lie algebra over an infinite field has a Cartan subalgebra.
\begin{example}
    For the simple Lie algebra $\slnr$, one choice of  Cartan subalgebra $\mathfrak{h}$ is the set
    \[\fh = \left\lbrace H= \begin{bmatrix}
            a_1    & 0      & \ldots & 0      \\
            0      & a_2    & \ldots & 0      \\
            \vdots & \vdots & \ddots & \vdots \\
            0      & 0      & \cdots & a_n
        \end{bmatrix}, a_1 + a_2+ \ldots + a_n = 0\right\rbrace\]
\end{example}
\begin{proof}
    Indeed, it is obvious that $[\mathfrak{h},\mathfrak{h}]=0$, thus the Lie algebra $\mathfrak{h}$ is nilpotent. To show the
    self-normalizing property, we pick an any element $\sum_{i,j}a_{ij}E_{ij}$ such that
    \[\left[\sum_{i,j}a_{ij}E_{ij},\mathfrak{h}\right]\subset \mathfrak{h}\]
    Here $E_{ij}$ denotes the matrix that has 1 at the $ij$ entry and 0 otherwise. Clearly the matrix
    $E_{pp}-E_{qq} \in \mathfrak{h}$ for $p \ne q$. We then have
    \[\left[\sum_{i,j}a_{ij}E_{ij},E_{pp}-E_{qq}\right] \in \mathfrak{h}\]
    or equivalently
    \[\sum_{i}a_{ip}E_{ip}-\sum_{i}a_{iq}E_{iq}-\sum_j a_{pj}E_{pj}-\sum_j a_{qj}E_{qj} \in \mathfrak{h}\]
    The coefficients of $E_{pq}$ in the above sum is $-2a_{pq}$. By our choice of $\mathfrak{h}$ it must be the case that $a_{pq}=0$. Thus,
    only the coefficients of $E_{pq}$ for $p=q$ survive. Hence
    $\sum_{i,j}a_{ij}E_{ij} \in \mathfrak{h}$  and $\mathfrak{h}$ is then a Cartan subalgebra of $\slnr$.
\end{proof}
\subsection{Root space decomposition}\label{root-space-decomp}
Given a semisimple Lie algebra $\mathfrak{g}$. With respect to a choice of Cartan subalgebra $\mathfrak{h}$, we have a root space decomposition. In particular, there is a finite set
$\Phi \subset \fh^{*}$ of linear forms on $\fh$, whose elements are called \textbf{roots}, such that
\[\fg = \fh \oplus \left(\bigoplus_{\alpha \in \Phi} \mathfrak{g}_\alpha\right),\]
where $\fg_\alpha = \left\lbrace x \in \fg: [h,x] = \alpha(h)x, \forall h \in \fh\right\rbrace$ for any $\alpha \in \Phi$. We refer to \cite[Chapter 8]{MR499562} for a detail analysis of this decomposition in general case. For this thesis,
most of the time, we stick with the following example
\begin{example}
    For the simple Lie algebra $\slnr$, one choice of  Cartan subalgebra $\mathfrak{h}$ is the set
    \[\fh = \left\lbrace H= \begin{bmatrix}
            a_1    & 0      & \ldots & 0      \\
            0      & a_2    & \ldots & 0      \\
            \vdots & \vdots & \ddots & \vdots \\
            0      & 0      & \cdots & a_n
        \end{bmatrix}, a_1 + a_2+ \ldots + a_n = 0\right\rbrace\]
    With respect to this Cartan subalgebra, we can define the linear function
    \[L_i \colon \fh \to \mathbb{R}, \quad H \mapsto  a_i\]
    Then the roots are given by $\alpha_{ij} :=L_i - L_j$ for distinct $i,j$. We have the root space decomposition for $\slnr$ as follows
    \[\fg = \fh \oplus \left(\bigoplus\fg_{\alpha_{ij}}\right) = \mathfrak{h}\oplus \left(\bigoplus \mathbb{R}E_{ij}\right)\]
    For the sake of brevity, we will denote $\alpha_{i,i+1}$ by $\alpha_i$ - these are called \textbf{simple roots}.
\end{example}
\subsection{Killing form}
To work with the geometry of the root system, we need the notion of Killing form
\begin{definition}
    Let $\mathfrak{g}$ be any Lie algebra and $x,y \in \mathfrak{g}$. We define the map
    \begin{align*}
        \mathrm{ad}(x) \colon \mathfrak{g} & \to \mathfrak{g}                  \\
        y                                  & \mapsto \mathrm{ad}(x)(y):= [x,y]
    \end{align*}
\end{definition}
The bilinear map $\kappa(x,y): = \mathrm{Tr}(\mathrm{ad}(x)\circ \mathrm{ad}(y))$ is called \textbf{Killing form}. If there is no confusion, we write $(x,y):=\kappa(x,y)$.
When $\mathfrak{g}$ is semisimple, the Killing form has many nice properties, as recorded in the following theorem
\begin{theorem}
    A Lie algebra $\mathfrak{g}$ is semi-simple if and only if the Killing form is non-degenerate.
\end{theorem}
It can also be showed that, given the root space decomposition as in section \ref{root-space-decomp}, if $\alpha, \beta \in \Delta$ satisfy $\alpha+\beta \ne 0$, then
the space $\mathfrak{g}_\alpha$ is orthogonal to $\mathfrak{g}_\beta$ with respect to the Killing form. A corollary of this is
\begin{theorem}\label{non-degenerate}
    Given a semisimple Lie algebra $\mathfrak{g}$ and a Cartan subalgebra $\mathfrak{h} \subset \mathfrak{g}$. Then the restriction of the Killing form to $\mathfrak{h}$ is non-degenerate.
\end{theorem}
By theorem \ref{non-degenerate}, there exists a bijective linear map
\begin{align*}
    f \colon \mathfrak{h} & \to \mathfrak{h}^\ast \\
    x                     & \mapsto f_x
\end{align*}
where $f_x(y):=\kappa(x,y)$ for all $y \in \mathfrak{h}$. We may thus define a bilinear map
\begin{align*}
    (\cdot,\cdot) \colon \mathfrak{h}^\ast \times \mathfrak{h}^\ast & \to \mathbb{R}      \\
    (f_x,f_y)                                                       & \mapsto \kappa(x,y)
\end{align*}
A more thorough treatment of the Killing form can be found in \cite[Chapter 5,8]{MR499562} .
\subsection{Co-roots}
In previous section, we showed that the Killing form gives rise to an inner product over
the dual space $\mathfrak{h}^* = \Hom(\mathfrak{h}, \mathbb{R})$. For each root $\alpha \in \mathfrak{h}^*$, we define the
corresponding co-root $\alpha^\vee$ to be the linear functional over $\mathfrak{h}^*$ such that
\[\left\langle \beta, \alpha^\vee\right\rangle:= \dfrac{2(\beta,\alpha)}{(\alpha,\alpha)}\]
On the other hand, since $\alpha^\vee$ is the dual space of $\mathfrak{h}^*$, it can be regarded
as an element $h_\alpha$ of $\mathfrak{h}$. So we have
\[\beta(h_\alpha)=\left\langle \beta, \alpha^\vee\right\rangle:= \dfrac{2(\beta,\alpha)}{(\alpha,\alpha)} \]
\begin{example}\label{simple root}
    We compute explicitly the fundamental coroot $\alpha_i^\vee$ of the fundamental root $\alpha_i$ for the Lie algebra
    $\slnr$. In previous section, we defined the fundamental roots to be $\alpha_{ij} = L_i-L_j$. It can be
    verified directly that $\alpha_{ij}^\vee = E_i-E_{j}$ as
    \[\left\langle \alpha_{ij}, \alpha_{ij}^\vee \right\rangle = (L_i-L_j)(E_i-E_j)=2\]
\end{example}
\subsection{Weights and co-weights}\label{weight-algebra-level}
Similarly, we can also define the $\lambda_i$ such that
\[\left\langle \lambda_i,\alpha_j^\vee\right\rangle = \delta_{ij} = \begin{cases}
        1, i =j \\
        0, i \ne j
    \end{cases}\]
Explicitly, it can be checked that $\lambda_i = L_1+L_2+\ldots+L_i$. In the same manner we define the co-weight $\lambda_i^\vee$ such that
\[\left\langle\alpha_i,\lambda_j^\vee \right\rangle= \delta_{ij} = \begin{cases}
        1, i =j \\
        0, i \ne j
    \end{cases}\]
From these definitions, we can see that $\widehat{\Delta } = \{ \lambda_i, i =1,2,\ldots,n-1\}$ is a basis for the the space $\mathfrak{h}^\ast$. 
We call $\widehat{\Delta}$ the set of \textbf{fundamental weights} . Similarly
the set $\widehat{\Delta }^\vee = \{ \lambda_i^\vee, i =1,2,\ldots,n-1\}$ is a basis for the Cartan subalgebra $\mathfrak{h}$.
\subsection{Roots at group level}
We want to understand how the roots behave at group level. The analog for the Cartan subalgebra is the maximal torus
\[T = \left\lbrace t= \begin{bmatrix}
        a_1    & 0      & \ldots & 0      \\
        0      & a_2    & \ldots & 0      \\
        \vdots & \vdots & \ddots & \vdots \\
        0      & 0      & \cdots & a_n
    \end{bmatrix} :  \prod_{i=1}^na_i =1\right\rbrace, \label{torus}\]
Then $T$ acts on $\fg$ by conjugation. Explicitly, we can check that
\[\Ad(t)(E_{ij}) = t_it_j^{-1}E_{ij}\]
Therefore, at the group level, the character $\alpha_{ij}(\text{diag}(t_1,\ldots,t_n))=t_it_j^{-1}$
is a root whenever $i \ne j$. We define the set of  \textbf{simple roots} as
\[\Delta = \left\lbrace \alpha_i\mid i =1,\ldots,n \right\rbrace\]
where
\[\alpha_i \colon T \mapsto \mathbb{R}, t \mapsto \dfrac{t_i}{t_{i+1}}.\]

We can decompose the set of roots $\Phi = \left\lbrace \alpha_{ij}, i \ne j\right\rbrace$ into disjoint subsets, namely
\[\Phi = \Phi_+ \coprod \Phi_{-}\]
where the positive set of roots $\Phi_+$  comprises $\alpha_{ij}$ for $i<j$ and the remaining roots are in the negative set of roots $\Phi_{-}$. We have the following lemma
\begin{lemma}\label{linear-comb-of-roots}
    Each $\alpha \in \Phi$ can be written uniquely as a linear combination
    \[\alpha = m_1\alpha_1+\ldots+m_{d}\alpha_{d}\]
    with all $m_i \in \mathbb{Z}_{\ge 0}$ or $m_i \in \mathbb{Z}_{\le 0}$. If $\alpha \in \Phi_+$ then all $m_i \ge 0$, otherwise $m_i \le 0$ for all $i$.
\end{lemma}
\subsection{Weyl group}
We only define the Weyl group explicitly for the group $\SLn$ or $\GLn$. It is a fact that the Weyl groups for
these two Lie groups are the same and equal to $W = S_n$ - the permutation group of $n$ letters.  We recall some basis
observation about this group
\begin{enumerate}
    \item Every $\sigma \in W$ can be written (non-uniquely) as a product of $s_{i_1} \cdots s_{i_k}$ for some integer $k$. Such a sequence is said to have length $k.$ If $k$ is the minimum, over all such writings, it is called the length of $\sigma$ and written $\ell(\sigma)$. Any expression of length $\ell(\sigma)$ for $\sigma$ is called a reduced expression.

    \item The group $S_n$ is generated by $S$ subject to the following two types of relations:
          \begin{itemize}
              \item (Reflection) $s_i^2=1$ for $i \in I$.
              \item (Braid relations) $s_i \, s_{i+1} \, s_i = s_{i+1}s_i s_{i+1}$ for $i = 1, \ldots, n-2$ and $s_i s_j = s_j s_i$ for $|j -i |\geq 2$.
          \end{itemize}
\end{enumerate}
Note that $W$ acts on $\Hom(T, \mathbb{R}^*)$ in the natural way: $w . \varphi(h) = \varphi(w^{-1} h)$. More explicitly,  $\sigma$ sends $\alpha:= \alpha_{ij}$ to $\alpha_{\sigma(i), \sigma(j)}$. Hence we find that
\[w_i \alpha_j = \begin{cases} - \alpha_i          & \mbox{if } i=j           \\
              \alpha_j            & \mbox{ if } |j - i | > 1 \\
              \alpha_i + \alpha_j & \mbox{ if } |j-i|=1\end{cases} \]


\subsection{Weights}\label{Weyl-vector-def}
Similarly, at Lie group level, we also have the notion of \textbf{fundamental weights}. For each $i \in I$, we define
a character $\lambda_i \in \Hom(T,\mathbb{R}^*)$ where $T$ is the torus defined in \ref{torus}
\[\lambda_i \colon T \to \mathbb{R}^*, \quad\lambda_i(t) = a_1\ldots a_i\]
We have the following
\begin{lemma}\label{linear-comb-of-weights}
    We can write
    \[\lambda_i := r_1\alpha_1 + r_2\alpha_2+\ldots + r_d\alpha_d\]
    where $r_i$'s are rational number such that $r_i \ge 0$.
\end{lemma}
This coefficients $r_i$'s is determined by inverting the Cartan matrix of $\slnr$. We refer to \cite{Wei_2017}  for an explicit formula for $r_i's$.
\begin{example}
    When $n=3$, we have the following relations
    \[\lambda_1 = \dfrac{2}{3}\alpha_1+\dfrac{1}{3}\alpha_2, \quad \lambda_2 = \dfrac{1}{3}\alpha_1+\dfrac{2}{3}\alpha_2\]
    We refer to the figure \ref{A_2-rootsystem} for an illustration.
\end{example}
\begin{figure}[hbt]
    \centering
    \begin{tikzpicture}
        \begin{rootSystem}[weight length=2cm]{A}
            \roots
            \simpleroots
            \node [above] at \Root {1}{0} {\(\alpha_1\)};
            \node [right] at \Root {0}{1} {\(\alpha_2\)};
            \fundamentalweights
            \node [right] at \weight {1}{0} {\(\lambda_1\)};
            \node [right] at \weight {0}{1} {\(\lambda_2\)};
            \draw[->] (0,0)--\weight{1}{0};
            \draw[->] (0,0)--\weight{0}{1};
            \draw [red, arrows = {-Stealth}] (0,0)--\Root {1}{0};
            \draw [red, arrows = {-Stealth}] (0,0)--\Root {0}{1};
        \end{rootSystem}
    \end{tikzpicture}
    \caption{Roots and weights for the Lie group $\text{SL}_3(\mathbb{R})$}
    \label{A_2-rootsystem}
\end{figure}


\begin{definition}
    A weight $\lambda$ is called \textbf{dominant} if it satisfies $\left\langle \lambda,\alpha^{\vee} \right\rangle \in \mathbb{Z}_{\ge 0}$ for all $\alpha$.
\end{definition}
Clearly by lemma \ref{linear-comb-of-roots}, the weight $\lambda$ is dominant if and only if $\left\langle\lambda,\alpha_i^\vee\right\rangle$ for all
fundamental root $\alpha_i$. It is also clear that the set of dominant weights is given by
\[\Lambda^+ := \left\lbrace c_1\lambda_1+\ldots+c_d\lambda_d \mid c_i \in \mathbb{Z}_{\ge 0}\right\rbrace\]
The set of dominant weights is denoted $\Lambda^+$. A weight $\lambda = \sum n_i \lambda_i$ is called strongly dominant if $n_i > 0$ for all $i$. One important example is the minimal strongly dominant weight given by
\[
    \rho = \sum \lambda_i
\]
This is called \textbf{Weyl vector} and is characterized in several ways:

\begin{enumerate}
    \item $\left\langle\rho,\alpha_i^\vee\right\rangle = 1$ for all $i$.
    \item
          \[
              \rho = \frac{1}{2} \sum_{\alpha \in \Phi^+} \alpha
          \]
\end{enumerate}

To prove the last equation we use the action of the Weyl group $W$. Let $\mu = \frac{1}{2} \sum \alpha$. Apply the simple reflection $s_i$ given by
\[
    s_i(x) = x - \left\langle x, \alpha_i^\vee\right\rangle \alpha_i
\]
We know that $s_i$ sends $\alpha_i$ to $-\alpha_i$ and permutes the other positive roots. So:
\[
    s_i(\mu) = \mu -  \left\langle x, \alpha_i^\vee\right\rangle  \alpha_i
\]
Therefore, $(\mu, \alpha_i) = \mu(h_i) = 1$ for all $i$. So, $\mu = \rho$.

We of course also have an action of $W$ on the weights.  For example, one can verify that
\[ \begin{array}{lcr} s_i(\lambda_i) = \lambda_i - \alpha_i & \text{ and } & s_j(\lambda_i) = \lambda_i \mbox{ for } i \neq j \end{array}.\]
Unlike Lemma \ref{linear-comb-of-weights}, if we try to express the fundamental weights in terms of the fundamental roots, we do not always
get positive coefficients. However, it is true that all the coefficients must be integer. In particular, we have
\[\alpha_j = \sum_{n_j}n_j\lambda_j, \quad n_j \in \mathbb{Z}.\]
To put it another way, the root lattice $$\mathbb{Z}\Delta = \{m_1\alpha_1+\ldots+m_{d}\alpha_{d}, m_i \in \mathbb{Z} \}$$ is contained inside the weight lattice $\mathbb{Z}\Pi = \{n_1\lambda_1+\ldots + n_d\lambda_d, n_i \in \mathbb{Z}\}$.



\subsection{Cartan matrix}
We fix a set of simple roots $\Delta = \left\lbrace \alpha_1,\ldots,\alpha_d \right\rbrace$ is defined to be the matrix
\[A = \begin{bmatrix} \left\langle \alpha_i,\alpha_j^\vee  \right\rangle
    \end{bmatrix}\]
If we let $a_{ij}=  \left\langle \alpha_i,\alpha_j^\vee\right\rangle$ then the Cartan matrix has the following simple properties:
\begin{lemma}
    \hfill
    \begin{itemize}
        \item For any $i$, we have $a_{ii}=2$.
        \item For any $i \ne j$, $a_{ij}$ is a non-positive integer, i.e. $a_{ij} \in \mathbb{Z}_{\le 0}$.
    \end{itemize}
\end{lemma}
We give an explicit example for $\slnr$, which has the root system $A_n$. The corresponding Cartan matrix is
\[A_n = \begin{bmatrix}
        2      & -1     & 0      & 0      & \ldots & 0      \\
        -1     & 2      & -1     & 0      & \ldots & 0      \\
        \vdots & \vdots & \vdots & \vdots & \ddots & \vdots \\
        0      & 0      & \ldots & -1     & 2      & -1     \\
        0      & 0      & \ldots & 0      & -1     & 2      \\
    \end{bmatrix}\]
The Cartan matrix provides a linear combination presentation of the fundamental roots $\alpha_j$'s in terms of fundamental weights $\lambda_i$,
as stated in the following lemma
\begin{lemma}\label{weight-root-comb}
    Fix a number $n$ and consider the set of simple roots $\alpha_j$ as well as the set of weights $\lambda_i$ of $\SLn$, we have the following relations
    \[\alpha_i = \begin{cases}
            2\lambda_1-\lambda_2,                    & \mbox{ if } i = 1 \\
            -\lambda_{i-1}+2\lambda_i-\lambda_{i+1}, & \mbox{ if } 1<i<n \\
            -\lambda_{n-1}+2\lambda_n,               & \mbox{ if } i =n
        \end{cases}.\]
\end{lemma}
\begin{proof}
    Note that we have
    \[\left\langle \alpha_i,\alpha_j^{\vee}\right\rangle = \begin{cases}
            2, \mbox{ if } i =j     \\
            -1, \mbox{ if } |i-j|=1 \\
            0, \mbox{ otherwise }
        \end{cases}. \]
    So if we let $\alpha_j = \sum_{i=1}^n a_{ij}\lambda_i$ and compute $\left\langle \alpha_i,\alpha_j^{\vee}\right\rangle$, the result follows immediately.
\end{proof}

\section{Parabolic subgroups and parabolic subalgebras}
We shall provide two equivalent viewpoints on parabolic subgroups. They will play different roles in defining
different notions of semi-stability in the next chapter.
\subsection{Parabolic subgroups I: An explicit description }\label{parabolic I}
For our purposes, it is enough to define the standard parabolic subgroups.  There exists a bijection between each parabolic subgroup of $\SLn$
and each partition of $n$. We can therefore define the parabolic subgroup explicitly as follows:
\begin{definition}
    The standard parabolic subgroup associated to the partition $n = n_1 + n_2 + \cdots + n_r$ is denoted $P_{n_1,\ldots,n_r}$ and is defined to be the group of all matrices of the form
    \[
        \begin{bmatrix}
            M_{n_1} & 0       & \ldots & 0       \\
            0       & M_{n_2} & \ldots & 0       \\
            \vdots  & \vdots  & \ddots & \vdots  \\
            0       & 0       & \ldots & M_{n_k}
        \end{bmatrix} ,
    \]
    where $M_{n_i} \in \mathrm{GL}(n_i, \mathbb{R})$ for $1 \leq i \leq r$. The integer $r-1$ is called the rank of the parabolic subgroup $P_{n_1,\ldots,n_r}$.
    We denote the collection of such subgroups \textbf{ParSt}.
\end{definition}
In this form, we can choose the standard basis  $\{e_i\}$ of $\mathbb{R}^n$. Let
\[d_i = n_1+n_2+\ldots+n_i\]
Then it is clearly that any element in $P_{n_1,\ldots,n_r}$ stabilizes the chain of vector spaces
\[0 \subset \mathbb{R}^{d_1} \subset \mathbb{R}^{d_2}\subset \ldots \subset \mathbb{R}^{d_k} = \mathbb{R}^n\]
Here $R^k$ is generated by the linearly independent subset $\{e_1,\ldots, e_k\}$ of the standard basis.
We call such chain of vector spaces a \textbf{flag} of type $(n_1,\ldots,n_k)$.

\begin{definition}
    The maximal standard parabolic subgroups in $\text{GL}_n(k)$ corresponds to the
    stabilizer of the flag of type $\rho_i =(i,n-i)$, where $i = 1,\ldots,n-1$ of $n$. We will
    further denote $Q_i = P_{\rho_i}$ and $\textbf{MaxParSt}$ the collection of such maximal parabolic subgroups.
\end{definition}

\begin{example}
    Below we list all the standard parabolic subgroup in $\text{GL}_3(\mathbb{R})$.
    For $\text{GL}_3(\mathbb{R})$, there are three standard parabolic subgroups corresponding to
    three partitions of $3$, namely
    \[ 3 = 1+ 1+ 1, \quad 3 =1+2 , \quad 3 = 2+1\]
    For a partition $(r_1,\ldots,r_{s+1})$, we denote $P_{(r_1,\ldots,r_{s+1})}$ the corresponding parabolic subgroups. Thus, we have
    \begin{align*}
         & P_{1,1,1} = \left\lbrace \begin{bmatrix}
                                        \ast & \ast & \ast \\
                                        0    & \ast & \ast \\
                                        0    & 0    & \ast
                                    \end{bmatrix}\right\rbrace, \quad P_{1,2} =\left\lbrace \begin{bmatrix}
                                                                                                \ast & \ast & \ast \\
                                                                                                0    & \ast & \ast \\
                                                                                                0    & \ast & \ast
                                                                                            \end{bmatrix}\right\rbrace \\
         & P_{2,1} = \left\lbrace \begin{bmatrix}
                                      \ast & \ast & \ast \\
                                      \ast & \ast & \ast \\
                                      0    & 0    & \ast
                                  \end{bmatrix}\right\rbrace
    \end{align*}
    Clearly $\textbf{MaxParSt} = \left\lbrace P_{2,1}, P_{1,2}\right\rbrace$.
\end{example}
\subsection{Parabolic subgroups II: Using BN-pairs}
There is another way to look at the parabolic subgroups, and it is related more closesly to the notion of roots, using Weyl group.
\begin{definition}[BN-pairs]
    A \textbf{BN-pairs} is a 4-tuple $(G,B,N,R)$ where $G$ is a group generated by subgroups $B$ and $N$.
    The subgroup $H = B \cap N$, $R$ is a finite set of involutions which generate the \textit{Weyl group} $W= N/H$. Moreover,
    the following theorem holds
    \begin{itemize}
        \item If $ r \in R$ and $w \in W$, then $rBw \subset BwB \cup BrwB$.
        \item If $r \in R, rBr \ne B$.
    \end{itemize}
\end{definition}
For our purpose, it is enough to concentrate on the following example
\begin{example}
    Let $G = \GLn$, then the sets $B,N,R$ are given explicitly as follows
    \begin{itemize}
        \item $B = \text{ upper triangular matrices}$
        \item $N =$  monomial matrices, namely, matrices that have exactly one non-zero entry in each row and column.
        \item From the above, it is clear that $H = B \cap N$ is the diagonal group, and this group is normal in $N$.
        \item It can be shown that $W = N/H \cong S_n$, and we can choose $R = \left\lbrace (i,i-1)\right\rbrace $ - the set of transpositions.
    \end{itemize}
\end{example}
Let $J \subset R$, we define $W_J$ to be the subgroup of $W$ generated by the involutions
$r \in J$. We call it \textbf{standard parabolic subgroup} of $W$. Set $P_J = BW_JB$ as in the notation of
BN-pairs. We have the following theorem
\begin{theorem}
    \hfill
    \begin{itemize}
        \item $P_J$ is a subgroup if $G$. In particular, we have
              \[G = BWB,\]
              which is called \textbf{Bruhat decomposition} of $G$.
        \item If $P_I=P_J$ then we have $I=J$.
        \item All subgroups of $G$ containing $B$ arises in this way. \textcolor{red}{Add proofs?}
    \end{itemize}
\end{theorem}
The above theorem leads to the following definition of parabolic subgroups
\begin{definition}[Parabolic subgroups]
    Using the same notation in the previous theorem, we call the subgroups $P_J$ with $J \subset R$  \textit{standard parabolic subgroups} of group G.
\end{definition}
We would like to explicitly describe the parabolic subgroup for $\SLn$.
\begin{example}
    As introduced in previous section, the set $\Phi = \left\lbrace \alpha_{ij}\right\rbrace $ forms a
    root system for $\SLn$. The Borel subgroup is just $B \cap \SLn$, the group of upper triangular matrices with determinant $1$.
    We also consider the set $N$ of monomial matrices  with determinant $1$. Then it is easy to check that
    $N/H$ is the set of all permutation matrices. That is, $N/H$ is generated by the
    matrices of the form \textcolor{red}{see Cambridge}
    \[r_i := \begin{bmatrix}
            I_{i-1} &   &   &           \\
                    & 0 & 1 &           \\
                    & 1 & 0 &           \\
                    &   &   & I_{n-i-1}
        \end{bmatrix}\]
    Via the identification $s_i \mapsto (i,i+1)$, we identify  the Weyl group $W = N/H \cong S_n$. So
    $R = \left\lbrace (i,i+1)| i = 1,2,\ldots,n\right\rbrace $. We consider the set
    $I = R \setminus \{r_i\}$ for some $i<n$. The associated parabolic subgroup of
    $W$ is
    \[W_I = \left\langle r_1,r_2,\ldots r_{i-1},r_{i+1},\ldots, r_{n-1} \right\rangle \cong
        S_i \times S_{n-i}\]
    The corresponding parabolic subgroup is
    \[P_{r_i} := P_I = \left\lbrace \begin{bmatrix}
            A & \ast \\
              & B
        \end{bmatrix} \in \SLn: A \in \text{GL}_i, B \in \text{GL}_{n-i}\right\rbrace \]
\end{example}

\subsection{Langlands decomposition}
We fix a partition of $n$ as
\[n=n_1+n_2+\ldots+n_k\]
and consider the parabolic subgroup of this type, i.e. the subgroup
\[P_{n_1,\ldots, n_k} = \left\lbrace \begin{bmatrix}
        \mathfrak{m}_1 & \ast           & \ldots & \ast           \\
        0              & \mathfrak{m}_2 & \ldots & \ast           \\
        \vdots         & \vdots         & \ddots & \vdots         \\
        0              & 0              & \ldots & \mathfrak{m}_k
    \end{bmatrix} \right\rbrace\]
where $\mathfrak{m_i}$ is invertible of size $n_i \times n_i$.

This group can be factored as
\[P_{n_1,\ldots, n_k} =M_{n_1,\ldots, n_k}\ltimes N_{n_1,\ldots, n_k}\]
where
\[N_{n_1,\ldots, n_k} = \left\lbrace \begin{bmatrix}
        I_1    & \ast   & \ldots & \ast   \\
        0      & I_2    & \ldots & \ast   \\
        \vdots & \vdots & \ddots & \vdots \\
        0      & 0      & \ldots & I_k
    \end{bmatrix} \right\rbrace \quad \left(I_k \text{ is the $n_k\times n_k$ identity matrix}\right)\]
and
\[M_{n_1,\ldots, n_k} = \left\lbrace \begin{bmatrix}
        M_1    & 0      & \ldots & 0      \\
        0      & M_2    & \ldots & 0      \\
        \vdots & \vdots & \ddots & \vdots \\
        0      & 0      & \ldots & M_k
    \end{bmatrix} \right\rbrace\]

The subgroup $M_{n_1,\ldots, n_k}$ is called \textbf{Levi component}. We can further factor this subgroup as
\[M_{n_1,\ldots, n_k} = M^1_{n_1,\ldots, n_k} \cdot A_{n_1,\ldots, n_k}\]
with $A_{n_1,\ldots, n_k}$ plays the role of the connected center of $M_{n_1,\ldots, n_k}$:
\[A_{n_1,\ldots, n_k} = \left\lbrace \begin{bmatrix}
        t_1I_1 & 0      & \ldots & 0      \\
        0      & t_2I_k & \ldots & 0      \\
        \vdots & \vdots & \ddots & \vdots \\
        0      & 0      & \ldots & t_kI_k
    \end{bmatrix} : t_i > 0\right\rbrace \]
and
\[M^1_{n_1,\ldots, n_k} = \left\lbrace \begin{bmatrix}
        \mathfrak{m}'_1 & 0               & \ldots & 0               \\
        0               & \mathfrak{m}'_2 & \ldots & 0               \\
        \vdots          & \vdots          & \ddots & \vdots          \\
        0               & 0               & \ldots & \mathfrak{m}'_k
    \end{bmatrix} \right\rbrace,\]
where $\det(\mathfrak{m}'_i) = \pm 1$.
\begin{definition}
    For a given parabolic subgroup $P$, the factorization
    \[P = M^1_P \times A_P \times N_P\]
    as above is called \textbf{Langlands decomposition}.
\end{definition}
\subsection{Iwasawa decomposition and $P$-horospherical decomposition}
We introduce two important ways to decompose a matrix in $\GLn$ into simpler parts. This will be frequently used
in the remaining of this thesis.
\begin{prop}
    Let \begin{align*}
        K=\On, \quad A = \left\lbrace\begin{bmatrix}
                                         a_1    & 0      & \ldots & 0      \\
                                         0      & a_2    & \ldots & 0      \\
                                         \vdots & \vdots & \vdots & \vdots \\
                                         0      & 0      & \ldots & a_n
                                     \end{bmatrix}: a_i >0\right\rbrace, \quad
        U = \left\lbrace  \begin{bmatrix}
                              1      & \ast   & \ldots & \ast   \\
                              0      & 1      & \ldots & \ast   \\
                              \vdots & \vdots & \vdots & \vdots \\
                              0      & 0      & \ldots & 1\end{bmatrix}
        \right\rbrace
    \end{align*}
    Then the natural product map \[p \colon K \times A \times U \to \GLn, \quad p(k,a,n)=kan\]
    is an isomorphism.
\end{prop}
Thus, the following decomposition makes sense
\begin{definition}
    The decomposition of $g = kan$ for $k \in K, a \in A$ and $n \in N$ is called \textbf{Iwasawa} decomposition.
\end{definition}
Using the Langlands decomposition as well as the Iwasawa decomposition $G= ANK = BK$, we observation
the decomposition
\[G = KP = KM_PA_PN_P \cong KM_P\times A_P\times N_P\]
As a consequence, we get
\[X_G  = K\backslash G = X_{M_P} \times A_P \times N_P\]
for $X_{M_P} := (K\cap M_P) \backslash M_{P}$.

This is \textbf{$P$-horospherical} \label{P-horospherical} decomposition of the lattice space $X_G$. In this way,
we can identify  $x \in X_G$ with the triple $(m_P(x), a_P(x),n_P(x))$.
\begin{example}
    When $n=2$, the only nontrivial standard parabolic subgroup is the Borel subgroup $P=B$,
    so the $P$-horospherical decomposition is
    \[X_G = K\backslash G = A_B \times N_B\]
    which is in fact just the Iwasawa decomposition.
\end{example}
Furthermore, if $Q\supset P$ is also a parabolic subgroup, it gives rise to a parabolic subgroup $\ast P := P \cap M_Q$ of $M_Q$.
The space $X_{M_Q}$ itself has the decomposition
\[X_{M_Q} = X_{\ast P} \times A(Q)_{\ast P} \times N(Q)_{\ast P}\]
where $$ X_{\ast P} = X_{M_P}, \quad A(Q)_{\ast P}A_Q=A_P, \quad N(Q)_{\ast P}N_Q=N_P.$$ This is called \textbf{relative $P$}-horospherical decomposition with respect to $Q$.
\begin{example}

\end{example}
\subsection{Parabolic sets and parabolic subalgebras}\label{parabolic-sets-algebra}
\begin{definition}
    Given a root system $\Phi$ . A \textbf{parabolic subset} $\Phi(P)$ is a subset of $\Phi$  that
    satisfies the following conditions:
    \begin{enumerate}
        \item $\Phi(P) \cup -\Phi(P) = \Phi$.
        \item It is closed, in the sense that, for any two root $\alpha,\beta \in \Phi(P)$ such that $\alpha+\beta$ is a root, then $\alpha+\beta \in \Phi(P)$.
    \end{enumerate}
\end{definition}
The parabolic set parametrizes the corresponding parabolic subalgebra with a choice of base $\Delta$ of the root system $\Phi$ as given in the following theorem
\begin{theorem}
    Given a semisimple Lie algebra $\mathfrak{g}$ with the root system $\Phi$ and a base $\Delta$. We denote $\Phi^+$ the set of positive roots with respect to this base. There exists a correspondence between
    parabolic subset  $\Phi(P)$ of $\Phi$ and subalgebras of $\mathfrak{g}$ containing the
    Borel subalgebra $\mathfrak{b}$. The correspondence is given by
    \[ \Phi(P) \longleftrightarrow \mathfrak{p}:= \mathfrak{b}\oplus \bigoplus_{
            \alpha \in \Phi(P) \setminus \Phi^+} g_{\alpha}\] 
\end{theorem}
\begin{proof}
    We refer to \cite[Chapter VI, \S 1.7, Proposition 20]{MR2109105} for a proof of this fact.
\end{proof}
\begin{example}
    We consider the case $\mathfrak{g} = \mathfrak{sl}_3(\mathbb{R})$. Let's denote
    $\Delta = \left\lbrace \alpha,\beta\right\rbrace$ a base for the root system of $\mathfrak{g}$. The set of positive roots is
    \[\Phi^+= \{\alpha, \beta, \alpha+\beta\}\] 
    There are 4 parabolic sets, corresponding to 4 parabolic subalgebras given as follows
    \begin{align*}
        \Phi(P) = \Phi^+       & \longleftrightarrow \mathfrak{p} = \mathfrak{b}                               \\
        \Phi(P) = \Phi^+ \cup\{-\beta\}  & \longleftrightarrow \mathfrak{p} = \mathfrak{b} \oplus \mathfrak{g}_{-\beta} \\
        \Phi(P)= \Phi^+ \cup\{-\alpha\} & \longleftrightarrow \mathfrak{p} = \mathfrak{b} \oplus \mathfrak{g}_{-\alpha}  \\
        \Phi(P)= \Phi^+ \cup -\Phi^+   & \longleftrightarrow \mathfrak{p} = \mathfrak{g}
    \end{align*}
\end{example}

\subsection{On the function $H_P$}
Recall that for the Lie group $\SLn$, we attach to it a root system $\Phi = \left\lbrace \alpha_{i,j}\right\rbrace$ with
\[\Delta = \left\lbrace \alpha_i\mid i \in I \right\rbrace, \quad I = \{1,2,\ldots,n-1\}\]
as the set of fundamental roots. It is a fact that the Weyl group $W$ satisfies
\[W = \left\langle r_{\alpha_i}: i \in I \right\rangle\]
For the sake of brevity, we denote $r_{\alpha_i}:=r_i$. For a standard parabolic subgroup $P = P_J$ where $J \subset I$,
we define the following subset of a fixed Cartan subalgebra $\fh$ of $\slnr$
\begin{align*}
    \mathfrak{a}(P) & :=\text{span}\{ \alpha^\vee_i: i \in J\}                                       \\
    \mathfrak{a}_P  & := \{H \in \fh: \left\langle \alpha_i,H\right\rangle=0 \mbox{ for } i \in J \}
\end{align*}
By the definition of weights, i.e. $\left\langle \lambda_i,\alpha^\vee_j \right\rangle = \delta_{ij}$, we can see 
that the subspace $ \mathfrak{a}_P$ of $\fh$ contains the subspace
\[\text{span}\left\lbrace \lambda^\vee_i: i \notin J\right\rbrace \]
It is also clear that the dimension of $\mathfrak{h}(J)$ is $|J|$ as the set $\{ \alpha^\vee_i: i \in J\}  $
is a linearly independent set. Assume that
\[\sum_{i \notin J}c_i\lambda_i^\vee =0\]
Pairing both sides with $\alpha_i$, we deduce that $c_i=0$ for all $i \notin J$. In particular, the set $\left\lbrace \lambda^\vee_i: i \notin J\right\rbrace $ contains linearly independent vectors and thus spans
a subspace of $\mathfrak{a}_P$ of dimension $|I|-|J|$. Therefore
\[ \mathfrak{a}_P=\text{span}\left\lbrace \lambda^\vee_i: i \notin J\right\rbrace\]
To put it another way, we have a direct sum decomposition
\[\mathfrak{h} = \mathfrak{a}(P) \oplus \mathfrak{a}_P\]
We also write $\lambda^{\vee}_{j,J}$ for the basis of $\fh(J)$ containing the fundamental
weight, namely $\left\langle \alpha_k,\lambda^{\vee}_{j,J}\right\rangle = \delta_{kj}$ for $j,k \in J$. We have
the following easy lemma
\begin{lemma}\label{coeff-H(J)}
    Let $H \in \fg$ be arbitrary. For $J \subsetneq I$ we have the decomposition
    \[H = H(J)+H_J,\]
    where $H(J) \in \mathfrak{a}(P) $ and $H_J \in  \mathfrak{a}_P$. If $H = \sum_{i=1}^n p_i\lambda^\vee_i$ then
    \[H(J) = \sum_{i \in J} p_i \lambda^\vee_{i,J}\]
\end{lemma}
\begin{proof}
    Let $H(J) = \sum_{i \in J} c_i \lambda^\vee_{i,J}$, then for $i \in J$
    \[p_j = \left\langle \alpha_i,H \right\rangle =\left\langle \alpha_i,H_J+H(J) \right\rangle = \sum_{i \in J}\left\langle \alpha_i,  c_i \lambda^\vee_{i,J}\right\rangle=c_i \]
    Hence $H(J) = \sum_{i \in J} p_i \lambda^\vee_{i,J}$.
\end{proof}
We now define the function $H_P$ as follows:
\begin{definition}\label{H-function}
    Assume that for $x \in X$,
    we have a $P$-horospherical decomposition
    \[x = (m_P(x),a_P(x),n_P(x))\]
    Then we define
    \begin{align*}
        H_P \colon X & \to \mathfrak{a}_P                      \\
        x            & \mapsto H_P(x) = \log(a_P(x)),
    \end{align*}
    where $\log$ means we  take the logarithm of the diagonal entries of $a_P(x)$.
\end{definition}
This definition is well-defined since $\log(a_P(x))$ is an element of $\mathfrak{a}_P  $ by definition.
We prove the following lemma
\begin{lemma}\label{H_P-decomp}
    Let $P \subset Q$ be standard parabolic subgroups of $G$, and $x \in K \backslash G$. Then
    Then we have
    \[H_P(x) = H_Q(x)+ H_{\ast P}(\text{pr}_{M_Q}(x))\]
    where $\text{pr}_{M_Q}(x)$ stands for the natural projection of $x \in X$ on $M_Q/K \cap M_Q$.
\end{lemma}
\begin{proof}
    This follows immediately from the observation that
    \[A_P = A_QA(Q)_{\ast P}\]
    and the definition \ref{H-function} of the function $H_P$ for parabolic subgroup $P$.
\end{proof}
\subsection{Some additional notations}
We define some important set that will be used in the later part of this thesis. Given a standard parabolic subgroup $P=P_J$, let 
\[\Delta(P):= \{\alpha_i^: \ i\in J\} \quad \text{ and } \quad \Delta_P:= \{{\alpha_i}_{\mid \mathfrak{a}_P}: i \notin J\}\]
Similarly, we define
\[\widehat{\Delta}(P): = \{ {\lambda_i}_{\mid \mathfrak{a}(P)}: i \in J\}\quad \text{ and } \quad \widehat{\Delta}_P:= \{\lambda_i, i \notin J\}\]

\begin{example}\label{sl3-parabolic-root-weight}
    Let $G = \mathrm{SL}_3(\mathbb{R})$. We can choose a base $\Delta = \{\alpha_1,\alpha_2\}$ as before, thus $I = \{1,2\}$. Assume that $J= \{2\}$. Then 
    $P_1: = P_{I\setminus J}$. We have the following sets
    \[\Delta_{P_1} = \{{\alpha_1}_{\mid\mathfrak{a}_{P_1}}\}, \quad \widehat{\Delta}_{P_1}= \{{\lambda_1}\}\]
\end{example}
\begin{remark}
        Over $\mathfrak{h}(J)$ we can also define the notion of weight $\omega_i^J$ by the relations
        \[\left\langle  \omega_j^J,\alpha_i^\vee\right\rangle=\delta_{ij}\]
        for $j\in J$. It is known that we have the relation
        \[\omega^J_i = \sum_{c_i, i \in J}c_i\alpha_j,\]
        where $c_j \ge 0$. Thus we can naturally define $\omega^J_i$ on the Cartan subalgebra $\mathfrak{h}$ by letting 
        $\omega^J_i$ to be identically zero on $\mathfrak{h}_J$. For example, the weights in $\mathfrak{h}(J)$ is different from that of $\mathfrak{h}$. In example 
        \ref{sl3-parabolic-root-weight}, we have the identification
        \[{\lambda_1}_{|\mathfrak{a}(P)} = \dfrac{1}{2}\alpha_1\]
        But they extend differently over $\mathfrak{h}$.
\end{remark}
\subsection{The linear functionals $\rho_P$ and $\rho_Q^P$}
We define an analog for Weyl vector $\rho$, which is essential in studying the semi-stablility in the later of this thesis.
\begin{definition}
    Given a standard parabolic subgroup $P = P_J$ of $G=\SLn$ or $G=\GLn$, we define 
    \[\rho(P): = \sum_{\omega \in \widehat{\Delta}(P)}\omega \quad \text{ and } \quad \rho_P:= \rho- \rho(P),\]
    where $\rho$ is the Weyl vector in section \ref{Weyl-vector-def}.
\end{definition}
If we are given another standard parabolic subgroup $Q\subset P$ then
\begin{definition}
    \[\rho^P_Q:= \rho(P)- \rho(Q) = \rho_Q-\rho_P\]
\end{definition}
We have a more concrete description for $\rho_P$ as follows
\begin{lemma}\label{rhoP}
    Fix a standard parabolic subgroup $P$ and define $d_P(\alpha): = \left\langle \rho(P), \alpha^\vee\right\rangle$ for each $\alpha \in \Delta_P$. Then $d_P(\alpha)$ is non-positive and 
    \[\rho_P = \sum_{\alpha \in \Delta_P}(1-d_P(\alpha))\lambda_{\alpha}\]
\end{lemma}
\begin{proof}
    We write $\rho_P = \sum_{i \in I}c_i\lambda_i$ and evaluate at $\alpha_i$. Then
    \[c_i = \left\langle \rho_P,\alpha_i^\vee\right\rangle = \left\langle \rho,\alpha_i^\vee\right\rangle - \left\langle \rho(P),\alpha_i^\vee\right\rangle = \begin{cases}
        0, \text{ if } i \in J\\
        1-d_P(\alpha), \text{ otherwise } 
    \end{cases}\]
Since $\rho(P)$ is a non-negative linear combination of roots in $\Delta(P)$,  says 
\[\rho(P) = \sum_{i \in J}{m_i\alpha_i}\]
we have 
\[d_P(\alpha) = \sum_{i \in J} m_i\left\langle \alpha_i,\alpha^\vee\right\rangle\]
where $\alpha \in \Delta_P$. It is known in Lie theory that the angle between two simple roots are obtuse, hence $\left\langle \alpha_i,\alpha^\vee\right\rangle  \le 0$, which implies $d_P(\alpha)$ is non-positive
\end{proof}


