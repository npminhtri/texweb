\documentclass[12pt]{article} % \documentclass{} is the first command in any LaTeX code.  It is used to define what kind of document you are creating such as an article or a book, and begins the document preamble

\usepackage{amsmath} % \usepackage is a command that allows you to add functionality to your LaTeX code

\usepackage[papersize={216mm,330mm},tmargin=20mm,bmargin=20mm,lmargin=20mm,rmargin=20mm]{geometry}
\usepackage[english]{babel}
\usepackage[utf8]{inputenc}
\usepackage{amsmath,amssymb,mathabx,amsthm}%\for eqref
\usepackage{mathabx}
\usepackage{lscape}
\usepackage{graphicx}
\usepackage{tikz}\usetikzlibrary{arrows.meta,calc} %library tikz
\usepackage{subcaption}


\usepackage{pgfplots}
\pgfplotsset{compat=1.15}
\usepackage{mathrsfs}
\usetikzlibrary{arrows}

\usepackage{color,soul} %package for highlining
\usepackage[colorinlistoftodos]{todonotes}
\usepackage{fancyhdr}
\usepackage{hyperref} %creat hyperlink
\hypersetup{
    colorlinks=true,
    linkcolor=blue,
    filecolor=magenta,      
    urlcolor=cyan,
    pdftitle={Overleaf Example},
    pdfpagemode=FullScreen,
    } %set up a hyperlink to be in blue 
\newtheorem{definition}{Definition}[section]
\newtheorem{theorem}[definition]{Theorem}
\newtheorem{prop}[definition]{Proposition}
\newtheorem{lemma}[definition]{Lemma}
\newtheorem{example}[definition]{Example}

\newtheorem*{remark}{Remark}

\pagestyle{fancy}
\cfoot{\thepage} % this is for the page numbering
\setlength\parindent{0pt} % noindent for the whole document.
\renewcommand{\baselinestretch}{1.1} % increase the distance between line.

\DeclareMathOperator{\SLn}{\text{SL}_n(\mathbb{R})}
\DeclareMathOperator{\GLn}{\text{GL}_n(\mathbb{R})}
\DeclareMathOperator{\slnz}{SL_n(\mathbb{Z})}
\DeclareMathOperator{\glnz}{GL_n(\mathbb{Z})}
\DeclareMathOperator{\vol}{vol}
\DeclareMathOperator{\Hom}{Hom}
\DeclareMathOperator{\fg}{\mathfrak{g}}
\DeclareMathOperator{\fh}{\mathfrak{h}}
\DeclareMathOperator{\SO}{SO_n(\mathbb{R})}
\DeclareMathOperator{\slnr}{\mathfrak{sl}_n(\mathbb{R})}
\DeclareMathOperator{\Ad}{\text{Ad}}
\newcommand{\tpoint}[1]{\subsection{#1}}

\newcommand{\spoint}{\subsection{}}



\title{CHAPTER II :ROOTS AND WEIGHTS FOR $\SLn$} % Sets article title
\date{\today} % Sets date for date compiled
\begin{document}
\maketitle
In this chapter, we review some basis theory of roots and weight. We will first recall the
general theory and compute explicitly the examples for $\SLn$/$\GLn$.
\section{Structure theory}
\subsection{The Cartan subalgebra}
First we need the notion of Cartan subalgebra
\begin{definition}
    For any Lie algebra $\mathfrak{g}$, a subalgebra $\mathfrak{h}$ of $\fg$ is said to be \textit{Cartan algebra} if it is
    \begin{itemize}
        \item $\fh$ is a nilpotent subalgebra.
        \item It is self normalizing. In particular, we have $\fh = \left\lbrace x \in \fg : [x,\fg] \subset \fg\right\rbrace$.
    \end{itemize}
\end{definition}
When $\fg$ is a semisimple Lie algebra, we have the following theorem
\begin{theorem}
    Let $\fg$ be a semisimple Lie algebra over an algebraically closed field $k$ of charateristic $0$ with a subalgebra $\fh$.
    Then $\fh$ is a Cartan subalgebra of $\fg$ if and only if it is a maximal toral subalgebra, i.e. is is maximal among all subalgebras
    contanining only semisimple elements.
\end{theorem}\todo{add citation.}
\subsection{Root space decomposition}
With respect to some choice of Cartan subalgebra, we have a root space decomposition. In particular, there is a finite set
$\Phi \subset \fh^{*}$ of linear forms on $H$, whose elements are called \textbf{roots}, such that
\[\fg = \fh \oplus \left(\bigoplus_{\alpha \in \Phi} \mathfrak{g}_\alpha\right),\]
where $\fg_\alpha = \left\lbrace x \in \fg: [h,x] = \alpha(h)x \forall h \in \fh\right\rbrace$ for any $\alpha \in \Phi$.
\subsection{A specific example: root space decomposition for $\mathfrak{sl}_n(\mathbb{R})$}
For the semisimple Lie algebra $\slnr$, a typical choice of the Cartan subalgebra is the set
\[\fh = \left\lbrace H= \begin{bmatrix}
        a_1    & 0      & \ldots & 0      \\
        0      & a_2    & \ldots & 0      \\
        \vdots & \vdots & \ddots & \vdots \\
        0      & 0      & \cdots & a_n
    \end{bmatrix}, a_1 + a_2+ \ldots + a_n = 0\right\rbrace\]
With respect to this Cartan subalgebra, we can define the linear function
\[L_i \colon \fh \to \mathbb{R}, \quad H \mapsto L_i(H)= a_i\]
Then the roots are given by $\alpha_{ij} :=L_i - L_j$ for distinct $i,j$. We have the root space decomposition for $\slnr$ as follows
\[\fg = \fh \oplus \left(\bigoplus\fg_{\alpha_{ij}}\right).\]
For the sake of brevity, we will denote $\alpha_{i,i+1}$ by $\alpha_i$ - these are called \textbf{simple roots}.
\subsection{Roots at group level}
Since the main object in this thesis is the Lie groups, we want to understand how the roots
behave at group level. The analog for the Cartan subalgebra is the maximal torus
\[T = \left\lbrace t= \begin{bmatrix}
        a_1    & 0      & \ldots & 0      \\
        0      & a_2    & \ldots & 0      \\
        \vdots & \vdots & \ddots & \vdots \\
        0      & 0      & \cdots & a_n
    \end{bmatrix} :  a_i \ne 0\right\rbrace,\]
Then $T$ acts on $\fg$ by conjugation. Explicitly, we can check that
\[\Ad(t)(E_{ij}) = t_it_j^{-1}E_{ij}\]
Therefore, at the group level, the character $\alpha_{ij}(\text{diag}(t_1,\ldots,t_n))=t_it_j^{-1}$
is a root whenever $i \ne j$. The set
\[\Delta = \left\lbrace \alpha_i\mid i =\overline{1,n}\right\rbrace\]
where
\[\alpha_i \colon T \mapsto \mathbb{R}, t \mapsto \dfrac{t_i}{t_{i+1}}\]
is the set of \textbf{simple roots}.
We can decompose the set of root into to disjoint subsets, namely
\[\Phi = \left\lbrace \alpha_{ij}, i \ne j\right\rbrace = \Phi_+ \coprod \Phi_{-}\]
where the set $\Phi_+$ comprises of $\alpha_{ij}$ for $i<j$ and the remaining roots are in $\Phi_{-}$. The former comprises of
\textbf{positive roots} while the latter contains \textbf{negative roots}. We have the following lemma
\begin{lemma}\label{linear-comb-of-roots}
    Each $\alpha \in \Phi$ can be written uniquely as a linear combination
    \[\alpha = m_1\alpha_1+\ldots+m_{d}\alpha_{d}\]
    with all $m_i \in \mathbb{Z}_{\ge 0}$ or $m_i \in \mathbb{Z}_{\le 0}$. If $\alpha \in \Phi_+$ then all $m_i \ge 0$, otherwise $m_i \le 0$ for all $i$.
\end{lemma}
\subsection{Weights}
\todo{define in term of Lie algebra}
Another class of linear forms that we are interested in are the \textbf{fundamental weights}. For each fundamental
weights $\lambda_i$, we define
\[\lambda_i \colon T \to \mathbb{R}, \quad\lambda_i(t) = a_1\ldots a_i\]
We have the following
\begin{lemma}\label{linear-comb-of-weights}
    We can write
    \[\lambda_i := r_1\alpha_1 + r_2\alpha_2+\ldots + r_d\alpha_d\]
    where $r_i$'s are rational number such that $r_i \ge 0$.\textcolor{red}{add proof}
\end{lemma}
\begin{example}
    When $n=3$, we have the following relations \textcolor{red}{add picture}
    \[\lambda_1 = \dfrac{2}{3}\alpha_1+\dfrac{1}{3}\alpha_2, \quad \lambda_2 = \dfrac{1}{3}\alpha_1+\dfrac{2}{3}\alpha_2\]
\end{example}
\begin{definition}
    A weight $\lambda$ is called \textbf{dominant} if it satisfies $\left\langle \lambda,\alpha^{\vee} \right\rangle \in \mathbb{Z}_{\ge 0}$ for all $\alpha$
\end{definition}
Clearly by lemma \ref{linear-comb-of-roots}, the weight $\lambda$ is dominant if and only if $\left\langle\lambda,\alpha_i^\vee\right\rangle$ for all
fundamental root $\alpha_i$. It is also clearly that the set of fundamental weight is given by addition of the fundamental weights, namely
\[\Lambda^+ := \left\lbrace c_1\lambda_1+\ldots+c_d\lambda_d \mid c_i \in \mathbb{Z}_{\ge 0}\right\rbrace\]
The set of dominant weights is denoted $\Lambda^+$. A weight $\lambda = \sum n_i \lambda_i$ is called strongly dominant if $n_i > 0$ for all $i$. One important example is the minimal strongly dominant weight given by
\[
    \rho = \sum \lambda_i
\]
This is called \textbf{Weyl vector} and is characterized in several ways:

\begin{enumerate}
    \item $\left\langle\rho,\alpha_i^\vee\right\rangle = 1$ for all $i$.
    \item
          \[
              \rho = \frac{1}{2} \sum_{\alpha \in \Phi^+} \alpha
          \]
\end{enumerate}

To prove the last equation we use the action of the Weyl group $W$. Let $\mu = \frac{1}{2} \sum \alpha$. Apply the simple reflection $s_i$ given by
\[
    s_i(x) = x - \left\langle x, \alpha_i^\vee\right\rangle \alpha_i
\]
We know that $s_i$ sends $\alpha_i$ to $-\alpha_i$ and permutes the other positive roots. So:
\[
    s_i(\mu) = \mu -  \left\langle x, \alpha_i^\vee\right\rangle  \alpha_i
\]
Therefore, $(\mu, \alpha_i) = \mu(h_i) = 1$ for all $i$. So, $\mu = \rho$.

Unlike lemma \ref{linear-comb-of-weights}, if we try to express the fundamental weights in term of the fundamental roots, we don't always
get positive coefficients. However, it is true that all the coefficients must be integer. In particular, we have
\[\alpha_j = \sum_{n_j}\lambda_j, \quad n_j \in \mathbb{Z}.\]
To put it another way, the root lattice $\mathbb{Z}\Delta$ is contained inside the weight lattice.
\subsection{Weyl group}
We only define the Weyl group explicitly for the group $\SLn$ or $\GLn$. It is a fact that the Weyl groups for
these two Lie groups are the same and equal to $W = S_n$ - the permutation group of $n$ letters.  We recall some basis
observation about this group
\begin{enumerate}
    \item Every $\sigma \in W$ can be written (non-uniquely) as a product of $w_{i_1} \cdots w_{i_k}$ for some integer $k$. Such a sequence is said to have length $k.$ If $k$ is the minimum, over all such writings, it is called the length of $\sigma$ and written $\ell(\sigma)$. Any expression of length $\ell(\sigma)$ for $\sigma$ is called a reduced expression.

    \item The group $S_n$ is generated by $S$ subject to the following two types of relations:
          \begin{itemize}
              \item (Reflection) $w_i^2=1$ for $i \in I$.
              \item (Braid relations) $w_i \, w_{i+1} \, w_i = w_{i+1}w_i w_{i+1}$ for $i = 1, \ldots, n-2$ and $w_i w_j = w_j w_i$ for $|j -i |\geq 2$.
          \end{itemize}
\end{enumerate}
Note that $W$ acts on $\Hom(H, \mathbb{R}^*)$ in the natural way: $w . \varphi(h) = \varphi(w^{-1} h)$. More explicitly,  $\sigma$ sens $\alpha:= \alpha_{ij}$ to $\alpha_{\sigma(i), \sigma(j)}$. Hence we find that
\[w_i \alpha_j = \begin{cases} - \alpha_i          & \mbox{if } i=j           \\
              \alpha_j            & \mbox{ if } |j - i | > 1 \\
              \alpha_i + \alpha_j & \mbox{ if } |j-i|=1\end{cases} \]
We of course also have an action of $W$ on the weights.  For example, one can verify that
\[ \begin{array}{lcr} s_i(\lambda_i) = \lambda_i - \alpha_i & \text{ and } & s_j(\lambda_i) = \lambda_i \mbox{ for } i \neq j \end{array}.\]
Recall the definition of Weyl vector $\rho$, we have the following generalized action of Weyl group of $\rho$:
\[ w \rho = \rho - \sum_{\alpha \in \Delta_{w^{-1}}} \alpha,\]
where \textcolor{red}{check this explicitly}
\[\Delta_{\sigma}:= \{ \alpha \in \Phi_+ \mid \sigma(\alpha) \in \Phi_- \}\]
\subsection{Cartan matrix}
We fix a set of simple roots $\Delta = \left\lbrace \alpha_1,\ldots,\alpha_d \right\rbrace$ is define to be the matrix
\[A = \begin{bmatrix} \left\langle \alpha_i,\alpha_j^\vee  \right\rangle
    \end{bmatrix}\]
If we let $a_{ij}=  \left\langle \alpha_i,\alpha_j^\vee\right\rangle$ then the Cartan matrix has the following simple properties:
\begin{lemma}
    \hfill
    \begin{itemize}
        \item For any $i$, we have $a_{ii}=2$.
        \item For any $i \ne j$, $a_{ij}$ is a non-positive integer, i.e. $a_{ij} \in \mathbb{Z}_{\le 0}$.
    \end{itemize}
\end{lemma}
We give an explicit example for $\slnr$, which has the root system $A_n$. The corresponding Cartan matrix is
\[A = \begin{bmatrix}
        2      & -1     & 0      & 0      & \ldots & 0      \\
        -1     & 2      & -1     & 0      & \ldots & 0      \\
        \vdots & \vdots & \vdots & \vdots & \ddots & \vdots \\
        0      & 0      & \ldots & -1     & 2      & -1     \\
        0      & 0      & \ldots & 0     & -1      & 2     \\
    \end{bmatrix}\]
\section{Parabolic subgroups}
\subsection{Parabolic sets and parabolic subalgebras}
to be added
\subsection{Parabolic subgroups}
to be added
\subsection{Langlands decomposition}
We fix a partition of $n$ as
\[n=n_1+n_2+\ldots+n_k\]
and consider the parabolic subgroup of this type, i.e. the subgroup
\[P_{n_1,\ldots, n_k} = \left\lbrace \begin{bmatrix}
        \mathfrak{m}_1 & \ast           & \ldots & \ast           \\
        0              & \mathfrak{m}_2 & \ldots & \ast           \\
        \vdots         & \vdots         & \ddots & \vdots         \\
        0              & 0              & \ldots & \mathfrak{m}_k
    \end{bmatrix} \right\rbrace\]
where $\mathfrak{m_i}$ is invertible of size $n_i \times n_i$.

This group can be factored as
\[P_{n_1,\ldots, n_k} =M_{n_1,\ldots, n_k}N_{n_1,\ldots, n_k}\]
where
\[N_{n_1,\ldots, n_k} = \left\lbrace \begin{bmatrix}
        I_1    & \ast   & \ldots & \ast   \\
        0      & I_2    & \ldots & \ast   \\
        \vdots & \vdots & \ddots & \vdots \\
        0      & 0      & \ldots & I_k
    \end{bmatrix} \right\rbrace \quad \left(I_k \text{ is the $n_k\times n_k$ identity matrix}\right)\]
and
\[M_{n_1,\ldots, n_k} = \left\lbrace \begin{bmatrix}
        \mathfrak{m}_1 & 0              & \ldots & 0              \\
        0              & \mathfrak{m}_2 & \ldots & 0              \\
        \vdots         & \vdots         & \ddots & \vdots         \\
        0              & 0              & \ldots & \mathfrak{m}_k
    \end{bmatrix} \right\rbrace\]

The subgroup $M_{n_1,\ldots, n_k}$ is called \textbf{Levi component}. We can further factor this subgroup as
\[M_{n_1,\ldots, n_k} = M'_{n_1,\ldots, n_k} \cdot A_{n_1,\ldots, n_k}\]
with $A_{n_1,\ldots, n_k}$ plays the role of the connected center of $M_{n_1,\ldots, n_k}$:
\[A_{n_1,\ldots, n_k} = \left\lbrace \begin{bmatrix}
        t_1I_1 & 0      & \ldots & 0      \\
        0      & t_2I_k & \ldots & 0      \\
        \vdots & \vdots & \ddots & \vdots \\
        0      & 0      & \ldots & t_kI_k
    \end{bmatrix} : t_i \ne 0\right\rbrace \]
and
\[M'_{n_1,\ldots, n_k} = \left\lbrace \begin{bmatrix}
        \mathfrak{m}'_1 & 0               & \ldots & 0               \\
        0               & \mathfrak{m}'_2 & \ldots & 0               \\
        \vdots          & \vdots          & \ddots & \vdots          \\
        0               & 0               & \ldots & \mathfrak{m}'_k
    \end{bmatrix} \right\rbrace,\]
where $\det(\mathfrak{m}'_i) = \pm 1$.
\begin{definition}
    For a given parabolic subgroup $P$, the factorization
    \[P = M_P \times A_P \times N_P\]
    as above is called \textbf{Langlands decomposition}.
\end{definition}
\end{document}_