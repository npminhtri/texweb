\documentclass[12pt]{article} % \documentclass{} is the first command in any LaTeX code.  It is used to define what kind of document you are creating such as an article or a book, and begins the document preamble

\usepackage{amsmath} % \usepackage is a command that allows you to add functionality to your LaTeX code

\usepackage[papersize={216mm,330mm},tmargin=20mm,bmargin=20mm,lmargin=20mm,rmargin=20mm]{geometry}
\usepackage[english]{babel}
\usepackage[utf8]{inputenc}
\usepackage{amsmath,amssymb,mathabx,amsthm}%\for eqref
\usepackage{lscape}
\usepackage{graphicx}
\usepackage{tikz}\usetikzlibrary{arrows.meta,calc} %library tikz
\usepackage{subcaption}


\usepackage{pgfplots}
\pgfplotsset{compat=1.15}
\usepackage{mathrsfs}
\usetikzlibrary{arrows}

\usepackage{color,soul} %package for highlining
\usepackage[colorinlistoftodos]{todonotes}
\usepackage{fancyhdr}
\usepackage{hyperref} %creat hyperlink
\hypersetup{
    colorlinks=true,
    linkcolor=blue,
    filecolor=magenta,      
    urlcolor=cyan,
    pdftitle={Overleaf Example},
    pdfpagemode=FullScreen,
    } %set up a hyperlink to be in blue 
\newtheorem{definition}{Definition}[section]
\newtheorem{theorem}[definition]{Theorem}
\newtheorem{prop}[definition]{Proposition}
\newtheorem{lemma}[definition]{Lemma}
\newtheorem{example}[definition]{Example}

\newtheorem*{remark}{Remark}

\pagestyle{fancy}
\cfoot{\thepage} % this is for the page numbering
\setlength\parindent{0pt} % noindent for the whole document.
\renewcommand{\baselinestretch}{1.1} % increase the distance between line.

\DeclareMathOperator{\SLn}{\text{SL}_n(\mathbb{R})}
\DeclareMathOperator{\GLn}{\text{GL}_n(\mathbb{R})}
\DeclareMathOperator{\slnz}{SL_n(\mathbb{Z})}
\DeclareMathOperator{\glnz}{GL_n(\mathbb{Z})}
\DeclareMathOperator{\vol}{vol}
\DeclareMathOperator{\fg}{\mathfrak{g}}
\DeclareMathOperator{\fh}{\mathfrak{h}}
\DeclareMathOperator{\SO}{SO_n(\mathbb{R})}
\DeclareMathOperator{\slnr}{\mathfrak{sl}_n(\mathbb{R})}
\newcommand{\tpoint}[1]{\subsection{#1}}

\newcommand{\spoint}{\subsection{}}



\title{CHAPTER II :ROOTS AND WEIGHTS FOR $\SLn$} % Sets article title
\date{\today} % Sets date for date compiled
\begin{document}
\maketitle
In this chapter, we review some basis theory of roots and weight. We will first recall the 
general theory and compute explicitly the exammples for $\SLn$/$\GLn$.
\section{Structure theory}
\subsection{The Cartan subalgebra}
First we need the notion of Cartan subalgebra
\begin{definition}
    For any Lie algebra $\mathfrak{g}$, a subalgebra $\mathfrak{h}$ of $\fg$ is said to be \textit{Cartan algebra} if it is 
    \begin{itemize}
        \item $\fh$ is a nilpotent subalgebra.
        \item It is self normalizing. In particular, we have $\fh = \left\lbrace x \in \fg : [x,\fg] \subset \fg\right\rbrace$.
    \end{itemize}
\end{definition}
when $\fg$ is a semisimple Lie algebra, we have the following theorem
\begin{theorem}
    Let $\fg$ be a semisimple Lie algebra over an algebraically closed field $k$ of charateristic $0$ with a subalgebra $\fh$.
    Then $\fh$ is a Cartan subalgebra of $\fg$ if and only if it is a maximal toral subalgebra, i.e. is is maximal among all subalgebras
    contanining only semisimple elements. 
\end{theorem}\todo{add citation.}
\subsection{Root space decomposition}
With respect to some choice of Cartan subalgebra, we have a root space decomposition. In particular, there is a finite set 
$\Phi \subset \fh^{*}$ of linear forms on $H$, whose elements are called \textbf{roots}, such that 
\[\fg = \fh \oplus \left(\bigoplus_{\alpha \in \Phi} \mathfrak{g}_\alpha\right),\]
where $\fg_\alpha = \left\lbrace x \in \fg: [h,x] = \alpha(h)x \forall h \in \fh\right\rbrace$ for any $\alpha \in \Phi$. 
\subsection{A specific example: root space decomposition for $\mathfrak{sl}_n(\mathbb{R})$}
For the semisimple Lie algebra $\slnr$, a typical choice of the Cartan subalgebra is the set 
\[\fh = \left\lbrace H= \begin{bmatrix}
    a_1 &  0 & \ldots & 0 \\
    0 &  a_2 & \ldots & 0 \\
    \vdots & \vdots    &\ddots & \vdots\\
    0 & 0 & \cdots & a_n
\end{bmatrix}, a_1 + a_2+ \ldots + a_n = 0\right\rbrace\]
With respect to this Cartan subalgebra, we can define the linear function 
\[L_i \colon \fh \to \mathbb{R}, \quad H \mapsto L_i(H)= a_i\]
Then the roots are given by $\alpha_{ij} :=L_i - L_j$ for distinct $i,j$. We have the root space decomposition for $\slnr$ as follows
\[\fg = \fh \oplus \left(\bigoplus\fg_{\alpha_{ij}}\right).\]
For the sake of brevity, we will denote $\alpha_{i,i+1}$ by $\alpha_i$ - these are called \textbf{fundamental roots}.
\subsection{Roots at group level}
Since the main object in this thesis is the Lie groups, we want to understand how the roots 
behave at group level. The analog for the Cartan subalgebra is the maximal torus 
\[T = \left\lbrace \begin{bmatrix}
    a_1 &  0 & \ldots & 0 \\
    0 &  a_2 & \ldots & 0 \\
    \vdots & \vdots    &\ddots & \vdots\\
    0 & 0 & \cdots & a_n
\end{bmatrix} \right\rbrace \]
something
\end{document}