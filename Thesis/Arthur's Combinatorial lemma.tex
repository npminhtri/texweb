\documentclass[12pt]{article}
\usepackage{amsmath} % \usepackage is a command that allows you to add functionality to your LaTeX code
\usepackage[papersize={215mm,280mm},tmargin=25mm,bmargin=25mm,lmargin=25mm,rmargin=25mm]{geometry}

\usepackage[english]{babel}
\usepackage[utf8]{inputenc}

\usepackage{amsmath,amssymb,mathabx,amsthm}%\for eqref
\usepackage{mathabx}
\usepackage{lscape}
\usepackage{graphicx}
\usepackage{tikz}\usetikzlibrary{arrows.meta,calc} %library tikz
\usepackage{subcaption}
\usepackage[labelfont=bf]{caption}
\usepackage{float}
\usepackage{pgfplots}
\pgfplotsset{compat=1.15}
\usepackage{mathrsfs}
\usetikzlibrary{arrows}
\usepackage{pstricks}
\usepackage{multido}
\usepackage{pst-plot}
\usepackage{rank-2-roots}


\usepackage{color,soul} %package for highlining
\usepackage[colorinlistoftodos]{todonotes}
\usepackage{fancyhdr}
\usepackage{hyperref} %creat hyperlink
\hypersetup{
    colorlinks=true,
    linkcolor=blue,
    filecolor=magenta,      
    urlcolor=cyan,
    pdftitle={Overleaf Example},
    pdfpagemode=FullScreen,
    } %set up a hyperlink to be in blue 
\newtheorem{definition}{Definition}[section]
\newtheorem{theorem}[definition]{Theorem}
\newtheorem{prop}[definition]{Proposition}
\newtheorem{lemma}[definition]{Lemma}
\newtheorem{example}[definition]{Example}

\newtheorem*{remark}{Remark}
\lhead{}%clear the lead head
\linespread{1.25}



\pagestyle{fancy}
\cfoot{\thepage} % this is for the page numbering
\setlength\parindent{0pt} % noindent for the whole document.
 % increase the distance between line.

    
\DeclareMathOperator{\SLn}{\text{SL}_n(\mathbb{R})}
\DeclareMathOperator{\GLn}{\text{GL}_n(\mathbb{R})}
\DeclareMathOperator{\glnr}{\mathfrak{gl}_n(\mathbb{R})}
\DeclareMathOperator{\slnz}{SL_n(\mathbb{Z})}
\DeclareMathOperator{\glnz}{GL_n(\mathbb{Z})}
\DeclareMathOperator{\vol}{vol}
\DeclareMathOperator{\rk}{rank}
\DeclareMathOperator{\Hom}{Hom}
\DeclareMathOperator{\fg}{\mathfrak{g}}
\DeclareMathOperator{\fh}{\mathfrak{h}}
\DeclareMathOperator{\SOn}{SO_n(\mathbb{R})}
\DeclareMathOperator{\On}{O_n(\mathbb{R})}
\DeclareMathOperator{\slnr}{\mathfrak{sl}_n(\mathbb{R})}
\DeclareMathOperator{\Ad}{\text{Ad}}
\DeclareMathOperator{\SLR}{SL_2(\mathbb{R})}
\DeclareMathOperator{\slz}{SL_2(\mathbb{Z})}
\DeclareMathOperator{\sl2}{\mathfrak{sl}_2(\mathbb{R})}
\DeclareMathOperator{\SO2}{SO_2(\mathbb{R})}
\DeclareMathOperator{\uH}{\mathfrak{H}}
\DeclareMathOperator{\cpx}{\textbf{cp}(x)}

\title{A short note on Arthur's combinatorial lemma}

\begin{document}
\maketitle
% ...
\section{Some basis notations}
In this section, we introduce some notations for the geometry of roots and weights. It should be noted that
these lemma can be generalized to a more broad setting. For the sake of simplicity, we
let $G = \SLn$. The set of roots and weights of $\SLn$ is the corresponding roots and weight of the
Lie algebera $\slnr$. In particular, we have a Cartan subalgebra
\[\mathfrak{a}:= a =\left\lbrace  \begin{bmatrix}
        a_1    & 0      & \ldots & 0      \\
        0      & a_2    & \ldots & 0      \\
        \vdots & \vdots & \ddots & \vdots \\
        0      & 0      & \ldots & a_n
    \end{bmatrix}: a_1 + a_2 + \ldots + a_n = 0\right\rbrace\]
Let $A \subset G$ be the set of diagonal matrices. Consider the exponential map
\[e^X: = I+X+\dfrac{X^2}{2!}+\cdots = \sum_{i=0}^\infty \dfrac{X^i}{i!}\]
It can be checked easily that, for any $X \in \mathfrak{a}$, $e^X$ is an element of $A$. Recall that, for any $g \in G$,
we have the Isawasa decomposition
\[g = n(g)a(g)k(g),\quad n(g) \in N, a(g) \in A, k(g) \in \SOn\]
where $n$ is the set of unipotent of matrix. Now where define the map
\[H_0 \colon G \to \mathfrak{a}\]
satisfying
\[e^{H_0(g)} = a(g)\]
Any element of $\mathfrak{a}_0^\asterisk = \Hom(\mathfrak{a_0},\mathbb{R})$ of the form $\alpha_{ij}(a)=a_i-a_j$ is called a \textit{root}.
In our particular situtation, a root is called \textit{positive} if $i<j$, and \textit{negative} otherwise. It can be show that the set
\[\Delta_0 = \left\lbrace \alpha_{i,i+1}: 1 \le i \le n-1\right\rbrace \]
is a basis for $\mathfrak{a}_0^*$. These elements are called \textit{simple roots}. Using the theory of Killing form, the vector space $\mathfrak{a}_0^\ast$ can be endowed
with an inner product $(\cdot,\cdot)$. Using this inner product, we define a pairing
\[\left\langle \cdot,\cdot \right\rangle \colon \mathfrak{a}^\ast_0 \times \mathfrak{a}_0 \to \mathbb{R} \]
given by
\[\left\langle \omega,\alpha^\vee \right\rangle:= \dfrac{2(\omega,\alpha)}{(\alpha,\alpha)}\]
We then have the notation
\[\widehat{\Delta}_0 = \{\omega_i \in a_0^\ast:\left\langle \omega_i,\alpha_{i}^\vee \right\rangle= \delta_{ij}  \}\]
The set $\widehat{\Delta}_0$ is referred as the set of \textit{fundamental} weight. In a more general settings, they arise from the set of
minimal parabolic subgroup $B= P_0$ - the set of upper triangular matrices in $\SLn$.

Generally speaking, given a parabolic subgroup $P:= P_J$ of the matrices that
stabilizes the flag
\[0 \subset E_{i_1} \subset \ldots \subset E_{i_k} \subset \mathbb{R}^n\]
where $I\setminus J = \{i_1,\ldots,i_k\}$, we define the following sets
\[\Delta^G_P = \Delta_P:= \{\alpha_i \in \Delta_0: i \notin J\}\]
and
\[\Delta^P_0 := \left\{\alpha_i \in \Delta_0: i \in J\right\}\]
Similarly we can define the corresponding set of weights as follows
\[\widehat{\Delta}^P_0:= \left\{\omega_i \in \widehat{\Delta}_0: i \notin J\right\} \quad \text{ and } \quad \widehat{\Delta}_P:= \left\{\omega_i \in \widehat{\Delta}_0: i \in J\right\}\]
\section{Arthur's partition lemma}
Given these notatition, we can prove the following combinatorial lemma of Arthur. This lemma plays a keyrole in his theory of
truncation of automorphic forms.
\begin{lemma}\label{partition}
    We have the identity holds for two standard parabolic subgroups $P\subset Q$.
    \[\sum_{ Q\subset G }\phi^{Q}_P(H)\tau_Q(H)=1\]
\end{lemma}
The goal of this note is to explain the lemma \ref{partition} and give an intuitively geometric description of
how the lemma works. First we need to understand the notations in the theorem. Using the same notations in the previous section, we define the following characteristic
functions: The function $\tau_P$ is the characteristic function over the set
\[\{H \in \mathfrak{a}_0: \alpha(H)> 0 \quad \forall \alpha \in \Delta_P\}\]
and the "dual" characteristic function $\widehat{\tau}_P$ is also a characteristic function over
\[\{H \in \mathfrak{a}_0: \omega(H)>0 \quad \forall \omega \in \widehat{\Delta}_P \}\]
The function $\phi^Q_P$ is defined to be the alternating sum
\[\phi^Q_P(H) := \sum_{S \subset Q}(-1)^{\#\Delta^Q_S}\widehat{\tau}^Q_S\]
In the case $G = \text{SL}_3(\mathbb{R})$ and $P = P_0$, the sum in the lemma \ref{partition} can be written as
\[1 = \phi^G_0(H)\tau^G_G(H)+\phi^1_0(H)\tau_1(H)+\phi^2_0(H)\tau_2(H)+\phi_0^0(H)\tau_0(H)\]
We analyze each summand as follows
\begin{itemize}
    \item $\phi^G_0(H)\tau^G_G(H) = \phi^G_0(H)$: By definition, we have \[\phi^G_0 = 1-\widehat{\tau}^1_0-\widehat{\tau}^2_0+\widehat{\tau}^G_0\]
          Using picture, we can illustrate the images of the above functions the first partition 
          on the leftmost column of the picture \ref{Arthur's combinatorial lemma}
    \item $\phi^1_0(H)\tau_1(H):$ Similarly, we have 
    \[\phi^1_0\tau_1 = \tau_1-\widehat{\tau}^1_0\tau_1\]
\end{itemize}
\begin{center}
            \includegraphics{alternate sum.png}
            \label{Arthur's combinatorial lemma}
          \end{center}
\end{document}

