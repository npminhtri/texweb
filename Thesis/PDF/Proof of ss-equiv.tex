\documentclass[12pt]{article} % \documentclass{} is the first command in any LaTeX code.  It is used to define what kind of document you are creating such as an article or a book, and begins the document preamble

\usepackage{amsmath} % \usepackage is a command that allows you to add functionality to your LaTeX code

\usepackage[papersize={216mm,330mm},tmargin=20mm,bmargin=20mm,lmargin=20mm,rmargin=20mm]{geometry}
\usepackage[english]{babel}
\usepackage[utf8]{inputenc}
\usepackage{amsmath,amssymb,mathabx,amsthm}%\for eqref
\usepackage{lscape}
\usepackage{graphicx}
\usepackage{tikz}\usetikzlibrary{arrows.meta,calc} %library tikz
\usepackage{subcaption}


\usepackage{pgfplots}
\pgfplotsset{compat=1.15}
\usepackage{mathrsfs}
\usetikzlibrary{arrows}

\usepackage{color,soul} %package for highlining
\usepackage[colorinlistoftodos]{todonotes}
\usepackage{fancyhdr}
\usepackage{hyperref} %creat hyperlink
\hypersetup{
    colorlinks=true,
    linkcolor=blue,
    filecolor=magenta,      
    urlcolor=cyan,
    pdftitle={Overleaf Example},
    pdfpagemode=FullScreen,
    } %set up a hyperlink to be in blue 
\newtheorem{theorem}{Theorem}
\newtheorem{definition}{Definition}[section]
\newtheorem{prop}[definition]{Proposition}
\newtheorem{lemma}[definition]{Lemma}
\newtheorem*{remark}{Remark}

\pagestyle{fancy}
\cfoot{\thepage} % this is for the page numbering
\setlength\parindent{0pt} % noindent for the whole document.
\renewcommand{\baselinestretch}{1.1} % increase the distance between line.

\DeclareMathOperator{\SLn}{\text{SL}_n(\mathbb{R})}
\DeclareMathOperator{\slnz}{SL_n(\mathbb{Z})}
\DeclareMathOperator{\vol}{vol}

\DeclareMathOperator{\SO}{SO_n(\mathbb{R})}



\title{CHAPTER  :SEMI-STABLE LATTICE IN HIGHER RANK} % Sets article title
\date{\today} % Sets date for date compiled
\begin{document}
\maketitle % creates a title using the information in the preamble (title, author, date)
In this chapter, we will establish the notion of semi-stable lattice. Heuristically,
this is the lattice that achieve all the successive minima at the same time, see \cite{}.

We will provide two different definitions of semi -stable lattice: one is geometric - which follows Grayson's idea of utilizing
the canonical plot, and one is purely algebraic, which make use of the maximal standard parabolic subgroups.
The toy model will be the moduli space of 2-dimensional lattice, which is essential the upper half plane in the complex field.
At the end, we will show that the two definitions coincide.
\section{Lattices and semi-stable lattice in higher rank}
For each $z$ with $\Im(z)>0$, we can attach to $z$ a lattice structure $L_z = \mathbb{Z}z \oplus \mathbb{Z}$. Roughly speaking
a lattice is a discrete  subgroup that is generated by a $k-$ basis of the $k$-space $V$.
In particular, we will only work with the real vector space $V$. Grayson works with lattice over a ring of algebraic integers, but we will restrict to
just the lattice that has the underlying structure as a $\mathbb{Z}-$ module.


The precise definition
of a lattice is as follows:
\begin{definition}[\label = Euclidean $\mathbb{Z}$-lattices]
    Let $L$ be a finitely generated $\mathbb{Z}$-module. In particular, it is a free $\mathbb{Z}$-module
    of finite rank. Suppose that $P$ is endowed with a real-valued symmetric positive definite bilinear form, called $Q$.
    Then the space $L_\mathbb{R}= L \otimes_\mathbb{Z} \mathbb{R}$ equipped with the bilinear form $Q$ forms a real
    inner product space. We will call  the pair $(L,Q)$ a \textbf{Euclidean $\mathbb{Z}$-lattice}.
\end{definition}
\todo{add proof showing $L$ is a lattice in the second definition}

If there is no further confusion, we can just denote a Euclidean lattice by $L$, without specifying the bilinear form
$Q$. The lattice $l$ determines a full-rank lattice inside $L_\mathbb{R}$, namely, the rank
of the lattice $L$ is equal to the dimension of $L_\mathbb{R}$. We first recall the definition of discrete subgroup
\begin{definition}
    Let $V$ be a finite-dimensional vector space over $\mathbb{R}$, endowed with the natural topology. A subgroup $L$ of the additive group underlying the vector space $V$ is said to be \textit{discrete} if each point $y$ in $L$ has a neighbourhood in $V$ whose intersection with $L$ is $\{y\}$ or, equivalently, if, given a bounded set $C$ in $V$, the set $C \cap L$ is finite.
\end{definition}
Thus, using the following Proposition, $L$ has a structure of a discrete subgroup $V = L_\mathbb{R}$.
\begin{prop}
    Given a finite-dimensional vector space $V$ over $\mathbb{R}$, let $L$ be a subgroup of the additive group $V$, and let $m$ be the dimension of the $\mathbb{R}$-span of $L$ in $V$. Then $L$ is a discrete subgroup if and only if $L$ is a free abelian group of rank $m$.
\end{prop}
A proof can be found in \cite{}.
We now can define the notion of covolume of a lattice:
\begin{definition}[\label = Volume]
    Let's assume that $L$ is a full-rank lattice and has a basis
    \[L = \mathbb{Z}l_1 \oplus \ldots \oplus\mathbb{Z}l_n\]
    Then the volume of this lattice is defined to be the volume of the fundamental
    parallepiped. In particular, let $\left\lbrace e_i\right\rbrace$ be any orthonormal
    basis of the vector space $V = L_\mathbb{R}$. Then
    \[\vol(L) := \left|\det Q(l_i,e_j)\right|\]
\end{definition}
However, for the sake of computation, we also usually adopt another definition of the lattice.
In particular, we view lattice as a free $\mathbb{Z}-$ module of rank $n$ that is isomorphic
to $\mathbb{R}^n$ via base changing.
In more detail
\begin{definition}

\end{definition}
\end{document}