\documentclass[12pt]{article} % \documentclass{} is the first command in any LaTeX code.  It is used to define what kind of document you are creating such as an article or a book, and begins the document preamble

\usepackage{amsmath} % \usepackage is a command that allows you to add functionality to your LaTeX code

\usepackage[papersize={216mm,330mm},tmargin=20mm,bmargin=20mm,lmargin=20mm,rmargin=20mm]{geometry}
\usepackage[english]{babel}
\usepackage[utf8]{inputenc}
\usepackage{amsmath,amssymb,mathabx,amsthm}%\for eqref
\usepackage{lscape}
\usepackage{graphicx}
\usepackage{tikz}\usetikzlibrary{arrows.meta,calc} %library tikz
\usepackage{subcaption}


\usepackage{pgfplots}
\pgfplotsset{compat=1.15}
\usepackage{mathrsfs}
\usetikzlibrary{arrows}

\usepackage{color,soul} %package for highlining
\usepackage[colorinlistoftodos]{todonotes}
\usepackage{fancyhdr}
\usepackage{hyperref} %creat hyperlink
\hypersetup{
    colorlinks=true,
    linkcolor=blue,
    filecolor=magenta,      
    urlcolor=cyan,
    pdftitle={Overleaf Example},
    pdfpagemode=FullScreen,
    } %set up a hyperlink to be in blue 
\newtheorem{theorem}{Theorem}
\newtheorem{definition}{Definition}[section]
\newtheorem{prop}{Proposition}[section]
\newtheorem{lemma}{Lemma}[section]
\newtheorem*{remark}{Remark}

\pagestyle{fancy}
\cfoot{\thepage} % this is for the page numbering
\setlength\parindent{0pt} % noindent for the whole document.
\renewcommand{\baselinestretch}{1.1} % increase the distance between line.

\DeclareMathOperator{\SLn}{\text{SL}_n(\mathbb{R})}
\DeclareMathOperator{\slnz}{SL_n(\mathbb{Z})}

\DeclareMathOperator{\SO}{SO_n(\mathbb{R})}



\title{CHAPTER  :SEMI-STABLE LATTICE IN HIGHER RANK} % Sets article title
\date{\today} % Sets date for date compiled
\begin{document}
\maketitle % creates a title using the information in the preamble (title, author, date)
In this chapter, we will establish the notion of semi-stable lattice. Heuristically,
this is the lattice that achieve all the successive minima at the same time, see \cite{}.

We will provide two different definitions of semi -stable lattice: one is geometric - which follows Grayson's idea of utilizing
the canonical plot, and one is purely algebraic, which make use of the maximal standard parabolic subgroups.
The toy model will be the moduli space of 2-dimensional lattice, which is essential the upper half plane in the complex field.
At the end, we will show that the two definitions coincide.
\section{Lattices and semi-stable lattice in higher rank}
For each $z$ with $\Im(z)>0$, we can attach to $z$ a lattice structure $L_z = \mathbb{Z}z \oplus \mathbb{Z}$. Roughly speaking
a lattice is a discrete  subgroup that is generated by a $k-$ basis of the $k$-space $V$.
In particular, we will only work with the real vector space $V$. Grayson works with lattice over a ring of algebraic integers, but we will restrict to
just the lattice that has the underlying structure as a $\mathbb{Z}-$ module.


The precise definition
of a lattice is as follows:
\begin{definition}

\end{definition}
\end{document}