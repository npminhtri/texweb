\chapter{Equivalence between Grayson's semistability and $\rho$-semistability}
\section{The equivalence between definitions of semi-stable lattices}
Recall that we have two distinct definitions of semi-stability, as defined in definition \ref{ss1} and definition \ref{ss2}. The following theorem asserts their equivalence:

\begin{theorem}\label{semi-stable-equiv}
    Let $x \in X_n = K \backslash \text{GL}_n(\mathbb{R})$ and $L_x$ be the corresponding Euclidean lattice. Then the following conditions on $x$ are equivalent

    \begin{enumerate}
        \item The bottom of the profile of $L_x$ is a line connect solely two points: the origin and $(n,0)$.
        \item The degree of instability of $x$ is nonnegative, namely, $\deg_{inst}(x) \ge 0$.
    \end{enumerate}
\end{theorem}
\subsection{A useful lemma}
\begin{lemma}\label{useful-lemma}
    There exists a bijection \label{bijection}
    \[    \text{GL}_n(\mathbb{Z})/(Q_i(\mathbb{Q}) \cap \text{GL}_n(\mathbb{Z})) \longleftrightarrow \left\lbrace \text{ sublattices of rank $i$ of $\mathbb{Z}^n$}\right\rbrace\]\end{lemma}
\begin{proof}
    We define the map from the collection of sublattices of rank $i$ to the cosets space as follows: For any sublattice $M \subset \mathbb{Z}^n$, there exists
    a basis of $M$, denoted by
    \[\mathfrak{C} = \left\lbrace v_1,v_2,\ldots, v_i \right\rbrace \]
    we can extend this basis to get a basis of $\mathbb{Z}^n$
    \[\mathfrak{C}_1 = \left\lbrace v_1,v_2,\ldots, v_n \right\rbrace \]
    In $\mathbb{Z}^n$ we have the standard basis $\mathfrak{B} = \left\lbrace e_1,e_2,\ldots,e_n\right\rbrace $. Clearly there
    exists a map $\gamma \in \glnz$ such that
    \[\gamma \cdot e_k = v_k \quad \forall k = 1,2,\ldots, n\]
    So we define the map
    \begin{align*}
        \varphi \colon \left\lbrace\text{sublattices of rank $i$ of $\mathbb{Z}^n$}\right\rbrace & \to \text{GL}_n(\mathbb{Z})/(Q_i(\mathbb{Q}) \cap \text{GL}_n(\mathbb{Z})) \\
        M                                                                                        & \mapsto [\gamma]
    \end{align*}
    where $[\gamma]$ denoted the equivalent class of $\gamma$ in the quotient space. This is a well-defined map. Indeed, assume that we have a different basis
    \[ \mathfrak{C}_2 = \left\lbrace w_1,\ldots, w_n \right\rbrace\]
    where \[\mathfrak{C}' = \left\lbrace w_1,\ldots, w_i \right\rbrace \]
    is a basis for sublattice $M$.  As above, there also exists $\gamma' \in \glnz$ such that
    \[\gamma' e_k = w_k \quad \forall k \]
    Since $\mathfrak{C}$ and $\mathfrak{C'}$ are basis for a sublattice of $\mathbb{Z}^n$ of $\rk i$, we can find an element $g \in \mathrm{GL}_i(\mathbb{Z})$ such that
    \[gv_k = w_k \quad\forall k \le i\]
    Similarly, there also exists $h \in \mathrm{GL}_{n-i}(\mathbb{Z})$ such that
    \[h v_k = w_k \quad \forall k \ge i,\]
    as $\mathbb{Z}^n/M$ is the quotient lattice, hence a lattice. In particular, we can immediately see that
    \[\gamma' \cdot (\gamma)^{-1}v_k=\gamma'e_k = w_k \quad \forall k\]
    Thus \[\gamma' \cdot (\gamma)^{-1} = \begin{bmatrix}
            g & \ast \\
            0 & h
        \end{bmatrix} \in Q_i(\mathbb{Q}) \cap \text{GL}_n(\mathbb{Z})\]
    This is exactly the desired result.
\end{proof}
\subsection{Proof of the main theorem}
Now we give a proof for theorem \ref{semi-stable-equiv}
\begin{proof}
    \hfill \\
    If we can prove there is a correspondence between
    $\gamma \in \text{GL}_n(\mathbb{Q})/Q_i(\mathbb{Q})$ and a sublattice of rank $i$ of $x$, then we are done.
    We first need a slight reduction - we identify the quotient $\text{GL}_n(\mathbb{Q})/Q_i(\mathbb{Q})$ with
    the quotient $\text{GL}_n(\mathbb{Z})/(Q_i(\mathbb{Q}) \cap \text{GL}_n(\mathbb{Z})) $. Now let $x$ be an arbitrary lattice of rank $n$.
    Lemma \ref{useful-lemma} generalizes in the obvious way for lattice $x = g\mathbb{Z}^n$ for some $g \in \text{GL}_n(\mathbb{R})$. Indeed, we just define
    the map
    \begin{align*}
        \phi_g \colon \left\lbrace\text{sublattices of rank $i$ of $x\mathbb{Z}^n$}\right\rbrace & \to \text{GL}_n(\mathbb{Z})/(Q_i(\mathbb{Q}) \cap \text{GL}_n(\mathbb{Z})) \\
        gM = g \bigoplus_{k=1}^i  \mathbb{Z}v_i                                                  & \mapsto [\gamma]
    \end{align*}
    where $\gamma e_k = v_k$ in $\mathbb{Z}^n$ for all $ k \le i$ and $\phi_g^{-1}([\gamma]) = g \bigoplus_{k=1}^i  \mathbb{Z}(\gamma \cdot e_i)$.

    Now apply lemma \ref{ss-equiv}, we have that $\deg_{\text{inst}}(x) \ge 0$ if and only if, for every maximal parabolic subgroup $Q\subset G$ and $\omega \in \widehat{\Delta}_Q$ we have \[\left\langle \omega,H_Q(x\delta)\right\rangle \ge 0\] for each $\delta \in G(\mathbb{Q})/Q_i(\mathbb{Q})$. Let $M = \phi_g([\delta])$. Note that for
    $\omega \in \widehat{\Delta}_Q$, we have
    \[\left\langle \omega, H_Q(x\delta)\right\rangle = \left\langle \omega, H_B(x\delta)\right\rangle = \dfrac{n}{2}\log(a_1\cdots a_i)\]
    by equation \ref{volume - H_P}. Hence
    \[\left\langle \omega, H_B(x\delta) \right\rangle = c\left\langle \rho_{Q_i}, H_B(x\gamma) \right\rangle = \dfrac{n}{2}\log(a_1a_2\ldots a_i) = c\vol(M)\]
    for some positive integer $c$. Using the remark as in section \ref{semi-stable-verified}, we have the desired conclusion.
\end{proof}

\section{Canonical pair and Canonical filtration}
\subsection{Equivalence between Canonical pair and canonical filtration}
We can further prove that, the equivalence between different notions of semistability
comes from the equivalence between the notion of canonical pair and the bottom of the
profile of a lattice - the canonical filtration. The main result is the following proposition
\begin{prop}\label{cp-equiv}
    Given an $x \in X_n = K\backslash G$. Then $(P,\delta)$ is the canonical pair for $x$ if and only if $P$
    is the stabilizer of the canonical filtration for the lattice $L_x = x\mathbb{Z}^n = x\delta \mathbb{Z}^n$.
\end{prop}
Recall that, from theorem \ref{Grayson's criterion} and lemma \ref{canonical-pair}, both canonical
pair and canonical filtration are characterized by two conditions. We will show that these two conditions are equivalent.
\subsection{Equivalence between chain condition and  condition \ref{extremal}}
Throughout this section, we fix an $x \in X_n$ and the corresponding lattice $L_x= x\mathbb{Z}^n$.
\begin{lemma}\label{increasing-slope}
    The the following conditions are equivalent for $x \in X_n$.
    \begin{enumerate}
        \item  $\left\langle \alpha, H_P(x\delta)\right\rangle <0$ for any $\alpha \in \Delta_P$ and for some $\delta \in G_\mathbb{Q}/P_\mathbb{Q}$.
        \item $L_x$ has a chain of lattices
              \[\mathcal{F}: 0 = L_0 \subset M_1 \subset M_2 \subset \cdots \subset M_{k-1} \subset M_k = L_x \]
              such that $\mu(M_i/M_{i-1}) <\mu(M_{i+1}/M_i)$ for all $i$.
    \end{enumerate}
\end{lemma}
\begin{remark}
    We also attach to the flag
    \[\mathcal{F}\otimes \mathbb{R} \colon L_0 \subset M_1 \otimes \mathbb{R} \subset M_2\otimes \mathbb{R} \subset \cdots \subset M_{k-1}\otimes \mathbb{R} \subset M_k\otimes \mathbb{R} =  \mathbb{R}^n \]
    a rational standard parabolic subgroup $P_\mathbb{Q}$.
\end{remark}
\begin{proof}

    We first prove (1) implies (2). Assume that $(P,\delta)$ is the canonical pair for $x$ and we further assume that the standard parabolic
    $P$ in the canonical pair is of type $(n_1,\ldots, n_k)$ as defined in subsection \ref{parabolic I}, and set
    \[d_i = n_1+n_2+\ldots +n_i\]
    In particular, the $A_P$-coordinate of $x\delta$ in the \hyperref[P-horospherical]{$P$-horospherical decomposition}
    \begin{equation}
    a_P(x\delta) = \begin{bmatrix}
            a_1I_{n_1} & 0          & \ldots & 0          \\
            0          & a_2I_{n_2} & \ldots & 0          \\
            \vdots     & \vdots     & \vdots & \vdots     \\
            0          & 0          & \ldots & a_kI_{n_k}
        \end{bmatrix} \label{A_P coor}
    \end{equation}

    Then from the chain of standard lattice
    \[0 \subset \mathbb{Z}^{d_1} \subset \mathbb{Z}^{d_2} \subset \cdots \subset \mathbb{Z}^{d_k}\]
    we obtain a chain of sublattices of $L_x$ as follows
    \[0 \subset M_1 \subset M_2 \subset \cdots \subset M_k\]
    where $$M_i := \bigoplus_{m=1}^{d_i}\mathbb{Z}x\delta \cdot e_m$$.

    Note that for $M_k = x\delta\mathbb{Z}^{d_k} = x\mathbb{Z}^n = L_x$.

    From the $P$-horospherical coordinate \ref{P-horospherical}
    \[x\delta = km_P(x\delta)a_P(x\delta)n_P(x\delta),\]
    where $k \in \SOn$, $n_P(x\delta) \in N_P$, $m_p(x\delta) \in M_P$ as in section \ref{P-horospherical} and $a_P(x)$ is defined as in equation \eqref{A_P coor}
    \[\vol(M_{i}/M_{i-1}) = \left| km_P(x)a_P(x)n_P(x) \bigwedge_{j=d_i+1}^{d_{i+1}}e_j\right| = \left|a_P(x)\bigwedge_{j=d_i+1}^{d_{i+1}}e_j\right| = a_i^{n_i}.\]
    The second equality is because  $k\in \SOn$ preserves the length, $n_p(x\delta)$ stabilizes the flag $0 \subset \mathbb{Z}^{d_1} \subset \mathbb{Z}^{d_2} \subset \cdots \subset \mathbb{Z}^{d_k}$ and $M_P$ is semisimple, so the $m_p(x\delta)$ acts trivially on 
    the length function. 
    Thus,
    \begin{align*}
        \mu(M_i/M_{i-1})-\mu(M_{i+1}/M_i) & = \dfrac{\log\vol(M_i/M_{i-1})}{n_i}-\dfrac{\log\vol(M_{i+1}/M_i)}{n_{i+1}} \\&=\log(a_i) - \log(a_{i+1})
    \end{align*}
    \textbf{Claim:}
    \begin{equation}
    H_P(x\delta ) =m_1\lambda^\vee_{d_1}+m_2\lambda^\vee_{d_2}+\ldots+m_k\lambda^\vee_{d_k} \label{hp-in-weights}
    where
    \end{equation}
    \[m_i = \log(a_i) - \log(a_{i+1}) \]
    If we can prove this then
    \[m_i = \left\langle \alpha_{d_i}, H_P(x\delta)\right\rangle < 0 \Leftrightarrow \mu(M_i/M_{i-1})-\mu(M_{i+1}/M_i) <0\]
    By the definition of \hyperref[H-function]{($H_P$-function)}, we know that 
    \[H_P(x\delta) = \log a_P(x\delta) 
    = \begin{bmatrix}
       \log(a_1)I_{n_1} & 0          & \ldots & 0          \\
            0          & \log(a_2)I_{n_2} & \ldots & 0          \\
            \vdots     & \vdots     & \vdots & \vdots     \\
            0          & 0          & \ldots & \log(a_k)I_{n_k} 
    \end{bmatrix}\]
    The set $(\widehat{\Delta}_P)^\vee =\{\lambda_{d_i}\}$ is a basis for $\mathfrak{a}_P$, which clearly yields the expression \eqref{hp-in-weights}. Thus we just need to 
    evaluate the value for $m_i$, but 
    \[m_i = \left\langle \alpha_{d_i}, H_P(x\delta) \right\rangle = \log(a_i) - \log(a_{i+1})\]
    This is the desired result.    
\end{proof}
\begin{remark}
    From the proof the lemma \ref{increasing-slope}, we explicitly construct an equivalence
    \[xG_\mathbb{Q}/P_\mathbb{Q} \longleftrightarrow \{\mbox{ flag of sublattices of $x\mathbb{Z}^n$ of the same type of partition type of $P$}\}\]

\end{remark}
\subsection{$P$-semistability and $M_P$-semistability}
Before proving the remaining equivalence, we need the following lemma
\begin{lemma}\label{P&M_P ss}
    Let $P$ be a standard parabolic subgroup of $\GLn$. For $x \in X = K\backslash \GLn$, we let $m:= pr_{M_P}(x)$, the projection of $x$ on the
    space $X_{M_P}:= M_P/M_P \cap K$. Then
    \[\deg_{inst}^P(x) = \deg_{inst}^{M_P}(m).\]
    In particular, $x$ is $P$-semi-stable if and only if its projection $m$ is $M_P$-semi-stable.
\end{lemma}
\begin{proof}
    The identity follows essentially from the definition:
    \[\deg_{inst}^P(x) = \min_{R \subset P, \delta \in P_\mathbb{Q}/R_\mathbb{Q}}\left\langle \rho_R^P, H_R(x\delta)\right\rangle \quad \text{ and } \quad \deg_{inst}^{M_P}(m) = \min_{\ast R \subset M_P, \overline{\delta} \in {M_P,}_\mathbb{Q}/R_\mathbb{Q}}\left\langle \rho_R^P, H_R(m\overline{\delta})\right\rangle \]
    where $R$ ranges over all standard parabolic subgroup of $P$. Note that $\ast R = R \cap M_P$ is also a standard parabolic subgroup of $M_P$. Moreover, the map
    \begin{align*}
        \text{pr}_{M_P} \colon  P_\mathbb{Q}/R_\mathbb{Q} & \to  M_{P,\mathbb{Q}}/\ast R_\mathbb{Q} \\
        \delta                                            & \mapsto \overline{\delta}
    \end{align*}
    is a bijection by theorem \ref{M_P - P - bijection}.    This implies that one takes the minimum over the same set in evaluating the degree of $P$-instability and $M_P$-instability.
    By \ref{H_P-decomp}, For any $R \subset P$ a standard parabolic subgroup, we have
    \[H_R(x\delta) = H_P(x\delta)+ H_{\ast R}(\text{pr}_{M_P}(x\delta)) = H_{P}(x\delta)+ H_{\ast R}(m\overline{\delta})\]
    Note that, by definition, we have
    \[\rho^P_R =  \rho(P)-\rho(R)\]
    which vanishing identically on $\mathfrak{a}_P$. Thus 
    \begin{align*}
        \deg_{inst}^P(x) &= \min_{R \subset P, \delta \in P_\mathbb{Q}/R_\mathbb{Q}}\left\langle \rho_R^P, H_R(x\delta)\right\rangle \\&= \min_{R \subset P, \delta \in P_\mathbb{Q}/R_\mathbb{Q}}\left\langle \rho_R^P, H_{P}(x)+ H_{\ast R}(m\overline{\delta})\right\rangle \\&= \min_{\ast R \subset M_P, \overline{\delta} \in {M_P,}_\mathbb{Q}/R_\mathbb{Q}}\left\langle \rho_R^P, H_{P}(x)+ H_{\ast R}(m\overline{\delta})\right\rangle \\&= \deg_{inst}^{M_P}(m)
    \end{align*}
    This is what we want.
\end{proof}
\subsection{Equivalence between increasing chain condition and condition \ref{destabilizing}}
\begin{lemma}\label{ss-slope}
    Fixed the notations as in lemma \ref{increasing-slope}, the following conditions are equivalent
    \begin{enumerate}
        \item $M_i/M_{i-1}$ is semi-stable for all $i$, where the flag
              \[\mathcal{F}: 0  \subset M_1 \subset M_2 \subset \cdots \subset M_{k-1} \subset M_k = L_x \]
              corresponds to $x\delta$.
        \item $x\delta$ is $P-$semi-stable for some $\delta \in G_\mathbb{Q}/P_\mathbb{Q}$. It is also clear
    \end{enumerate}
\end{lemma}
\begin{proof}
By lemma \ref{P&M_P ss}, the condition $x\delta$ is $P-$semi-stable is equivalent to $m\overline{\delta}$ is $M_P-$semi-stable. Observe that, if $P$ is of type 
$(n_1,\ldots,n_k)$, then 
\[m\overline{\delta} = \begin{bmatrix}
     A_1 & 0          & \ldots & 0          \\
            0          & A_2 & \ldots & 0          \\
            \vdots     & \vdots     & \vdots & \vdots     \\
            0          & 0          & \ldots & A_k
\end{bmatrix}\]
where $A_i$ is a matrix of size $n_i \times n_i$ with determinant $\pm 1$. We can then construct a flag of lattices 
\[\mathcal{F'} \colon 0 \subset M'_1 \subset M'_2 \ldots \subset M'_k = L'_x\]
where $L'_x$ is just the normalization of lattice $L_x$, i.e. we rescale $L_x$ to get a similar lattice $L'_x$ of volume $1$. The sublattice $M'_i$ is constructed 
by setting
\[M_i' := \bigoplus_{m=1}^{d_i}\mathbb{Z}m\overline{\delta}\cdot e_m \]
It is clear that $M'_i/M'_{i-1} \cong A_{i}\mathbb{Z}^{n_i}$. It is also clear that, $m\overline{\delta}$ is semi-stable if and only if each quotient $M'_i/M_{i-1}$ is semi-stable.
Rescaling does not change the semistability, so we can rescale the flag $\mathcal{F'}$ by a suitable factor to get the desired flag $\mathcal{F}$
              \[\mathcal{F}: 0  \subset M_1 \subset M_2 \subset \cdots \subset M_{k-1} \subset M_k = L_x \]
where  $M_i/M_{i-1}$ is semi-stable for all $i$. 
\subsection{Proof of proposition \ref{cp-equiv}}
By \cite[Subsection 4.3.4]{patnaik2025borel}, the two conditions in lemma \ref{canonical-pair} are sufficient to guarantee that $\text{cp}(x) =(P,\delta)$. Thus, by lemma \ref{ss-slope} and lemma \ref{increasing-slope}, $\text{cp}(x) = (P,\delta)$ if and only if
there exists a flag of sublattices 
\[\mathcal{F} \colon 0 = M_0 \subset M_1 \subset \ldots \subset M_k = L_x\]
 stabilized by $P$, such that 
\begin{itemize}
    \item $\mu(M_i/M_{i-1}) <\mu(M_{i+1}/M_i)$.
    \item $M_i/M_{i-1}$ is semi-stable for all $i$.
\end{itemize}
By theorem \ref{Grayson's criterion}, $\mathcal{F}$ satisfies the above conditions if and only if $\mathcal{F}$ is the canonical filtration for $L_x$.
\end{proof}